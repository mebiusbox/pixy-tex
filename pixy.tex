%------------------------------------------------------------
% Font
\usepackage[deluxe]{otf}
\usepackage[noto,noalphabet,unicode]{pxchfon}
% \usepackage[noalphabet,unicode]{pxchfon}
% \setminchofont{VL-PGothic-Regular.ttf}
% \setgothicfont[0]{VL-Gothic-Regular.ttf}
\usepackage[T1]{fontenc}
\usepackage{fourier}
\usepackage[scaled]{helvet}
\renewcommand{\ttdefault}{pcr} %Courier
% Layout
% \usepackage{geometry}
% \geometry{left=20mm,right=10mm,top=10mm,bottom=15mm}
% a4paper
% \setlength{\textwidth}{13cm}
% \setlength{\textheight}{22cm}
% \setlength{\leftmargin}{-1cm}
% \setlength{\topmargin}{-2cm}
\setlength{\textwidth}{50zw}
\setlength{\textheight}{72\baselineskip}
\setlength{\leftmargin}{-1truecm}
\setlength{\topmargin}{-2truecm}
\setlength{\oddsidemargin}{0cm}
\setlength{\evensidemargin}{0cm}
% a5paper
% \setlength{\textwidth}{32zw}
% \setlength{\textheight}{46\baselineskip}
% \setlength{\leftmargin}{0truecm}
% \setlength{\topmargin}{-3truecm}
% \setlength{\oddsidemargin}{-1truecm}
% \setlength{\evensidemargin}{-1truecm}
% etc
\setlength{\parindent}{0em}
\linespread{1.2} % 行間
%--------------------------------------------------------
\usepackage{subfiles}
%--------------------------------------------------------
% \pagestyle{empty} % ページ番号を表示しない
% ページ番号を -n- 形式に
% \def\ps@myplain{\let\@mkboth\@gobbletwo%
% \let\ps@jpl@in\ps@plain
% \let\@oddhead\empty
% \def\@oddfoot{\reset@font\hfil--\,\thepage\,--\hfil}%
% \let\@evenhead\@empty
% \let\@evenfoot\@oddfoot}
% \pagestyle{myplain}
%------------------------------------------------------------
\usepackage[bookmarks=true,bookmarksnumbered=true,colorlinks=true,linkcolor={blue},urlcolor={blue},hyperfootnotes=false,pdfborder={0,0,0},pdfpagemode=UseNone]{hyperref}
\usepackage{pxjahyper}
% \usepackage{footnote}
% \usepackage{titletoc}
% \usepackage{fancyhdr}
% \usepackage{enumerate}
% \usepackage{lastpage}
% \usepackage{url}
\usepackage{comment,dashrule}
\usepackage{amsmath,amssymb}
\usepackage{mathtools}
\usepackage{polynom}
\usepackage{ifthen}
\usepackage{bm}
\usepackage{stmaryrd}
\makeatletter
% \cdotsfill[#1]
% #1: 点同士の間隔
\newcommand{\cdotsfillsub}[1][.25zw]{\leavevmode\cleaders\hb@xt@#1{\hss$\cdot\m@th$\hss}\hfill\kern\z@\ignorespaces}
\newcommand{\cdotsfill}[1]{\noindent\cdotsfillsub #1 \cdotsfillsub\par}
\newcommand{\leaderfill}[1][0pt]{\noindent\hspace{#1}\cleaders\hbox{…………}\hfill \hspace{#1}\par}
\makeatother
\def\dashfill{\cleaders\hbox to 2em{-}\hfill}
\def\Tasuki#1#2#3#4#5#6#7{
\setbox0\hbox{
\unitlength 0.1in
\begin{picture}( 3.5000, 1.0000)( 0.0000, -1.0000)
\special{pn 8}%
\special{pa 0 0}%
\special{pa 350 100}%
\special{fp}%
\special{pa 0 100}%
\special{pa 350 0}%
\special{fp}%
\end{picture}%
}
\begin{array}{cccccl}
#1 & &#3 & \longrightarrow & #5 & \\[-1ex]
#2 & \raise1ex\box0 &#4 & \longrightarrow & #6 & (+ \\ \cline{1-5}
& & & & #7
\end{array}
}
%
\newcommand{\Rev}[1]{\frac{1}{#1}}
\newcommand{\Pdrv}[2]{\frac{\partial #1}{\partial #2}}
\newcommand{\Drv}[2]{\frac{d#1}{d#2}}
\newcommand{\Ddrv}[2]{\frac{\delta #1}{\delta #2}}
\newcommand{\PVEC}[1]{{\small\begin{pmatrix}#1\end{pmatrix}}}
\newcommand{\BVEC}[1]{{\small\begin{bmatrix}#1\end{bmatrix}}}
\newcommand{\PVECs}[1]{{\scriptsize\begin{pmatrix}#1\end{pmatrix}}}
\newcommand{\BVECs}[1]{{\scriptsize\begin{bmatrix}#1\end{bmatrix}}}
\newcommand{\PMAT}[2][]{{\small\begin{pmatrix}{#1}#2\end{pmatrix}}}
\newcommand{\BMAT}[2][]{{\small\begin{bmatrix}{#1}#2\end{bmatrix}}}
\newcommand{\DMAT}[2][]{{\small\left|\begin{matrix}{#1}#2\end{matrix}\right|}}
% \newcommand{\Vvec}[1]{\left(\begin{array}{c}#1\end{array}\right)}
% \newcommand{\Vmatrix}[2][]{\left(\begin{array}{#1}#2\end{array}\right)}
% \newcommand{\Dmatrix}[2][]{\left|\begin{array}{#1}#2\end{array}\right|}
\newcommand{\Inversion}[2][]{#1^{\!\mbox{\sf\tiny T}}}
\newcommand{\Vecbm}[1]{\mbox{\boldmath $#1$}}
\newcommand{\Va}{\Vecbm{a}}
%------------------------------------------------------------
% Color
\usepackage{color,soul}
\usepackage[x11names]{xcolor}
\usepackage{multicol}
\definecolor{colorthemeback}{RGB}{255,255,255}
\definecolor{colorthemefore}{RGB}{77,77,77}
\definecolor{colorthememain1}{RGB}{0,113,188}
\definecolor{colorthememain2}{RGB}{255,80,80}
\definecolor{colorthemesub1}{RGB}{242,242,242}
\definecolor{colorthemesub2}{RGB}{226,241,250}
\definecolor{colorthemesub3}{RGB}{255,234,234}
\definecolor{graphcolor}{RGB}{204,0,0}
\definecolor{graphsubcolor}{RGB}{0,80,255}
%------------------------------------------------------------
% Figure, Table, Graph
\usepackage{tikz}
\usetikzlibrary{calc,intersections,quotes,angles,shapes.misc,patterns,decorations.markings}
\usepackage{tkz-euclide}
\usepackage{booktabs}
\usepackage{colortbl}
\usepackage{here}
\graphicspath{{images/}{../images/}}
\usepackage{pgfplots}
\DeclareMathOperator{\CDF}{cdf}

\def\cdf(#1)(#2)(#3){0.5*(1+(erf((#1-#2)/(#3*sqrt(2)))))}%
\tikzset{
    declare function={
        normcdf(\x,\m,\s)=1/(1 + exp(-0.07056*((\x-\m)/\s)^3 - 1.5976*(\x-\m)/\s));
        normpdf(\x,\m,\s)=exp(-(\x-\m)^2/(2*\s)^2)/(sqrt(2*pi)*\s);
        gamma(\n)=(\n-1)!;
        beta(\a,\b)=(gamma(\a)*gamma(\b))/gamma(\a+\b);
        betapdf(\x,\a,\b)=(\x^(\a-1)*(1-\x)^(\b-1))/beta(\a,\b);
    }
}
% \pgfplotsset{compat=1.12}
\usepgfplotslibrary{fillbetween}
\pgfplotsset{compat=newest}
\pgfplotsset{grid style={dashed,gray}}
\usepackage{wrapfig}
\usepackage[hang,small,bf]{caption}
\usepackage[subrefformat=parens]{subcaption}
\captionsetup{compatibility=false}
%------------------------------------------------------------
% Frame, Box
\usepackage{framed}
\usepackage{mdframed}
\usepackage[many]{tcolorbox}
\usepackage{varwidth}
% 囲み(中央タイトル)
\newenvironment{STDBOX}[2][]{
    \ifthenelse{\equal{#2}{}}{
        \mdfsetup{
            frametitle={\colorbox{white}{\space#2\space}},
            innertopmargin=10pt,
            innerbottommargin=10pt,
            % frametitleaboveskip=-\ht\strutbox,
            frametitleaboveskip=-8pt,
            frametitlealignment=\center,
            #1
            }
    }{
        \mdfsetup{
        frametitle={\colorbox{white}{\space#2\space}},
        innertopmargin=0pt,
        innerbottommargin=10pt,
        % frametitleaboveskip=-\ht\strutbox,
        frametitleaboveskip=-8pt,
        frametitlealignment=\center,
        #1
        }
    }
    \begin{mdframed}%
    }
    {\end{mdframed}}

% \mdfdefinestyle{guidelinestyle}{%
%     linecolor=black,
%     linewidth=1pt,
%     frametitlerule=true,
%     frametitlefont=\sffamily\bfseries,
%     frametitlebackgroundcolor=gray!20,
%     innertopmargin=\topskip
% }
% \makeatletter
% \define@key{thmdef}{mdthm}[{}]{%
% \thmt@trytwice{\def\thmt@theoremdefiner{\mdtheorem[#1]}}{}}
% \makeatother
% \declaretheorem[mdthm={style=guidelinestyle},numberwithin=section]{theorem}

\mdfdefinestyle{DUALSIDELINEBOX}{%
  rightline=true,
  innerleftmargin=10,
  innerrightmargin=10,
  outerlinewidth=3pt,
  linewidth=3pt,
  topline=false,
  rightline=true,
  bottomline=false,
  skipabove=\topsep,
  skipbelow=\topsep
}

% http://tex.bootmath.com/how-to-create-highlight-boxes-in-latex.html
% \def\bracketcolor{red!75!black}
% \def\bracketwidth{3pt}
\def\bracketcolor{black}
\def\bracketwidth{.5pt}
\newtcolorbox{BRACKETBOX}[1][]{
  breakable,
  freelance,
  title=#1,
  colback=white,
  colbacktitle=white,
  coltitle=black,
  fonttitle=\bfseries,
  before skip=20pt plus 2pt minus 2pt,
  after skip=20pt plus 2pt minus 2pt,
  bottomrule=0pt,
  boxrule=0pt,
  colframe=white,
  overlay unbroken and first={
  \draw[\bracketcolor,line width=\bracketwidth]
    ([xshift=5pt]frame.north west) --
    (frame.north west) --
    (frame.south west);
  \draw[\bracketcolor,line width=\bracketwidth]
    ([xshift=-5pt]frame.north east) --
    (frame.north east) --
    (frame.south east);
  },
  overlay unbroken app={
  \draw[\bracketcolor,line width=\bracketwidth,line cap=rect]
    (frame.south west) --
    ([xshift=5pt]frame.south west);
  \draw[\bracketcolor,line width=\bracketwidth,line cap=rect]
    (frame.south east) --
    ([xshift=-5pt]frame.south east);
  },
  overlay middle and last={
  \draw[\bracketcolor,line width=\bracketwidth]
    (frame.north west) --
    (frame.south west);
  \draw[\bracketcolor,line width=\bracketwidth]
    (frame.north east) --
    (frame.south east);
  },
  overlay last app={
  \draw[\bracketcolor,line width=\bracketwidth,line cap=rect]
    (frame.south west) --
    ([xshift=5pt]frame.south west);
  \draw[\bracketcolor,line width=\bracketwidth,line cap=rect]
    (frame.south east) --
    ([xshift=-5pt]frame.south east);
  },
}

\newcommand{\SIMPLEBOX}[2]{%
  \newcounter{#1}
  \newenvironment{#1}[1][]{%
    \stepcounter{#1}%
    \mdfsetup{%
        frametitle={%
            \tikz[baseline=(current bounding box.east),outer sep=0pt]
            \node[anchor=east,rectangle,fill=white]
            {\strut \MakeUppercase#1~\csname the#1\endcsname\ifstrempty{##1}{}{:~##1}};}}%
    \mdfsetup{innertopmargin=-5pt,innerbottommargin=1em,linecolor=#2,%
        linewidth=0.5pt,topline=true,
        frametitleaboveskip=\dimexpr-\ht\strutbox\relax,}%
    \begin{mdframed}[]\relax%
    }{\end{mdframed}}%
}
% \SIMPLEBOX{definition}{red!60}
% \SIMPLEBOX{theorem}{black}
\SIMPLEBOX{EXERCISEBOX}{black!50}
\SIMPLEBOX{EXAMPLEBOX}{black!50}
\makeatletter
\newtcolorbox{COLORTHEMENOTE}[1]{
    breakable,
    before skip=20pt plus 2pt minus 2pt,
    after skip=20pt plus 2pt minus 2pt,
    boxrule=0.4pt,
    colframe=colorthemefore, colback=colorthemeback,
    fonttitle=\gtfamily\bfseries, title=#1
}
\newtcolorbox{COLORTHEMENOTEBLUE}[1]{
    breakable,
    before skip=20pt plus 2pt minus 2pt,
    after skip=20pt plus 2pt minus 2pt,
    boxrule=0.4pt,
    colframe=colorthememain1, colback=colorthemeback,
    fonttitle=\gtfamily\bfseries, title=#1
}
\newtcolorbox{COLORTHEMENOTERED}[1]{
    breakable,
    before skip=20pt plus 2pt minus 2pt,
    after skip=20pt plus 2pt minus 2pt,
    boxrule=0.4pt,
    colframe=colorthememain2, colback=colorthemeback,
    fonttitle=\gtfamily\bfseries, title=#1
}
\newtcbox{\colorthemebox}{
    boxrule=0.4pt, colframe=colorthemefore, colback=colorthemeback,
    top=0mm, bottom=0mm, left=0mm, right=0mm, on line, arc=0.5mm
}
\newtcbox{\colorthemebox@main}{
    boxrule=0.4pt, colframe=colorthememain1, colback=colorthemeback,
    top=0mm, bottom=0mm, left=0mm, right=0mm, on line, arc=0.5mm
}
\newtcbox{\colorthemebox@accent}{
    boxrule=0.4pt, colframe=colorthememain2, colback=colorthemeback,
    top=0mm, bottom=0mm, left=0mm, right=0mm, on line, arc=0.5mm
}
\newtcbox{\colorthemeboxinvert}{
    boxrule=0.4pt, colframe=colorthemeback, colback=colorthemefore,
    top=0mm, bottom=0mm, left=0mm, right=0mm, on line, arc=0.5mm
}
\newtcbox{\colorthemeboxinvert@main}{
    boxrule=0.4pt, colframe=colorthememain1, colback=colorthememain1,
    top=0mm, bottom=0mm, left=0mm, right=0mm, on line, arc=0.5mm
}
\newtcbox{\colorthemeboxinvert@accent}{
    boxrule=0.4pt, colframe=colorthememain2, colback=colorthememain2,
    top=0mm, bottom=0mm, left=0mm, right=0mm, on line, arc=0.5mm
}
\newtcbox{\colorthemebox@subgray}{
    boxrule=0.4pt, colframe=colorthemefore, colback=colorthemesub1,
    top=0mm, bottom=0mm, left=0mm, right=0mm, on line, arc=0.5mm
}
\newtcbox{\colorthemebox@subblue}{
    boxrule=0.4pt, colframe=colorthememain1, colback=colorthemesub2,
    top=0mm, bottom=0mm, left=0mm, right=0mm, on line, arc=0.5mm
}
\newtcbox{\colorthemebox@subred}{
    boxrule=0.4pt, colframe=colorthememain2, colback=colorthemesub3,
    top=0mm, bottom=0mm, left=0mm, right=0mm, on line, arc=0.5mm
}
\newtcbox{\colorthemebox@transparent}{
    boxrule=0.4pt, colframe=colorthemeback, colback=colorthemeback,
    top=0mm, bottom=0mm, left=0mm, right=0mm, on line, arc=0.5mm
}
\newcommand{\COLORTHEMEBOX}[2]{
    \begin{itemize}
        \item[\colorthemebox{\small #1}] #2
    \end{itemize}
}
\newcommand{\COLORTHEMEBOXBLUE}[2]{
    \begin{itemize}
        \item[\colorthemebox@main{\color{colorthememain1}\small #1}] #2
    \end{itemize}
}
\newcommand{\COLORTHEMEBOXRED}[2]{
    \begin{itemize}
        \item[\colorthemebox@accent{\color{colorthememain2}\small #1}] #2
    \end{itemize}
}
\newcommand{\COLORTHEMEBOXINVERT}[2]{
    \begin{itemize}
        \item[\colorthemeboxinvert{\color{colorthemeback}\small #1}] #2
    \end{itemize}
}
\newcommand{\COLORTHEMEBOXINVERTBLUE}[2]{
    \begin{itemize}
        \item[\colorthemeboxinvert@main{\color{colorthemeback}\small #1}] #2
    \end{itemize}
}
\newcommand{\COLORTHEMEBOXINVERTRED}[2]{
    \begin{itemize}
        \item[\colorthemeboxinvert@accent{\color{colorthemeback}\small #1}] #2
    \end{itemize}
}
\newcommand{\COLORTHEMEBOXSUBGRAY}[2]{
    \begin{itemize}
        \item[\colorthemebox@subgray{\color{colorthemefore}\small #1}] #2
    \end{itemize}
}
\newcommand{\COLORTHEMEBOXSUBBLUE}[2]{
    \begin{itemize}
        \item[\colorthemebox@subblue{\color{colorthememain1}\small #1}] #2
    \end{itemize}
}
\newcommand{\COLORTHEMEBOXSUBRED}[2]{
    \begin{itemize}
        \item[\colorthemebox@subred{\color{colorthememain2}\small #1}] #2
    \end{itemize}
}
\newcommand{\COLORTHEMEBOXTRANS}[2]{
    \begin{itemize}
        \item[\colorthemebox@transparent{\small #1}] #2
    \end{itemize}
}
\makeatother
%------------------------------------------------------------
% Code
\usepackage{listings,jlisting}
\usepackage{verbatim}
%------------------------------------------------------------
\newtcbox{\hlbox}[1][]{
    boxrule=0.4pt,
    boxsep=2pt,
    sharp corners,
    colframe=white,
    % colframe=gray!40,
    % colframe=black,
    colback=yellow,
    top=0mm,
    bottom=0mm,
    left=0mm,
    right=0mm,
    on line,
    #1
}
% \newtcbox{\HLBOX}[1][]{
%     equal height group=dashedbox,
%     nobeforeafter,
%     colback=yellow,
%     boxrule=0pt,
%     boxsep=0pt,
%     top=2pt,
%     bottom=2pt,
%     left=2pt,
%     right=2pt,
%     enhanced jigsaw,
%     borderline horizontal={1pt}{0pt}{dashed},
%     borderline vertical={1pt}{0pt}{dashed},
%     #1
% }
% http://otoya8bit.hatenablog.jp/entry/2013/11/28/153226
% \newcommand{\HL}[1]{\hl{\mbox{#1}}}
\newcommand{\HL}[1]{\hlbox{\mbox{\small\rmfamily#1}}}
\newcommand{\EM}[1]{\color{red!80!black}\textbf{#1}\color{black}}
\newcommand{\EMLIGHT}[1]{\color{red!80!black}#1\color{black}}
\definecolor{MSBlue}{rgb}{.204,.353,.541}
\definecolor{MSLightBlue}{rgb}{.31,.506,.741}
\newcommand{\EMSUB}[1]{\mbox{\hspace{.3zw}\color{MSLightBlue}\textbf{#1}\color{black}\hspace{.3zw}}}

\newtcbox{\embox}{%
    boxrule=0.4pt,%
    colframe=red!60!black,%
    colback=red!60!black,%
    top=0mm, bottom=0mm, left=0mm, right=0mm,%
    on line, arc=0.5mm%
}
\newcommand{\EMBOX}[1]{\embox{\small\color{white} #1\color{black}}}

\newtcbox{\emsubbox}{%
    boxrule=0.4pt,%
    colframe=colorthemesub1,%
    colback=colorthemesub1,%
    top=0mm, bottom=0mm, left=0mm, right=0mm,%
    on line, arc=0.5mm%
}
\newcommand{\EMSUBBOX}[1]{\mbox{\hspace{.3zw}\emsubbox{\small\bfseries\color{black} #1}\hspace{.3zw}}}
\newcommand{\EMCODE}[1]{\mbox{\hspace{.3zw}\emsubbox{\small\ttfamily\color{black} #1}\hspace{.3zw}}}
\newtcbox{\INLINE}{
    size=fbox, on line,
    colframe=black!60,
    colback=gray!10,
    boxrule=1pt,
    top=0.5mm,bottom=0.5mm,left=0.5mm,right=0.5mm,
    fontupper=\ttfamily\small
}
%------------------------------------------------------------
\newtcolorbox{REDBOX}{
    colback=red!5!white,
    colframe=red!75!black
}
\newtcolorbox{REDBOXT}[1]{
    colback=red!5!white,
    colframe=red!75!black,
    fonttitle=\bfseries,title=#1
}

\definecolor{MathColor}{rgb}{0.0,0.56,0.0}
\newtcolorbox{MATHBOX}[1][]{%
    % on line,
    nobeforeafter, math upper, tcbox raise base, enhanced,
    colframe=MathColor,
    colback=white,
    arc=4pt,
    boxrule=2pt,
    drop fuzzy shadow
}

% \newcommand{\EM}[1]{\color{red}\textbf{#1}\color{black}}
% \newcommand{\EM}[1]{\color{red}#1\color{white}}

% \lstset{basicstyle=\footnotesize\ttfamily,breaklines=true}
% \lstset{framextopmargin=50pt,frame=bottomline}
\renewcommand{\lstlistingname}{リスト}
% \lstset{%
%     % language=c++,%
%     % belowcaptionskip=1\baselineskip,%
%     % breaklines=true,%
%     % showstringspaces=false,%
%     basicstyle=\small\ttfamily,%
%     % keywordstyle=\bfseries\color{green},%
%     keywordstyle=\bfseries\color{blue},
%     keywordstyle={[2]\color{green}},
%     keywordstyle={[3]\color{red}},
%     otherkeywords={String,async,await,Task,var},
%     keywords=[2]{DatabaseField,DatabaseTable},
%     keywords=[3]{@},
%     commentstyle=\itshape\color{purple!40!black},%
%     identifierstyle=\color{blue},%
%     stringstyle=\color{orange},%
%     % numberstyle=\small,%
%     escapeinside={(*@}{@*)},%
%     % escapeinside={\%*}{*)},
%     % lineskip=-1.ex,%
%     captionpos=b,
% }
\lstset{
    % language=java,%
    language=c++,%
    tabsize=4,%
    belowcaptionskip=1\baselineskip,%
    breaklines=true,%
    showstringspaces=false,%
    basicstyle=\small\ttfamily,%
    % keywordstyle=\bfseries\color{blue},%
    keywordstyle=\color{blue},%
    keywordstyle={[2]\color{orange}},%
    keywordstyle={[3]\color{cyan}},%
    commentstyle=\itshape\color{green!60!black},%
    identifierstyle=\color{black},%
    stringstyle=\color{red!80!black},%
    numberstyle=\small,%
    % escapeinside={(*@}{@*)},%
    escapeinside={\%*}{*)},
    lineskip=-1.ex,%
    captionpos=b,
    framesep=0pt,
    framextopmargin=6pt,
    framexbottommargin=6pt,
    framerule=0.5pt,
    framexleftmargin=6pt,
    xleftmargin=6pt,
    % otherkeywords={String,async,await,Task,var},%
    % keywords=[2]{DatabaseField,DatabaseTable},%
    % keywords=[3]{@},%
%     numbers=left,
%     numberstyle=\small,
%     numbersep=16pt,
%     frame=single,
%     backgroundcolor=\color[gray]{0.95}
}
\lstnewenvironment{pre}[1][]
    {\vspace{1zw}\lstset{
        frame=lrtb,
        framerule=0.5pt,
        #1}}
    {}
\lstnewenvironment{code}[1][]
    {\vspace{1zw}\lstset{
        frame=lrtb,
        % numbers=left,
        backgroundcolor=\color[gray]{0.95},
        rulesepcolor=\color[gray]{0.95},
        rulecolor=\color[gray]{0.5},
        #1}}
    {}
\lstnewenvironment{source}[1][]
    {\vspace{1zw}\lstset{
        frame=lines,
        framerule=0.5pt,
        backgroundcolor=\color{LemonChiffon1},
        #1}}
    {}
% \makeatletter
% \def\lst@lettertrue{\let\lst@ifletter\iffalse}
% \makeatother
%------------------------------------------------------------
\makeatletter
\newdimen\fb@xsep
\newdimen\fb@xrule
\newcommand{\image}[5][H]{%
    \ifthenelse{\equal{#4}{}}{%
        \begin{figure}[#1]%
            \centering%
            \includegraphics[#2]{#3}%
        \end{figure}%
    }{%
        \begin{figure}[#1]%
            \centering%
            \includegraphics[#2]{#3}%
            \caption{#4}%
            \label{#5}%
        \end{figure}%
    }%
}
\newcommand{\imageH}[4][H]{\image[#1]{width=10cm}{#2}{#3}{#4}}
\newcommand{\imageL}[4][H]{\image[#1]{width=8cm}{#2}{#3}{#4}}
\newcommand{\imageM}[4][H]{\image[#1]{width=6cm}{#2}{#3}{#4}}
\newcommand{\imageS}[4][H]{\image[#1]{width=4cm}{#2}{#3}{#4}}
\newcommand{\frameimage}[5][H]{%
    \fb@xsep=\fboxsep%
    \fb@xrule=\fboxrule%
    \fboxsep=0pt%
    \fboxrule=0.5pt%
    \ifthenelse{\equal{#4}{}}{%
        \begin{figure}[#1]%
            \centering%
            \fbox{\includegraphics[#2]{#3}}%
        \end{figure}%
    }{%
        \begin{figure}[#1]%
            \centering%
            \fbox{\includegraphics[#2]{#3}}%
            \caption{#4}%
            \label{#5}%
        \end{figure}%
    }%
    \fboxsep=\fb@xsep%
    \fboxrule=\fb@xrule%
}
\newcommand{\frameimageH}[4][H]{\frameimage[#1]{width=10cm}{#2}{#3}{#4}}
\newcommand{\frameimageL}[4][H]{\frameimage[#1]{width=8cm}{#2}{#3}{#4}}
\newcommand{\frameimageM}[4][H]{\frameimage[#1]{width=6cm}{#2}{#3}{#4}}
\newcommand{\frameimageS}[4][H]{\frameimage[#1]{width=4cm}{#2}{#3}{#4}}
\makeatother
%------------------------------------------------------------
% Section, Subsection
\makeatletter
\renewcommand{\section}{\@startsection{section}
    {1}{\z@}
    {1.5\Cvs \@plus.5\Cvs \@minus.2\Cvs}
    {1\Cvs \@plus.3\Cvs}
    {\normalfont\Large\headfont}}
\renewcommand{\subsection}{\@startsection{subsection}
    {2}{\z@}
    {1\Cvs \@plus.5\Cvs \@minus.2\Cvs}
    {.5\Cvs \@plus.3\Cvs}
    % {-1ex plus 1ex minus 1ex}
    {\normalfont\large\headfont}}
\renewcommand{\subsubsection}{\@startsection{subsubsection}
    {3}{\z@}%
    {-3.25ex\@plus -1ex \@minus -.2ex}%
    {1.5ex \@plus .2ex}%
    {\normalfont\normalsize\bfseries}}%
%------------------------------
\usepackage[explicit]{titlesec}
\renewcommand*{\thesection}{\arabic{section}}
% \titleformat*{\section}{\Large\bfseries}
% \titleformat*{\subsection}{\normalize\bfseries}
%------------------------------
\titleformat{\section}[block]{}{}{0pt}{
    \normalfont \Large\bfseries ~\thesection~.\hspace{0.5zw} #1
}[\normalsize\vspace*{-.15\baselineskip}\titlerule]
\titleformat{\subsection}
    {\large\bfseries}
    {\colorbox{black}{\color{white}\thesubsection}}
    {0.1cm}
    {\gdef\currentsectiontitle{#1}#1}
\titleformat{\subsubsection}
    {}
    {}
    {0pt}
    {\normalfont \normalsize\bfseries \color{MSLightBlue}~\thesubsubsection~ \hspace{1zw}#1}
\setcounter{secnumdepth}{3}
%------------------------------
% \titleformat{\chapter}[display]
%   {\normalfont\LARGE\sffamily}{\chaptertitlename\ \thechapter}{0pt}  {\titlerule\vskip2pt\titlerule\vskip20pt\Huge\bfseries\filleft#1}
% \titleformat{\section}
%     {\normalfont\Large\bfseries\sffamily}{\rule[.12ex]{8pt}{8pt}~\thesection~#1}{0.5em}{}
% \titleformat{\subsection}
%     {\normalfont\large\bfseries\sffamily}{\rule[.12ex]{8pt}{8pt}~\thesubsection~#1}{0.5em}{}
%------------------------------
% \titleformat{\section}
%     {\Large\bfseries}
%     {\colorbox{black}{\color{white}\thesection}}
%     {0.1cm}
%     {\gdef\currentsectiontitle{#1}#1}
% \titleformat{\subsection}
%     {\large\bfseries}
%     {\colorbox{black}{\color{white}\thesubsection}}
%     {0.1cm}
%     {\gdef\currentsectiontitle{#1}#1}
%------------------------------
% \newcommand\titlebar@{%sections
% \tikz[baseline,trim left=3.1cm,trim right=3.0cm] {
%     \fill [black!15] (2.5cm,-1ex) rectangle (\textwidth+3.1cm, 2.5ex);
%     \node[
%         fill=black!100!white,
%         anchor= base east,
%         % rounded rectangle,
%         minimum height=3.75ex] at (3cm,0.01) {
%             \color{white}\textbf{\thesection}
%         };
% }}
% \newcommand\titlebar@@{%osections
% \tikz[baseline,trim left=3.1cm,trim right=3.15cm] {
%     \fill [black!15] (2.5cm,-1ex) rectangle (\textwidth+3.1cm, 2.5ex);
%     \node[
%         fill=black!100!white,
%         anchor= base east,
%         % rounded rectangle,
%         minimum height=3.75ex] at (3cm,0.15) {
%         };
% }}
% \definecolor{gray75}{gray}{0.75}
% \newcommand{\hsp}{\hspace{20pt}}
% \titleformat{\chapter}[hang]{\Huge\bfseries}{\thechapter\hsp\textcolor{gray75}{|}\hsp}{0pt}{\Huge\bfseries #1}
% \titleformat{\chapter}[display]
%     {}
%     {}
%     {0pt}
%     {\normalfont\Huge\bfseries #1}
% \titleformat{\section}
%     {\large\bfseries}
%     {\colorbox{black}{\color{white}\thesection}}
%     {0.1cm}
%     {\gdef\currentsectiontitle{#1}#1}
% \newcommand\titlebar{\@ifstar\titlebar@@\titlebar@}
% \titleformat{\section}
%     {\large\bfseries}
%     {\titlebar}
%     {0.1cm}
%     {\gdef\currentsectiontitle{#1}#1}
% \titleformat{\subsection}
%     {\large\bfseries}
%     {\colorbox{black}{\color{white}\thesubsection}}
%     {0.1cm}
%     {\gdef\currentsectiontitle{#1}#1}
% \newcommand{\osection}[1]{\section*{\titlebar*#1}}
%------------------------------------------------------------
% \usepackage[normalem]{ulem}
% \titleformat{\section}
%   {\normalfont\Large\sffamily\bfseries}{}{0em}{\uline{\thesection\hspace*{ 1em}\MakeUppercase{#1}}}
% \titleformat{name=\section,numberless}
%   {\normalfont\Large\sffamily\bfseries\filcenter}{}{0em}{\uline{\MakeUppercase{#1}}}
\makeatother
%--------------------------------------------------------
% \usepackage{setspace}
% \newpagestyle{mypagestyle}{%
% \headrule
% \sethead{\thechapter. \chaptertitle}{}{\thesection. \sectiontitle}
% \setfoot{}{\usepage}{}%
% }
% \newpagestyle{mypagestyle}{%
% \sethead{}{}{\small \thesection. \sectiontitle $\bullet$ \usepage}
% \setfoot{}{}{}%
% }
% \pagestyle{mypagestyle}
%------------------------------------------------------------
\definecolor{frameinnercolor}{RGB}{49,44,44}
\newcounter{theorem}
\numberwithin{theorem}{section}
\newtcolorbox{theobox}[1]{%
    enhanced, frame empty, interior empty,
    coltitle=white, fonttitle=\bfseries, colbacktitle=frameinnercolor,
    extras broken={frame empty, interior empty},
    borderline={0.5mm}{0mm}{frameinnercolor},
    % sharp corners=downhill,
    sharp corners,
    breakable=true,
    top=4mm,
    before skip=3.5mm,
    attach boxed title to top left={yshift=-3mm,xshift=3mm},
    boxed title style={boxrule=0pt,sharp corners=all}, varwidth boxed title, title=#1}
\newenvironment{theo}[2][]
    {\refstepcounter{theorem}
     \ifstrempty{#1}{\begin{theobox}{#2~\thetheorem.}}
     {\begin{theobox}{#2~\thetheorem:~{#1}}}}
    {\end{theobox}}
\newenvironment{THEOREMBOX}[1][]
    {\begin{theo}[#1]{定理}}
    {\end{theo}}
%------------------------------------------------------------
\makeatletter
\newenvironment{LBARBOX}[2][black]{%
    \def\FrameCommand{\textcolor{#1}{\vrule width #2} \hspace{10pt}}%
    \MakeFramed {\advance\hsize-\width \FrameRestore}}%
    {\endMakeFramed}
\makeatother
%------------------------------------------------------------
% 数式を下線または囲みで表示してその下に文字を表示
\newcommand{\EXPRULINE}[2]{%
    \mathop{\mathrm{\underline{#1}}}_{\text{\scriptsize#2}}%
}
\newcommand{\EXPRBOXED}[2]{%
    \mathop{\mathrm{\boxed{#1}}}_{\text{\scriptsize#2}}%
}
\newcommand{\EXPRNOTE}[2][blue]{%
    {\color{#1}\leftarrow\scalebox{0.65}{$\displaystyle #2$}}%
}
%------------------------------------------------------------
\renewcommand{\contentsname}{目次}
% \renewcommand{\figurename}{図}
\renewcommand{\figurename}{Fig.}
\renewcommand{\tablename}{表}
%------------------------------------------------------------
\def\myspreadlines{2.5ex}
\def\myabovedisplayskip{0pt}
\def\mybelowdisplayskip{10pt}
\def\myabovedisplayshortskip{-5pt}
\def\mybelowdisplayshortskip{10pt}
\def\mytextwidth{\fullwidth}
\def\myevensidemargin{\oddsidemargin}
%------------------------------------------------------------
% User definitions
\newcommand{\REF}[1]{\COLORTHEMEBOXSUBGRAY{参考}{#1}}
\newcommand{\EXAMPLE}[1]{\COLORTHEMEBOX{例題}{#1}}
\makeatletter
\newcommand{\EXAMPLEJ}[3][.5]{
    \vspace{#1\Cvs}
    \begin{minipage}{\textwidth}
    \begin{itemize}
        \item[\colorthemebox{\small 例文}] #2
        \item[\colorthemebox@transparent{}] \scriptsize #3
    \end{itemize}
    \end{minipage}
    \vspace{.5\Cvs}
}
\newcommand{\EXAMPLEJITEM}[2]{
    \begin{itemize}
        \item[\colorthemebox{\small 例文}] #1
        \item[\colorthemebox@transparent{}] \scriptsize #2
    \end{itemize}
}
\makeatother
\newenvironment{NOTE}[1][]
    {\begin{COLORTHEMENOTE}{#1}}
    {\end{COLORTHEMENOTE}}
\newenvironment{RULE}[1][]
    {\begin{theo}[#1]{原則}}
    {\end{theo}}
\makeatletter
\newcommand{\nobreaklist}{\par\nobreak\@afterheading}
\newcommand{\nobreaklistend}{\vspace{-.5\Cvs}}
\makeatother
