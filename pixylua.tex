\usepackage{subfiles}
\usepackage{ifthen}
\usepackage{amsmath,amssymb}
\usepackage{mathtools}
\usepackage{polynom}
\usepackage{multicol}
\ifthenelse{\isundefined{\codebasewidth}}{
    \def\codebasewidth{0.6em}
}{}
\ifthenelse{\isundefined{\pixycodesize}}{
    \def\pixycodesize{\small}
}{}
\usepackage[bookmarks=true,bookmarksnumbered=true,colorlinks=true,linkcolor={blue},urlcolor={blue},hyperfootnotes=false,pdfborder={0,0,0},pdfpagemode=UseNone,unicode=true]{hyperref}
\usepackage{bm}
\usepackage{stmaryrd}
\makeatletter
% #1: 点同士の間隔
\newcommand{\cdotsfillsub}[1][.25\zw]{\leavevmode\cleaders\hb@xt@#1{\hss\(\cdot\m@th\)\hss}\hfill\kern\z@\ignorespaces}
\newcommand{\cdotsfill}[1]{\noindent\cdotsfillsub\ #1 \cdotsfillsub\par}
\newcommand{\leaderfill}[1][0pt]{\noindent\hspace{#1}\cleaders\hbox{…………}\hfill \hspace{#1}\par}
\makeatother
\def\dashfill{\cleaders\hbox\ to 2em{-}\hfill}
\def\Tasuki#1#2#3#4#5#6#7{% chktex 15
\setbox0\hbox{
\setlength\unitlength{0.1in}
\begin{picture}( 3.0000, 1.0000)( 0.0000, 0.0000)
    \put(0,0){\line(3,1){3}}
    \put(0,1){\line(3,-1){3}}
\end{picture}%
}
\begin{array}{cccccl}
#1 & &#3 & \longrightarrow & #5 & \\[-0.5ex]
#2 & \raise1ex\box0 &#4 & \longrightarrow & #6 & (+ \\[0.5ex] \cline{1-5}
& & & & #7 &
\end{array}
}% chktex 9
%
\newcommand{\Rev}[1]{\frac{1}{#1}}
\newcommand{\Pdrv}[2]{\frac{\partial #1}{\partial #2}}
\newcommand{\Drv}[2]{\frac{d#1}{d#2}}
\newcommand{\Ddrv}[2]{\frac{\delta #1}{\delta #2}}
\newcommand{\PVEC}[1]{{\small\begin{pmatrix}#1\end{pmatrix}}}
\newcommand{\BVEC}[1]{{\small\begin{bmatrix}#1\end{bmatrix}}}
\newcommand{\PVECs}[1]{{\scriptsize\begin{pmatrix}#1\end{pmatrix}}}
\newcommand{\BVECs}[1]{{\scriptsize\begin{bmatrix}#1\end{bmatrix}}}
\newcommand{\PMAT}[2][]{{\small\begin{pmatrix}{#1}#2\end{pmatrix}}}
\newcommand{\BMAT}[2][]{{\small\begin{bmatrix}{#1}#2\end{bmatrix}}}
\newcommand{\DMAT}[2][]{{\small\left|\begin{matrix}{#1}#2\end{matrix}\right|}}
% \newcommand{\Vvec}[1]{\left(\begin{array}{c}#1\end{array}\right)}
% \newcommand{\Vmatrix}[2][]{\left(\begin{array}{#1}#2\end{array}\right)}
% \newcommand{\Dmatrix}[2][]{\left|\begin{array}{#1}#2\end{array}\right|}
\newcommand{\Inversion}[2][]{#1^{\!\mbox{\sf\tiny T}}}
\newcommand{\Vecbm}[1]{\mbox{\boldmath \(#1\)}}
\newcommand{\Beq}{\begin{equation}}
\newcommand{\Eeq}{\end{equation}}
\newcommand{\Beqn}{\begin{equation*}}
\newcommand{\Eeqn}{\end{equation*}}
\newcommand{\Va}{\Vecbm{a}}
%------------------------------------------------------------
% Color
%
\usepackage{color}
\usepackage[x11names]{xcolor}
\definecolor{colorthemeback}{RGB}{255,255,255}
\definecolor{colorthemefore}{RGB}{77,77,77}
\definecolor{colorthememain1}{RGB}{0,113,188}
\definecolor{colorthememain2}{RGB}{255,80,80}
\definecolor{colorthemesub1}{RGB}{242,242,242}
\definecolor{colorthemesub2}{RGB}{226,241,250}
\definecolor{colorthemesub3}{RGB}{255,234,234}
\definecolor{graphcolor}{RGB}{204,0,0}
\definecolor{graphsubcolor}{RGB}{0,80,255}
%------------------------------------------------------------
% Figure, Table, Graph
%
\usepackage{tikz}
\usetikzlibrary{calc,intersections,quotes,angles,shapes.misc,patterns,decorations.markings}
\usepackage{tkz-euclide}
\usepackage{booktabs}
\usepackage{colortbl}
\usepackage{here}
\graphicspath{{images/}{../images/}}
\usepackage{pgfplots}
\DeclareMathOperator{\CDF}{cdf}

\def\cdf(#1)(#2)(#3){0.5*(1+(erf((#1-#2)/(#3*sqrt(2)))))}%
\tikzset{
    declare function={
        normcdf(\x,\m,\s)=1/(1 + exp(-0.07056*((\x-\m)/\s)^3 - 1.5976*(\x-\m)/\s));
        normpdf(\x,\m,\s)=exp(-(\x-\m)^2/(2*\s)^2)/(sqrt(2*pi)*\s);
        gamma(\n)=(\n-1)!;
        beta(\a,\b)=(gamma(\a)*gamma(\b))/gamma(\a+\b);
        betapdf(\x,\a,\b)=(\x^(\a-1)*(1-\x)^(\b-1))/beta(\a,\b);
    }
}
% \pgfplotsset{compat=1.12}
\usepgfplotslibrary{fillbetween}
\pgfplotsset{compat=newest}
\pgfplotsset{grid style={dashed,gray}}
\usepackage{wrapfig}
\usepackage[hang,small,bf]{caption}
\usepackage[subrefformat=parens]{subcaption}
\captionsetup{compatibility=false}
%------------------------------------------------------------
% Frame, Box
%
\usepackage{framed}
\usepackage[framemethod=TikZ]{mdframed}
\usepackage[many]{tcolorbox}
\usepackage{varwidth}
% 囲み(中央タイトル)
\newenvironment{STDBOX}[2][]{
    \ifthenelse{\equal{#2}{}}{
        \mdfsetup{
            frametitle={\colorbox{white}{\space#2\space}},
            innertopmargin=10pt,
            innerbottommargin=10pt,
            % frametitleaboveskip=-\ht\strutbox,
            frametitleaboveskip=-8pt,
            frametitlealignment=\center,
            #1
            }
    }{
        \mdfsetup{
        frametitle={\colorbox{white}{\space#2\space}},
        innertopmargin=0pt,
        innerbottommargin=10pt,
        % frametitleaboveskip=-\ht\strutbox,
        frametitleaboveskip=-8pt,
        frametitlealignment=\center,
        #1
        }
    }
    \begin{mdframed}%
    }
    {\end{mdframed}}

% \mdfdefinestyle{guidelinestyle}{%
%     linecolor=black,
%     linewidth=1pt,
%     frametitlerule=true,
%     frametitlefont=\sffamily\bfseries,
%     frametitlebackgroundcolor=gray!20,
%     innertopmargin=\topskip
% }
% \makeatletter
% \define@key{thmdef}{mdthm}[{}]{%
% \thmt@trytwice{\def\thmt@theoremdefiner{\mdtheorem[#1]}}{}}
% \makeatother
% \declaretheorem[mdthm={style=guidelinestyle},numberwithin=section]{theorem}

\mdfdefinestyle{DUALSIDELINEBOX}{%
    rightline=true,
    innerleftmargin=10,
    innerrightmargin=10,
    outerlinewidth=3pt,
    linewidth=3pt,
    topline=false,
    rightline=true,
    bottomline=false,
    skipabove=\topsep,
    skipbelow=\topsep
}

% http://tex.bootmath.com/how-to-create-highlight-boxes-in-latex.html
% \def\bracketcolor{red!75!black}
% \def\bracketwidth{3pt}
\def\bracketcolor{black}
\def\bracketwidth{.5pt}
\newtcolorbox{BRACKETBOX}[1][]{
    breakable,
    freelance,
    title=#1,
    colback=white,
    colbacktitle=white,
    coltitle=black,
    fonttitle=\bfseries,
    before skip=20pt plus 2pt minus 2pt,
    after skip=20pt plus 2pt minus 2pt,
    bottomrule=0pt,
    boxrule=0pt,
    colframe=white,
    overlay unbroken and first={
    \draw[\bracketcolor,line width=\bracketwidth]
        ([xshift=5pt]frame.north west) -- 
        (frame.north west) -- 
        (frame.south west);
    \draw[\bracketcolor,line width=\bracketwidth]
        ([xshift=-5pt]frame.north east) -- 
        (frame.north east) -- 
        (frame.south east);
    },
    overlay unbroken app={
    \draw[\bracketcolor,line width=\bracketwidth,line cap=rect]
        (frame.south west) -- 
        ([xshift=5pt]frame.south west);
    \draw[\bracketcolor,line width=\bracketwidth,line cap=rect]
        (frame.south east) -- 
        ([xshift=-5pt]frame.south east);
    },
    overlay middle and last={
    \draw[\bracketcolor,line width=\bracketwidth]
        (frame.north west) -- 
        (frame.south west);
    \draw[\bracketcolor,line width=\bracketwidth]
        (frame.north east) -- 
        (frame.south east);
    },
    overlay last app={
    \draw[\bracketcolor,line width=\bracketwidth,line cap=rect]
        (frame.south west) --
        ([xshift=5pt]frame.south west);
    \draw[\bracketcolor,line width=\bracketwidth,line cap=rect]
        (frame.south east) --
        ([xshift=-5pt]frame.south east);
    },
}

\newcommand{\SIMPLEBOX}[2]{%
    \newcounter{#1}
    \newenvironment{#1}[1][]{%
        \stepcounter{#1}%
        \mdfsetup{%
            innertopmargin=0pt,%
            % innertopmargin=-5pt,%
            frametitleaboveskip=-8pt,%
            % frametitleaboveskip=\dimexpr-\ht\strutbox\relax,%
            frametitle={%
                \tikz[baseline=(current bounding box.east),outer sep=0pt]
                \node[anchor=east,rectangle,fill=white]
                {\strut \MakeUppercase#1~\csname the#1\endcsname\ifstrempty{##1}{}{:~##1}};},% chktex 41
            innerbottommargin=1em,linecolor=#2,%
            linewidth=0.5pt,topline=true,%
        }%
        \begin{mdframed}[]\relax%
        }{\end{mdframed}}%
}
% \SIMPLEBOX{definition}{red!60}
% \SIMPLEBOX{theorem}{black}
\SIMPLEBOX{EXERCISEBOX}{black!50}
\SIMPLEBOX{EXAMPLEBOX}{black!50}
\makeatletter
\newtcolorbox{CTNOTE}[1]{
    breakable,
    before skip=20pt plus 2pt minus 2pt,
    after skip=20pt plus 2pt minus 2pt,
    boxrule=0.4pt,
    colframe=colorthemefore, colback=colorthemeback,
    fonttitle=\gtfamily\bfseries, title=#1
}
\newtcolorbox{CTNOTEBLUE}[1]{
    breakable,
    before skip=20pt plus 2pt minus 2pt,
    after skip=20pt plus 2pt minus 2pt,
    boxrule=0.4pt,
    colframe=colorthememain1, colback=colorthemeback,
    fonttitle=\gtfamily\bfseries, title=#1
}
\newtcolorbox{CTNOTERED}[1]{
    breakable,
    before skip=20pt plus 2pt minus 2pt,
    after skip=20pt plus 2pt minus 2pt,
    boxrule=0.4pt,
    colframe=colorthememain2, colback=colorthemeback,
    fonttitle=\gtfamily\bfseries, title=#1
}
\newtcbox{\ctb}{
    boxrule=0.4pt, colframe=colorthemefore, colback=colorthemeback, 
    top=0mm, bottom=0mm, left=0mm, right=0mm, on line, arc=0.5mm
}
\newtcbox{\ctb@main}{
    boxrule=0.4pt, colframe=colorthememain1, colback=colorthemeback, 
    top=0mm, bottom=0mm, left=0mm, right=0mm, on line, arc=0.5mm
}
\newtcbox{\ctb@accent}{
    boxrule=0.4pt, colframe=colorthememain2, colback=colorthemeback, 
    top=0mm, bottom=0mm, left=0mm, right=0mm, on line, arc=0.5mm
}
\newtcbox{\ctbinv}{
    boxrule=0.4pt, colframe=colorthemeback, colback=colorthemefore, 
    top=0mm, bottom=0mm, left=0mm, right=0mm, on line, arc=0.5mm
}
\newtcbox{\ctbinv@main}{
    boxrule=0.4pt, colframe=colorthememain1, colback=colorthememain1, 
    top=0mm, bottom=0mm, left=0mm, right=0mm, on line, arc=0.5mm
}
\newtcbox{\ctbinv@accent}{
    boxrule=0.4pt, colframe=colorthememain2, colback=colorthememain2, 
    top=0mm, bottom=0mm, left=0mm, right=0mm, on line, arc=0.5mm
}
\newtcbox{\ctb@subgray}{
    boxrule=0.4pt, colframe=colorthemefore, colback=colorthemesub1, 
    top=0mm, bottom=0mm, left=0mm, right=0mm, on line, arc=0.5mm
}
\newtcbox{\ctb@subblue}{
    boxrule=0.4pt, colframe=colorthememain1, colback=colorthemesub2, 
    top=0mm, bottom=0mm, left=0mm, right=0mm, on line, arc=0.5mm
}
\newtcbox{\ctb@subred}{
    boxrule=0.4pt, colframe=colorthememain2, colback=colorthemesub3, 
    top=0mm, bottom=0mm, left=0mm, right=0mm, on line, arc=0.5mm
}
\newtcbox{\ctb@transparent}{
    boxrule=0.4pt, colframe=colorthemeback, colback=colorthemeback, 
    top=0mm, bottom=0mm, left=0mm, right=0mm, on line, arc=0.5mm
}
\newcommand{\CTBOX}[2]{
    \begin{itemize}
        \item[\ctb{\small #1}] #2
    \end{itemize}
}
\newcommand{\CTBOXBLUE}[2]{
    \begin{itemize}
        \item[\ctb@main{\color{colorthememain1}\small #1}] #2
    \end{itemize}
}
\newcommand{\CTBOXRED}[2]{
    \begin{itemize}
        \item[\ctb@accent{\color{colorthememain2}\small #1}] #2
    \end{itemize}
}
\newcommand{\CTBOXINV}[2]{
    \begin{itemize}
        \item[\ctbinv{\color{colorthemeback}\small #1}] #2
    \end{itemize}
}
\newcommand{\CTBOXINVBLUE}[2]{
    \begin{itemize}
        \item[\ctbinv@main{\color{colorthemeback}\small #1}] #2
    \end{itemize}
}
\newcommand{\CTBOXINVRED}[2]{
    \begin{itemize}
        \item[\ctbinv@accent{\color{colorthemeback}\small #1}] #2
    \end{itemize}
}
\newcommand{\CTBOXSUBGRAY}[2]{
    \begin{itemize}
        \item[\ctb@subgray{\color{colorthemefore}\small #1}] #2
    \end{itemize}
}
\newcommand{\CTBOXSUBBLUE}[2]{
    \begin{itemize}
        \item[\ctb@subblue{\color{colorthememain1}\small #1}] #2
    \end{itemize}
}
\newcommand{\CTBOXSUBRED}[2]{
    \begin{itemize}
        \item[\ctb@subred{\color{colorthememain2}\small #1}] #2
    \end{itemize}
}
\newcommand{\CTBOXTRANS}[2]{
    \begin{itemize}
        \item[\ctb@transparent{\small #1}] #2
    \end{itemize}
}
\makeatother
%------------------------------------------------------------
% Code
%
\usepackage{listings}
\usepackage{verbatim}
%------------------------------------------------------------
\newtcbox{\hlbox}[1][]{
    boxrule=0.4pt,
    boxsep=2pt,
    sharp corners,
    colframe=white,
    % colframe=gray!40,
    % colframe=black,
    colback=yellow,
    top=0mm,
    bottom=0mm,
    left=0mm,
    right=0mm,
    on line,
    #1
}
% \newtcbox{\HLBOX}[1][]{
%     equal height group=dashedbox,
%     nobeforeafter,
%     colback=yellow,
%     boxrule=0pt,
%     boxsep=0pt,
%     top=2pt,
%     bottom=2pt,
%     left=2pt,
%     right=2pt,
%     enhanced jigsaw,
%     borderline horizontal={1pt}{0pt}{dashed},
%     borderline vertical={1pt}{0pt}{dashed},
%     #1
% }
\newcommand{\HL}[1]{\hlbox{\mbox{\small\rmfamily#1}}}
\newcommand{\EM}[1]{\color{red!80!black}\textbf{#1}\color{black}}
\newcommand{\EMM}[1]{\color{red!80!black}#1\color{black}}
\definecolor{MSBlue}{rgb}{.204,.353,.541}
\definecolor{MSLightBlue}{rgb}{.31,.506,.741}
\newcommand{\EMSUB}[1]{\mbox{\hspace{.3\zw}\color{MSLightBlue}\textbf{#1}\color{black}\hspace{.3\zw}}}

\newtcbox{\embox}{%
    boxrule=0.4pt,%
    colframe=red!75!black,%
    colback=red!75!black,%
    top=0mm, bottom=0mm, left=0mm, right=0mm,%
    on line, arc=0.5mm%
}
\newcommand{\EMBOX}[1]{\mbox{\hspace{.3\zw}\embox{\small\color{white} #1\color{black}}\hspace{.3\zw}}}

\newtcbox{\emcbox}{%
    boxrule=0.4pt,%
    colframe=colorthemesub1,%
    colback=colorthemesub1,%
    top=0mm, bottom=0mm, left=0mm, right=0mm,%
    on line, arc=0.5mm%
}
\newcommand{\EMCBOX}[1]{\mbox{\hspace{.3\zw}\emcbox{\small\bfseries\color{black} #1}\hspace{.3\zw}}}
\newcommand{\EMMCBOX}[1]{\mbox{\hspace{.3\zw}\emcbox{\small\ttfamily\color{black} #1}\hspace{.3\zw}}}

\newtcolorbox{REDBOX}{
    colback=red!5!white,
    colframe=red!75!black
}
\newtcolorbox{REDBOXT}[1]{
    colback=red!5!white,
    colframe=red!75!black,
    fonttitle=\bfseries,title=#1
}
\newtcbox{\INLINE}{
    size=fbox, on line,
    colframe=black!60,
    colback=gray!10,
    boxrule=1pt,
    top=0.5mm,bottom=0.5mm,left=0.5mm,right=0.5mm,
    fontupper=\ttfamily\small
}
\definecolor{MathColor}{rgb}{0.0,0.56,0.0}
\newtcolorbox{MATHBOX}[1][]{%
    % on line,
    nobeforeafter, math upper, tcbox raise base, enhanced,
    colframe=MathColor,
    colback=white,
    arc=4pt,
    boxrule=2pt,
    drop fuzzy shadow
}
%------------------------------------------------------------
\renewcommand{\lstlistingname}{リスト}
\lstset{
    tabsize=4,
    breaklines=true,
    breakindent=0pt,
    showspaces=false,
    showstringspaces=false,
    % showlines=false,
    basicstyle=\codefont\pixycodesize,
    % keywordstyle=\bfseries\color{green},
    keywordstyle=\bfseries\color{blue},
    keywordstyle={[2]\color{red!90!black}},
    keywordstyle={[3]\color{red}},
    commentstyle=\itshape\color{purple!40!black},
    % identifierstyle=\color{blue},
    stringstyle=\color{magenta},
    xleftmargin=0mm,
    framexleftmargin=0mm,
    % framesep=0pt,
    framextopmargin=6pt,
    framexbottommargin=6pt,
    % framerule=0.5pt,%
    commentstyle=\color{green!50!black},
    comment=[l]\%\ ,% chktex 26
    % otherkeywords={String,async,await,Task,var},
    % keywords=[2]{DatabaseField,DatabaseTable},
    % keywords=[3]{@},
    % escapeinside={(*@}{@*)},%
    % escapeinside={\%*}{*)},
    % lineskip=-1.ex,%
    % lineskip=-1pt,
    belowcaptionskip=1\baselineskip,
    captionpos=b,
    columns=fixed,
    basewidth=\codebasewidth,
}
\lstnewenvironment{pre}[1][]
    {\vspace{1\zw}\lstset{
        frame=lrtb,
        framerule=0.5pt,
        xleftmargin=6pt,
        framexleftmargin=6pt,
        framesep=0pt,
        framextopmargin=6pt,
        framexbottommargin=6pt,
        #1}}
    {}
\lstnewenvironment{code}[1][]{
    \mdframed[%
        backgroundcolor=black!5!white,%
        innertopmargin=0pt,%
        innerbottommargin=0pt,%
        ]
    \lstset{#1}
    }{\endmdframed}
\lstnewenvironment{code2}[1][]{
    \mdframed[backgroundcolor=black!5!white]
    \renewcommand{\thelstnumber}{\arabic{lstnumber}:}
    \lstset{
        numbers=left,
        numberstyle=\codefont\pixycodesize,
        xleftmargin=16pt,
        framexleftmargin=6pt,
        #1}
    }{\endmdframed}
\lstnewenvironment{source}[1][]
    {\vspace{1\zw}
     \lstset{
        frame=lines,
        framerule=0.5pt,
        xleftmargin=6pt,
        framexleftmargin=6pt,
        framesep=0pt,
        framextopmargin=6pt,
        framexbottommargin=6pt,
        rulecolor=\color{cyan},
        backgroundcolor=\color{cyan!5!white},
        #1}}
    {}
\lstnewenvironment{source2}[1][]
    {\vspace{1\zw}
     \renewcommand{\thelstnumber}{\arabic{lstnumber}:}
     \lstset{
        numbers=left,
        numberstyle=\codefont\pixycodesize,
        numbersep=4pt,
        frame=lines,
        framerule=0.5pt,
        xleftmargin=16pt,
        framexleftmargin=16pt,
        framesep=0pt,
        framextopmargin=6pt,
        framexbottommargin=6pt,
        rulecolor=\color{cyan},
        backgroundcolor=\color{cyan!5!white},
        #1}}
    {}
% \makeatletter
% \def\lst@lettertrue{\let\lst@ifletter\iffalse}
% \makeatother
%------------------------------------------------------------
\makeatletter
\newdimen\fb@xsep%
\newdimen\fb@xrule%
\newcommand{\image}[5][H]{%
    \ifthenelse{\equal{#4}{}}{%
        \begin{figure}[#1]%
            \centering%
            \includegraphics[#2]{#3}%
        \end{figure}%
    }{%
        \begin{figure}[#1]%
            \centering%
            \includegraphics[#2]{#3}%
            \caption{#4}%
            \label{#5}%
        \end{figure}%
    }%
}
\newcommand{\imageF}[4][H]{\image[#1]{width=16cm}{#2}{#3}{#4}}
\newcommand{\imageH}[4][H]{\image[#1]{width=10cm}{#2}{#3}{#4}}
\newcommand{\imageL}[4][H]{\image[#1]{width=8cm}{#2}{#3}{#4}}
\newcommand{\imageM}[4][H]{\image[#1]{width=6cm}{#2}{#3}{#4}}
\newcommand{\imageS}[4][H]{\image[#1]{width=4cm}{#2}{#3}{#4}}
\newcommand{\frameimage}[5][H]{%
    \fb@xsep=\fboxsep%
    \fb@xrule=\fboxrule%
    \fboxsep=0pt%
    \fboxrule=0.5pt%
    \ifthenelse{\equal{#4}{}}{%
        \begin{figure}[#1]%
            \centering%
            \fbox{\includegraphics[#2]{#3}}%
        \end{figure}%
    }{%
        \begin{figure}[#1]%
            \centering%
            \fbox{\includegraphics[#2]{#3}}%
            \caption{#4}%
            \label{#5}%
        \end{figure}%
    }%
    \fboxsep=\fb@xsep%
    \fboxrule=\fb@xrule%
}
\newcommand{\frameimageF}[4][H]{\frameimage[#1]{width=16cm}{#2}{#3}{#4}}
\newcommand{\frameimageH}[4][H]{\frameimage[#1]{width=10cm}{#2}{#3}{#4}}
\newcommand{\frameimageL}[4][H]{\frameimage[#1]{width=8cm}{#2}{#3}{#4}}
\newcommand{\frameimageM}[4][H]{\frameimage[#1]{width=6cm}{#2}{#3}{#4}}
\newcommand{\frameimageS}[4][H]{\frameimage[#1]{width=4cm}{#2}{#3}{#4}}
\makeatother
%------------------------------------------------------------
\definecolor{frameinnercolor}{RGB}{49,44,44}
\newcounter{theorem}
\numberwithin{theorem}{section}
\newtcolorbox{theobox}[1]{%
    enhanced, frame empty, interior empty,
    coltitle=white, fonttitle=\bfseries, colbacktitle=frameinnercolor,
    extras broken={frame empty, interior empty},
    borderline={0.5mm}{0mm}{frameinnercolor},
    % sharp corners=downhill,
    sharp corners,
    breakable=true,
    top=4mm,
    before skip=3.5mm,
    attach boxed title to top left={yshift=-3mm,xshift=3mm},
    boxed title style={boxrule=0pt,sharp corners=all}, varwidth boxed title, title=#1}
\newenvironment{theo}[2][]
    {\refstepcounter{theorem}
     \ifstrempty{#1}{\begin{theobox}{#2~\thetheorem.}}
     {\begin{theobox}{#2~\thetheorem:~{#1}}}}
    {\end{theobox}}
\newenvironment{THEOREMBOX}[1][]
    {\begin{theo}[#1]{定理}}
    {\end{theo}}
%------------------------------------------------------------
\makeatletter
\newenvironment{LBARBOX}[1][black]{%
    \def\FrameCommand{\textcolor{#1}{\vrule width 1pt} \hspace{10pt}}% 
    \MakeFramed {\advance\hsize-\width \FrameRestore}}% chktex 41
    {\endMakeFramed}
\newenvironment{LBARBOX2}[2][black]{%
    \def\FrameCommand{\textcolor{#1}{\vrule width #2} \hspace{10pt}}% 
    \MakeFramed {\advance\hsize-\width \FrameRestore}}% chktex 41
    {\endMakeFramed}
\makeatother
%------------------------------------------------------------
% 数式を下線または囲みで表示してその下に文字を表示
\newcommand{\EXPRULINE}[2]{%
    \mathop{\mathrm{\underline{#1}}}_{\text{\scriptsize#2}}%
}
\newcommand{\EXPRBOXED}[2]{%
    \mathop{\mathrm{\boxed{#1}}}_{\text{\scriptsize#2}}%
}
\newcommand{\EXPRNOTE}[2][blue]{%
    {\color{#1}\leftarrow\scalebox{0.65}{\(\displaystyle #2\)}}%
}
%------------------------------------------------------------
\renewcommand{\contentsname}{目次}
% \renewcommand{\figurename}{図}
\renewcommand{\figurename}{Fig.}
\renewcommand{\tablename}{表}
%------------------------------------------------------------
% User definitions
%
\newcommand{\REF}[1]{\CTBOXSUBGRAY{参考}{#1}}
\newcommand{\EXAMPLE}[1]{\CTBOX{例題}{#1}}
\makeatletter
\newcommand{\EXAMPLEJ}[3][.5]{
    \vspace{#1\Cvs}
    \begin{minipage}{\textwidth}
    \begin{itemize}
        \item[\ctb{\small 例文}] #2
        \item[\ctb@transparent{}] \scriptsize #3
    \end{itemize}
    \end{minipage}
    \vspace{.5\Cvs}
}
\newcommand{\EXAMPLEJITEM}[2]{
    \begin{itemize}
        \item[\ctb{\small 例文}] #1
        \item[\ctb@transparent{}] \scriptsize #2
    \end{itemize}
}
\makeatother
\newenvironment{NOTE}[1][]
    {\begin{CTNOTE}{#1}}
    {\end{CTNOTE}}
\newenvironment{RULE}[1][]
    {\begin{theo}[#1]{原則}}
    {\end{theo}}
\makeatletter
\newcommand{\nobreaklist}{\par\nobreak\@afterheading}% chktex 21
\newcommand{\nobreaklistend}{\vspace{-.5\Cvs}}
\makeatother
\renewcommand{\labelenumi}{\textcircled{\scriptsize \theenumi}}
