\usepackage{subfiles}
\usepackage{ifthen}
\usepackage{amsmath,amssymb}
\usepackage{mathtools}
\usepackage{polynom}
\usepackage{multicol}
\ifthenelse{\isundefined{\codebasewidth}}{
    \def\codebasewidth{0.6em}
}{}
\ifthenelse{\isundefined{\pixycodesize}}{
    \def\pixycodesize{\small}
}{}
\usepackage[bookmarks=true,bookmarksnumbered=true,colorlinks=true,linkcolor={blue},urlcolor={blue},citecolor=magenta,hyperfootnotes=false,pdfborder={0,0,0},pdfpagemode=UseNone,unicode=true]{hyperref}
\usepackage{cleveref}
\usepackage{autonum}
\usepackage{bm}
\usepackage{stmaryrd}
%------------------------------------------------------------
% MARK: Common
% #1: 点同士の間隔
\makeatletter
\NewDocumentCommand \pixy@hruledots{O{.25\zw}}{%
    \leavevmode\cleaders\hb@xt@#1{\hss\(\cdot\m@th\)\hss}\hfill\kern\z@\ignorespaces
}
\NewDocumentCommand \HRuleDots{m}{%
    \noindent\pixy@hruledots\ #1 \pixy@hruledots\par%
}
\NewDocumentCommand \HRuleLeader{O{0pt}}{%
    \noindent\hspace{#1}\cleaders\hbox{…………}\hfill \hspace{#1}\par%
}
\makeatother
\def\HRuleDash{\cleaders\hbox\ to 2em{-}\hfill}
\def\Tasuki#1#2#3#4#5#6#7{% chktex 15
\setbox0\hbox{
\setlength\unitlength{0.1in}
\begin{picture}( 3.0000, 1.0000)( 0.0000, 0.0000)
    \put(0,0){\line(3,1){3}}
    \put(0,1){\line(3,-1){3}}
\end{picture}%
}
\begin{array}{cccccl}
#1 & &#3 & \longrightarrow & #5 & \\[-0.5ex]
#2 & \raise1ex\box0 &#4 & \longrightarrow & #6 & (+ \\[0.5ex] \cline{1-5}
& & & & #7 &
\end{array}
}% chktex 9
%
\NewDocumentCommand \Rev{m}{\frac{1}{#1}}
\NewDocumentCommand \Drv{m m}{\frac{d#1}{d#2}}
\NewDocumentCommand \PDrv{m m}{\frac{\partial #1}{\partial #2}}
\NewDocumentCommand \DDrv{m m}{\frac{\delta #1}{\delta #2}}
\NewDocumentCommand \PVec{m}{{\small\begin{pmatrix}#1\end{pmatrix}}}
\NewDocumentCommand \BVec{m}{{\small\begin{bmatrix}#1\end{bmatrix}}}
\NewDocumentCommand \PVecs{m}{{\scriptsize\begin{pmatrix}#1\end{pmatrix}}}
\NewDocumentCommand \BVecs{m}{{\scriptsize\begin{bmatrix}#1\end{bmatrix}}}
\NewDocumentCommand \PMat{O{} m}{{\small\begin{pmatrix}{#1}#2\end{pmatrix}}}
\NewDocumentCommand \BMat{O{} m}{{\small\begin{bmatrix}{#1}#2\end{bmatrix}}}
\NewDocumentCommand \DMat{O{} m}{{\small\left|\begin{matrix}{#1}#2\end{matrix}\right|}}
%\NewDocumentCommand \Vvec{m}{\left(\begin{array}{c}#1\end{array}\right)}
%\NewDocumentCommand \Vmatrix{O{} m}{\left(\begin{array}{#1}#2\end{array}\right)}
%\NewDocumentCommand \Dmatrix{O{} m}{\left|\begin{array}{#1}#2\end{array}\right|}
\NewDocumentCommand \Inversion{m m}{#1^{\!\mbox{\sf\tiny #2}}}
\NewDocumentCommand \Vecbm{m}{\mbox{\boldmath \(#1\)}}
%------------------------------------------------------------
% MARK: Color
\usepackage{color}
\usepackage[x11names]{xcolor}
\definecolor{ColorThemeBack}{RGB}{255,255,255}
\definecolor{ColorThemeFore}{RGB}{77,77,77}
\definecolor{ColorThemePrimary}{RGB}{0,113,188}
\definecolor{ColorThemeSecondary}{RGB}{255,80,80}
\definecolor{ColorThemeSub1}{RGB}{242,242,242}
\definecolor{ColorThemeSub2}{RGB}{226,241,250}
\definecolor{ColorThemeSub3}{RGB}{255,234,234}
\definecolor{ColorThemeBlack}{RGB}{49,44,44}
\definecolor{ColorThemeGraph}{RGB}{204,0,0}
\definecolor{ColorThemeGraphSub}{RGB}{0,80,255}
%------------------------------------------------------------
% MARK: Figure, Table, Graph
\usepackage{tikz}
\usetikzlibrary{calc,intersections,quotes,angles,shapes.misc,patterns,decorations.markings}
\usepackage{tkz-euclide}
% \usepackage{booktabs}
% \usepackage{colortbl}
\usepackage{here}
\graphicspath{{images/}{../images/}}
\usepackage{pgfplots}
\DeclareMathOperator{\CDF}{cdf}

\def\cdf(#1)(#2)(#3){0.5*(1+(erf((#1-#2)/(#3*sqrt(2)))))}%
\tikzset{
    declare function={
        normcdf(\x,\m,\s)=1/(1 + exp(-0.07056*((\x-\m)/\s)^3 - 1.5976*(\x-\m)/\s));
        normpdf(\x,\m,\s)=exp(-(\x-\m)^2/(2*\s)^2)/(sqrt(2*pi)*\s);
        gamma(\n)=(\n-1)!;
        beta(\a,\b)=(gamma(\a)*gamma(\b))/gamma(\a+\b);
        betapdf(\x,\a,\b)=(\x^(\a-1)*(1-\x)^(\b-1))/beta(\a,\b);
    }
}
% \pgfplotsset{compat=1.12}
\usepgfplotslibrary{fillbetween}
\pgfplotsset{compat=newest}
\pgfplotsset{grid style={dashed,gray}}
\usepackage{wrapfig}
\usepackage[hang,small,bf]{caption}
\usepackage[subrefformat=parens]{subcaption}
\captionsetup{compatibility=false}
\usepackage{tabularray}
% Common Table
\makeatletter
\pgfkeys{%
    /pixy/table/.cd,
    rowhdr/.store in=\pixy@table@rowhdr,
    rowhdr/.default=yes,
    rowhdr/.is choice,
    rowhdr/no/.style={rowbg=white,rowfg=black},
    rowhdr/yes/.style={rowbg=azure3,rowfg=white},
    rowbg/.store in=\pixy@table@rowbg,
    rowfg/.store in=\pixy@table@rowfg,
    colhdr/.store in=\pixy@table@colhdr,
    colhdr/.default=yes,
    colhdr/.is choice,
    colhdr/no/.style={colbg=white,colfg=black,col=no},
    colhdr/yes/.style={colbg=azure8,colfg=black,col=yes},
    colbg/.store in=\pixy@table@colbg,
    colfg/.store in=\pixy@table@colfg,
    col/.store in=\pixy@table@col,
    even/.store in=\pixy@table@even,
    even/.default=yes,
    even/.is choice,
    even/no/.style={evenbg=white},
    even/yes/.style={evenbg=azure9},
    evenbg/.store in=\pixy@table@evenbg
}
\NewDocumentEnvironment{Table}{o m +b}{%
    \pgfkeys{/pixy/table/.cd, rowhdr=no, colhdr=no, even=no}
    \IfValueT{#1}{\pgfkeys{/pixy/table/.cd, #1}}{}
    \ifthenelse{\equal{\pixy@table@col}{yes}}{
        \begin{tblr}{colspec={#2},vline{2}={solid},hlines={0.05em,solid,azure2},hline{1,Z}={1.5pt},%
            row{even}={bg={\pixy@table@evenbg}},%
            row{1}={bg={\pixy@table@rowbg},fg={\pixy@table@rowfg}},%
            column{1}={bg={\pixy@table@colbg},fg={\pixy@table@colfg}}}%
            #3
    }{
        \begin{tblr}{colspec={#2},vline{2}={solid},hlines={0.05em,solid,azure2},hline{1,Z}={1.5pt},%
            row{even}={bg={\pixy@table@evenbg}},%
            row{1}={bg={\pixy@table@rowbg},fg={\pixy@table@rowfg}}}%
            #3
    }
    \end{tblr}
}{}
\makeatother
%------------------------------------------------------------
% MARK: Frame, Box
\usepackage[many]{tcolorbox}
\tcbuselibrary{listings}
\usepackage{varwidth}
% 囲み(中央タイトル)
\makeatletter
\pgfkeys{
    /pixy/note/.cd,
    colback/.store in=\pixy@note@colback,
    colback/.default=black!5!white,
    trans/.store in=\pixy@note@trans,
    trans/.default=yes,
    type/.store in=\pixy@note@type,
    type/.default=normal,
    sharp/.store in=\pixy@note@sharp,
    sharp/.default=yes,
    thin/.store in=\pixy@note@thin,
    thin/.default=yes,
    eval/sharp/.is choice,
    eval/sharp/no/.style={},
    eval/sharp/yes/.style={sharp corners},
    eval/trans/.is choice,
    eval/trans/no/.style={},
    eval/trans/yes/.style={colback=white},
    eval/thin/.is choice,
    eval/thin/no/.style={},
    eval/thin/yes/.style={boxrule=0.05em},% default=0.5mm
    eval/colback/.style={colback=#1},
    eval/type/.is choice,
    eval/type/normal/.style={},
    eval/type/left/.style={%
        coltitle=black, colbacktitle=white, top=4mm,%
        attach boxed title to top left={yshift=-3mm,xshift=3mm},%
        boxed title style={colframe=white, sharp corners}%
    },
    eval/type/invleft/.style={%
        frame empty, interior empty,%
        coltitle=white, colbacktitle=ColorThemeBlack, top=4mm,%
        extras broken={frame empty, interior empty},%
        borderline={0.5mm}{0mm}{ColorThemeBlack},%
        attach boxed title to top left={yshift=-3mm,xshift=3mm},%
        boxed title style={boxrule=0pt,sharp corners=all}, varwidth boxed title%
    },
    eval/title/.estyle={\ifstrempty{#1}{}{title=#1}}
}
\NewDocumentEnvironment{Note}{o m}{
    \pgfkeys{/pixy/note/.cd, colback, type, sharp=no, trans=no, thin=no}
    \IfValueT{#1}{\pgfkeys{/pixy/note/.cd,#1}}
    \begin{tcolorbox}[%
        breakable, enhanced,%
        fonttitle=\bfseries,%
        top=2mm, bottom=2mm, before skip=3.5mm,%
        /pixy/note/eval/title={#2},%
        /pixy/note/eval/sharp={\pixy@note@sharp},%
        /pixy/note/eval/thin={\pixy@note@thin},%
        /pixy/note/eval/type={\pixy@note@type},%
        /pixy/note/eval/colback={\pixy@note@colback},%
        /pixy/note/eval/trans={\pixy@note@trans}%
    ]
}{\end{tcolorbox}}
% http://tex.bootmath.com/how-to-create-highlight-boxes-in-latex.html
% \def\bracketcolor{red!75!black}
% \def\bracketwidth{3pt}
\def\bracketcolor{black}
\def\bracketwidth{.5pt}
\newtcolorbox{BracketBox}[1][]{%
    breakable,%
    freelance,%
    title=#1,%
    colback=white,%
    colbacktitle=white,%
    coltitle=black,%
    fonttitle=\bfseries,%
    before skip=20pt plus 2pt minus 2pt,%
    after skip=20pt plus 2pt minus 2pt,%
    bottomrule=0pt,%
    boxrule=0pt,%
    colframe=white,%
    overlay unbroken and first={
    \draw[\bracketcolor,line width=\bracketwidth]
        ([xshift=5pt]frame.north west) --
        (frame.north west) --
        (frame.south west);
    \draw[\bracketcolor,line width=\bracketwidth]
        ([xshift=-5pt]frame.north east) --
        (frame.north east) --
        (frame.south east);
    },
    overlay unbroken app={
    \draw[\bracketcolor,line width=\bracketwidth,line cap=rect]
        (frame.south west) --
        ([xshift=5pt]frame.south west);
    \draw[\bracketcolor,line width=\bracketwidth,line cap=rect]
        (frame.south east) --
        ([xshift=-5pt]frame.south east);
    },
    overlay middle and last={
    \draw[\bracketcolor,line width=\bracketwidth]
        (frame.north west) --
        (frame.south west);
    \draw[\bracketcolor,line width=\bracketwidth]
        (frame.north east) --
        (frame.south east);
    },
    overlay last app={
    \draw[\bracketcolor,line width=\bracketwidth,line cap=rect]
        (frame.south west) --
        ([xshift=5pt]frame.south west);
    \draw[\bracketcolor,line width=\bracketwidth,line cap=rect]
        (frame.south east) --
        ([xshift=-5pt]frame.south east);
    },
}
\pgfkeys{
    /pixy/adm/.cd,
    type/.store in=\pixy@adm@type,
    type/.default=note,
    eval/type/.is choice,
    eval/type/note/.style={colframe=ColorThemeFore, colback=ColorThemeBack},
    eval/type/info/.style={colframe=ColorThemePrimary, colback=ColorThemeBack},
    eval/type/warn/.style={colframe=ColorThemeSecondary, colback=ColorThemeBack},
    eval/type/error/.style={colframe=red!75!black, colback=red!5!white},
    eval/title/.estyle={\ifstrempty{#1}{}{title=#1}}
}
\NewDocumentEnvironment{Admonition}{o m}{%
    \pgfkeys{/pixy/adm/.cd, type}
    \IfValueT{#1}{\pgfkeys{/pixy/adm/.cd, #1}}
    \begin{tcolorbox}[%
        breakable,%
        before skip=10pt plus 2pt minus 2pt,%
        after skip=10pt plus 2pt minus 2pt,%
        boxrule=0.4pt,%
        fonttitle=\gtfamily\bfseries,%
        /pixy/adm/eval/title=#2,%
        /pixy/adm/eval/type={\pixy@adm@type}%
    ]
}{\end{tcolorbox}}
\pgfkeys{
    /pixy/label/.cd,
    type/.store in=\pixy@label@type,
    type/.default=normal,
    eval/type/.is choice,
    eval/type/normal/.style={colframe=ColorThemeFore, colback=ColorThemeBack},
    eval/type/main/.style={colframe=ColorThemePrimary, colback=ColorThemeBack},
    eval/type/accent/.style={colframe=ColorThemeSecondary, colback=ColorThemeBack},
    eval/type/inv/.style={colframe=ColorThemeBack, colback=ColorThemeFore},
    eval/type/invmain/.style={colframe=ColorThemePrimary, colback=ColorThemePrimary},
    eval/type/invaccent/.style={colframe=ColorThemeSecondary, colback=ColorThemeSecondary},
    eval/type/gray/.style={colframe=ColorThemeFore, colback=ColorThemeSub1},
    eval/type/blue/.style={colframe=ColorThemePrimary, colback=ColorThemeSub2},
    eval/type/red/.style={colframe=ColorThemeSecondary, colback=ColorThemeSub3},
    eval/type/trans/.style={colframe=ColorThemeBack, colback=ColorThemeBack},
    eval/color/.is choice,
    eval/color/normal/.code={},
    eval/color/main/.code={\color{ColorThemePrimary}},
    eval/color/accent/.code={\color{ColorThemeSecondary}},
    eval/color/inv/.code={\color{ColorThemeBack}},
    eval/color/invmain/.code={\color{ColorThemeBack}},
    eval/color/invaccent/.code={\color{ColorThemeBack}},
    eval/color/gray/.code={\color{ColorThemeFore}},
    eval/color/blue/.code={\color{ColorThemePrimary}},
    eval/color/red/.code={\color{ColorThemeSecondary}},
    eval/color/trans/.code={}
}
\NewDocumentCommand \pixy@labelcbox{m m}{
    \pgfkeys{/pixy/label/.cd, type}
    \IfValueT{#1}{\pgfkeys{/pixy/label/.cd, #1}}
    \tcbox[%
        boxrule=0.4pt, top=0mm, bottom=0mm, left=0mm, right=0mm, on line, arc=0.5mm,%
        /pixy/label/eval/type={\pixy@label@type}%
    ]{\pgfkeys{/pixy/label/eval/color={\pixy@label@type}}\small #2}
}
\NewDocumentCommand \LabelItem{o m m}{
    \item[\pixy@labelcbox{#1}{#2}] #3
}
\NewDocumentCommand \LabelText{o m m}{
    \begin{itemize}
        \LabelItem[#1]{#2}{#3}
    \end{itemize}
}
\makeatother
%------------------------------------------------------------
% MARK: Code
\usepackage{listings}
\usepackage{verbatim}
%------------------------------------------------------------
\newtcbox{\hlbox}[1][]{
    boxrule=0.4pt,
    boxsep=2pt,
    sharp corners,
    colframe=white,
    % colframe=gray!40,
    % colframe=black,
    colback=yellow,
    top=0mm,
    bottom=0mm,
    left=0mm,
    right=0mm,
    on line,
    #1
}
\NewDocumentCommand \Hl{m}{\hlbox{\mbox{\small\rmfamily#1}}}
\NewDocumentCommand \Em{m}{\color{red!80!black}\textbf{#1}\color{black}}
\NewDocumentCommand \EmLight{m}{\color{red!80!black}#1\color{black}}
\definecolor{MSBlue}{rgb}{.204,.353,.541}
\definecolor{MSLightBlue}{rgb}{.31,.506,.741}
\NewDocumentCommand \EmSub{m}{\mbox{\hspace{.3\zw}\color{MSLightBlue}\textbf{#1}\color{black}\hspace{.3\zw}}}

\newtcbox{\embox}{%
    boxrule=0.4pt,%
    colframe=red!75!black,%
    colback=red!75!black,%
    top=0mm, bottom=0mm, left=0mm, right=0mm,%
    on line, arc=0.5mm%
}
\NewDocumentCommand \EmBox{m}{\mbox{\hspace{.3\zw}\embox{\small\color{white} #1\color{black}}\hspace{.3\zw}}}

\newtcbox{\emcbox}{%
    boxrule=0.4pt,%
    colframe=ColorThemeSub1,%
    colback=ColorThemeSub1,%
    top=0mm, bottom=0mm, left=0mm, right=0mm,%
    on line, arc=0.5mm%
}
\NewDocumentCommand \EmSubBox{m}{\mbox{\hspace{.3\zw}\emcbox{\small\bfseries\color{black} #1}\hspace{.3\zw}}}
\NewDocumentCommand \EmCode{m}{\mbox{\hspace{.3\zw}\emcbox{\small\ttfamily\color{black} #1}\hspace{.3\zw}}}
\newtcbox{\Inline}{
    size=fbox, on line,
    colframe=black!60,
    colback=gray!10,
    boxrule=1pt,
    top=0.5mm,bottom=0.5mm,left=0.5mm,right=0.5mm,
    fontupper=\ttfamily\small
}
\definecolor{ColorThemeRound}{rgb}{0.0,0.56,0.0}
\newtcolorbox{RoundBox}[1][ColorThemeRound]{%
    % on line,
    nobeforeafter, math upper, tcbox raise base, enhanced,
    colframe=#1,
    colback=white,
    arc=4pt,
    boxrule=2pt,
    drop fuzzy shadow
}
%------------------------------------------------------------
% MARK: Listing
\renewcommand{\lstlistingname}{リスト}
\lstset{
    tabsize=4,
    breaklines=true,
    breakindent=0pt,
    showspaces=false,
    showstringspaces=false,
    % showlines=false,
    basicstyle=\codefont\pixycodesize,
    % keywordstyle=\bfseries\color{green},
    keywordstyle=\bfseries\color{blue},
    keywordstyle={[2]\color{red!90!black}},
    keywordstyle={[3]\color{red}},
    commentstyle=\itshape\color{purple!40!black},
    % identifierstyle=\color{blue},
    stringstyle=\color{magenta},
    xleftmargin=0mm,
    framexleftmargin=0mm,
    % framesep=0pt,
    framextopmargin=6pt,
    framexbottommargin=6pt,
    % framerule=0.5pt,%
    commentstyle=\color{green!50!black},
    comment=[l]\%\ ,% chktex 26
    % otherkeywords={String,async,await,Task,var},
    % keywords=[2]{DatabaseField,DatabaseTable},
    % keywords=[3]{@},
    % escapeinside={(*@}{@*)},%
    % escapeinside={\%*}{*)},
    % lineskip=-1.ex,%
    % lineskip=-1pt,
    belowcaptionskip=1\baselineskip,
    captionpos=b,
    columns=fixed,
    basewidth=\codebasewidth,
}
\lstdefinestyle{PixyCodeStyle}{
    tabsize=4,
    breaklines=true,
    breakindent=0pt,
    showspaces=false,
    showstringspaces=false,
    % showlines=false,
    basicstyle=\codefont\pixycodesize,
    % keywordstyle=\bfseries\color{green},
    keywordstyle=\bfseries\color{blue},
    keywordstyle={[2]\color{red!90!black}},
    keywordstyle={[3]\color{red}},
    commentstyle=\itshape\color{purple!40!black},
    % identifierstyle=\color{blue},
    stringstyle=\color{magenta},
    % xleftmargin=0mm,
    % framexleftmargin=0mm,
    % framesep=0pt,
    % framextopmargin=6pt,
    % framexbottommargin=6pt,
    % framerule=0.5pt,%
    commentstyle=\color{green!50!black},
    comment=[l]\%\ ,% chktex 26
    % otherkeywords={String,async,await,Task,var},
    % keywords=[2]{DatabaseField,DatabaseTable},
    % keywords=[3]{@},
    % escapeinside={(*@}{@*)},%
    % escapeinside={\%*}{*)},
    % lineskip=-1.ex,%
    % lineskip=-1pt,
    belowcaptionskip=1\baselineskip,
    captionpos=b,
    % columns=fixed,
    basewidth=\codebasewidth,
}
\renewcommand{\thelstnumber}{\arabic{lstnumber}\,:}
\NewTCBListing{Code}{O{} m}{
    breakable,
    enhanced,
    % colframe=gray!20,
    % title=#3,
    boxrule=0.4pt,
    % before skip=10mm,%
    top=0mm, bottom=0mm, middle=0pt, boxsep=0pt,%
    % borderline={0.5mm}{0mm}{ColorThemeBlack},%
    % leftlower=0pt,rightlower=0pt,ColorThemeGraph=0pt,
    colframe=black, colback=black!3!white,%
    % colframe=red!50!black, colback=yellow!10!white,
    sharp corners,%
    listing only,
    listing options={
        % numbers=left,
        % numberstyle={\codefont\pixycodesize},
        % numbersep=1.5\zw,
        % xleftmargin=3\zw,
        % lineskip=-0.5ex,
        style=PixyCodeStyle,
        #2},
    #1
}
\NewTCBListing{NCode}{O{} m}{
    breakable,
    enhanced,
    % colframe=gray!20,
    % title=#3,
    boxrule=0.4pt,
    % before skip=10mm,%
    top=0mm, bottom=0mm, middle=0pt, boxsep=0pt,%
    % borderline={0.5mm}{0mm}{ColorThemeBlack},%
    % leftlower=0pt,rightlower=0pt,ColorThemeGraph=0pt,
    colframe=black, colback=black!3!white,%
    % colframe=red!50!black, colback=yellow!10!white,
    sharp corners,%
    listing only,
    listing options={
        numbers=left,
        numberstyle={\codefont\pixycodesize},
        numbersep=1.5\zw,
        xleftmargin=3\zw,
        % lineskip=-0.5ex,
        style=PixyCodeStyle,
        #2},
    #1
}
\NewTCBListing{Source}{O{} m}{
    breakable,
    enhanced,
    % colframe=gray!20,
    % title=#3,
    boxrule=0.4pt,
    % before skip=10mm,%
    top=0mm, bottom=0mm, middle=0pt, boxsep=0pt,%
    % borderline={0.5mm}{0mm}{ColorThemeBlack},%
    % leftlower=0pt,rightlower=0pt,ColorThemeGraph=0pt,
    colframe=azure3, colback=cyan!3!white,%
    % colframe=red!50!black, colback=yellow!10!white,
    sharp corners,%
    listing only,
    listing options={
        % numbers=left,
        % numberstyle={\codefont\pixycodesize},
        % numbersep=1.5\zw,
        % xleftmargin=3\zw,
        % lineskip=-0.5ex,
        style=PixyCodeStyle,
        #2},
    #1
}
\NewTCBListing{NSource}{O{} m}{
    breakable,
    enhanced,
    % colframe=gray!20,
    % title=#3,
    boxrule=0.4pt,
    % before skip=10mm,%
    top=0mm, bottom=0mm, middle=0pt, boxsep=0pt,%
    % borderline={0.5mm}{0mm}{ColorThemeBlack},%
    % leftlower=0pt,rightlower=0pt,ColorThemeGraph=0pt,
    colframe=azure3, colback=cyan!3!white,%
    % colframe=red!50!black, colback=yellow!10!white,
    sharp corners,%
    listing only,
    listing options={
        numbers=left,
        numberstyle={\codefont\pixycodesize},
        numbersep=1.5\zw,
        xleftmargin=3\zw,
        % lineskip=-0.5ex,
        style=PixyCodeStyle,
        #2},
    #1
}
\NewTCBListing{Pre}{O{} m}{
    breakable,
    enhanced,
    % colframe=gray!20,
    % title=#3,
    boxrule=0.4pt,
    % before skip=10mm,%
    top=0mm, bottom=0mm, middle=0pt, boxsep=0pt,%
    % borderline={0.5mm}{0mm}{ColorThemeBlack},%
    % leftlower=0pt,rightlower=0pt,ColorThemeGraph=0pt,
    colframe=black, colback=white,%
    % colframe=red!50!black, colback=yellow!10!white,
    sharp corners,%
    listing only,
    listing options={
        % numbers=left,
        % numberstyle={\codefont\pixycodesize},
        % numbersep=1.5\zw,
        % xleftmargin=3\zw,
        % lineskip=-0.5ex,
        style=PixyCodeStyle,
        #2},
    #1
}
%------------------------------------------------------------
% MARK: Images
\makeatletter
\newdimen\fb@xsep%
\newdimen\fb@xrule%
\pgfkeys{%
    /pixy/image/.cd,
    size/.is choice,
    size/wf/.style={width},
    size/w0/.style={width=1.0\textwidth},
    size/w9/.style={width=0.9\textwidth},
    size/w8/.style={width=0.8\textwidth},
    size/w7/.style={width=0.7\textwidth},
    size/w6/.style={width=0.6\textwidth},
    size/w5/.style={width=0.5\textwidth},
    size/w4/.style={width=0.4\textwidth},
    size/w3/.style={width=0.3\textwidth},
    size/w2/.style={width=0.2\textwidth},
    size/w1/.style={width=0.1\textwidth},
    size/.default=wf,
    width/.store in=\pixy@image@width,
    width/.default={},
    frame/.store in=\pixy@image@frame,
    frame/.default=yes,
    center/.store in=\pixy@image@center,
    center/.default=yes,
    caption/.store in=\pixy@image@caption,
    caption/.default={},
    label/.store in=\pixy@image@label,
    label/.default={},
    eval/center/.is choice,
    eval/center/no/.code={},
    eval/center/yes/.code={\centering},
    eval/caption/.code={
        \ifthenelse{\equal{#1}{}}{}{\caption{#1}}
    },
    eval/label/.code={
        \ifthenelse{\equal{#1}{}}{}{\label{#1}}
    },
    eval/image/.code n args={3}{
        \ifthenelse{\equal{#1}{}}{
            \includegraphics[#2]{#3}
        }{
            \includegraphics[width=#1,#2]{#3}
        }
    }
}
\NewDocumentCommand \Image{o m m m}{%
    \pgfkeys{/pixy/image/.cd, size, frame=no, center=yes, caption, label}%
    \IfValueT{#1}{\pgfkeys{/pixy/image/.cd, #1}}%
    \begin{figure}[#2]%
        \pgfkeys{/pixy/image/eval/center=\pixy@image@center}%
        \ifthenelse{\equal{\pixy@image@frame}{yes}}{%
            \fb@xsep=\fboxsep%
            \fb@xrule=\fboxrule%
            \fboxsep=0pt%
            \fboxrule=0.5pt%
            \fbox{%
                \pgfkeys{/pixy/image/eval/image={\pixy@image@width}{#3}{#4}}
            }%
            \fboxsep=\fb@xsep%
            \fboxrule=\fb@xrule%
        }{
            \pgfkeys{/pixy/image/eval/image={\pixy@image@width}{#3}{#4}}
        }
        \pgfkeys{/pixy/image/eval/caption={\pixy@image@caption}}%
        \pgfkeys{/pixy/image/eval/label={\pixy@image@label}}%
    \end{figure}%
}
\makeatother
%------------------------------------------------------------
\definecolor{ColorThemeFrameInner}{RGB}{49,44,44}
\newcounter{theorem}
\numberwithin{theorem}{section}
\newtcolorbox{theoremcbox}[1]{%
    enhanced, frame empty, interior empty,
    coltitle=white, fonttitle=\bfseries, colbacktitle=ColorThemeFrameInner,
    extras broken={frame empty, interior empty},
    borderline={0.5mm}{0mm}{ColorThemeFrameInner},
    % sharp corners=downhill,
    sharp corners,
    breakable=true,
    top=4mm,
    before skip=3.5mm,
    attach boxed title to top left={yshift=-3mm,xshift=3mm},
    boxed title style={boxrule=0pt,sharp corners=all}, varwidth boxed title, title=#1}
\NewDocumentEnvironment{Theorem}{O{定理} m}
    {\refstepcounter{theorem}
     \ifstrempty{#2}{\begin{theoremcbox}{#1~\thetheorem}}
     {\begin{theoremcbox}{#1~\thetheorem:~{#2}}}}
    {\end{theoremcbox}}
%------------------------------------------------------------
\makeatletter
\pgfkeys{%
    /pixy/memo/.cd,%
    width/.store in=\pixy@memo@width,
    width/.default=1mm,
    offset/.store in=\pixy@memo@offset,
    offset/.default=0pt,
    color/.store in=\pixy@memo@color,
    color/.default=black
}
\NewDocumentEnvironment{Memo}{o m}{%
    \pgfkeys{/pixy/memo/.cd, width, offset, color}
    \IfValueT{#1}{\pgfkeys{/pixy/memo/.cd, #1}}{}
    \begin{tcolorbox}[
        blanker,
        colback=white,%
        colbacktitle=white,%
        coltitle=black,%
        fonttitle=\bfseries,%
        left=2mm,
        before skip=6pt, after skip=6pt,
        bottomrule=0pt,%
        boxrule=0pt,%
        borderline west={\pixy@memo@width}{\pixy@memo@offset}{\pixy@memo@color},
        #2
    ]
}{\end{tcolorbox}}
\makeatother
%------------------------------------------------------------
% 数式を下線または囲みで表示してその下に文字を表示
\NewDocumentCommand \ExprLine{m m}{%
    \mathop{\underline{#1}}_{\text{\scriptsize#2}}%
}
\NewDocumentCommand \ExprBoxed{m m}{%
    \mathop{\boxed{#1}}_{\text{\scriptsize#2}}%
}
\NewDocumentCommand \ExprNote{O{blue} m}{%
    {\color{#1}\leftarrow\scalebox{0.65}{\(\displaystyle #2\)}}%
}
%------------------------------------------------------------
\renewcommand{\contentsname}{目次}
\renewcommand{\figurename}{図}
% \renewcommand{\figurename}{Fig.}
\renewcommand{\tablename}{表}
%------------------------------------------------------------
% MARK: User definitions
\NewDocumentCommand \RefLabelText{m}{\LabelText[type=gray]{参考}{#1}}
\NewDocumentCommand \ExampleLabelText{m}{\LabelText{例題}{#1}}
\makeatletter
\pgfkeys{%
    /pixy/esent/.cd,%
    top/.store in=\pixy@esent@top,
    top/.default={.5},%
    bottom/.store in=\pixy@esent@bottom,
    bottom/.default={-.5},%
    vspace/.code={\vspace{#1\Cvs}}%
}
\NewDocumentCommand \ExampleSentence{o m m}{
    \pgfkeys{/pixy/esent/.cd,top,bottom}
    \IfValueT{#1}{\pgfkeys{/pixy/esent/.cd,#1}}{}
    \pgfkeys{/pixy/esent/vspace={\pixy@esent@top}}
    \begin{minipage}{\textwidth}
    \begin{itemize}
        \LabelItem[type=gray]{例文}{#2}
        \LabelItem[type=trans]{}{\scriptsize #3}
    \end{itemize}
    \end{minipage}
    \pgfkeys{/pixy/esent/vspace={\pixy@esent@bottom}}
}
\makeatother
\NewDocumentCommand \ExampleSentenceItem{m m}{
    \begin{itemize}
        \LabelItem[type=gray]{例文}{#1}
        \LabelItem[type=trans]{}{\scriptsize #2}
    \end{itemize}
}
\NewDocumentEnvironment{Rule}{O{}}
    {\begin{Theorem}[原則]{#1}}
    {\end{Theorem}}
\makeatletter
\NewDocumentCommand \nobreaklist{}{\par\nobreak\@afterheading}% chktex 21
\NewDocumentCommand \nobreaklistend{}{\vspace{-.5\Cvs}}
\makeatother
\renewcommand{\labelenumi}{\textcircled{\scriptsize \theenumi}}
\renewcommand{\theequation}{\thesection.\arabic{equation}}
\renewcommand{\thefigure}{\thesection.\arabic{figure}}
\renewcommand{\thetable}{\thesection.\arabic{table}}
\numberwithin{equation}{section}
\numberwithin{figure}{section}
\numberwithin{table}{section}
\crefname{equation}{式}{式}
\crefname{figure}{図}{図}
\crefname{table}{表}{表}
% \crefname{algorithm}{Algorithm}{Algorithm}
\newcommand{\crefpairconjunction}{と}
\newcommand{\crefrangeconjunction}{から}
\newcommand{\crefmiddleconjunction}{,}
\newcommand{\creflastconjunction}{,および}
