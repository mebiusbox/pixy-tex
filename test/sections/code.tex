\documentclass[../main]{subfiles}
\begin{document}
\setcounter{section}{3}
\section{コード}
%------------------------------------------------------------

\subsection{lstlisting}
\begin{lstlisting}[language=c++]
#include <cstdio>
void main() {
    std::cout << "Hello, World!" << std::endl;
}
\end{lstlisting}

\subsection{pre, code, Source}
\begin{Pre}{}
#include <cstdio>
void main() {
    std::cout << "Hello, World!" << std::endl;
}
\end{Pre}

\begin{Code}{language=c++}
#include <cstdio>
void main() {
    std::cout << "Hello, World!" << std::endl;
}
\end{Code}

\begin{NCode}{language=c++}
#include <cstdio>
void main() {
    std::cout << "Hello, World!" << std::endl;
}
\end{NCode}

\begin{Source}{language=c++}
#include <cstdio>
void main() {
	std::cout << "Hello, World!" << std::endl;
}
\end{Source}

\begin{NSource}{language=c++}
#include <cstdio>
void main() {
    std::cout << "Hello, World!" << std::endl;
}
\end{NSource}

\subsection{Python}
% from http://wiki.scipy.org/Numpy_Example_List
\begin{Code}{language=python}
>>> from numpy import *
>>> from numpy.fft import *
>>> signal = array([-2., 8., -6., 4., 1., 0., 3., 5.])
>>> fourier = fft(signal)
>>> N = len(signal)
>>> timestep = 0.1 # if unit=day -> freq unit=cycles/day
>>> freq = fftfreq(N, d=timestep) # freqs corresponding to 'fourier'
>>> freq
array([ 0. , 1.25, 2.5 , 3.75, -5. , -3.75, -2.5 , -1.25])
>>> fftshift(freq) # freqs in ascending order
array([-5. , -3.75, -2.5 , -1.25, 0. , 1.25, 2.5 , 3.75])
\end{Code}

\subsection{c++}

\begin{Code}{language=c++}
#include <cstdio>
void main() {
    std::cout << "Hello, World!" << std::endl;
}
\end{Code}

\subsection{scilab}

\begin{NCode}{language=scilab}
N=[10, 1000, 1000000];
for i=1:length(N)
	x=2*rand(1,N(i));
	ans=2*mean(x.*x);
	disp(ans);
end
\end{NCode}

%------------------------------------------------------------
\end{document}
