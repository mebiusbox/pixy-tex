\documentclass[../main]{subfiles}
\begin{document}
\setcounter{section}{3}
\section{コード}
%------------------------------------------------------------

\subsection{lstlisting}
\begin{lstlisting}[language=c++]
#include <cstdio>
void main() {
    std::cout << "Hello, World!" << std::endl;
}
\end{lstlisting}

\subsection{pre}
\begin{pre}[language=c++]
#include <cstdio>
void main() {
    std::cout << "Hello, World!" << std::endl;
}
\end{pre}

\EMMCBOX{numbers=left}のとき

\begin{pre}[language=c++,numbers=left]
#include <cstdio>
void main() {
    std::cout << "Hello, World!" << std::endl;
}
\end{pre}

\subsection{code}
\begin{code}[language=c++]
#include <cstdio>
void main() {
    std::cout << "Hello, World!" << std::endl;
}
\end{code}

\subsection{source}
\begin{source}[language=c++]
#include <cstdio>
void main() {
    std::cout << "Hello, World!" << std::endl;
}
\end{source}

\EMMCBOX{numbers=left}のとき

\begin{source}[language=c++,numbers=left]
#include <cstdio>
void main() {
    std::cout << "Hello, World!" << std::endl;
}
\end{source}

\subsection{code2}
\begin{code2}[language=c++]
#include <cstdio>
void main() {
    std::cout << "Hello, World!" << std::endl;
}
\end{code2}

\subsection{source2}
\begin{source2}[language=c++]
#include <cstdio>
void main() {
    std::cout << "Hello, World!" << std::endl;
}
\end{source2}

\subsection{Java}
\begin{pre}[language=java]
class HelloWorldApp {
    public static void main(String[] args) {
        System.out.println("Hello World!"); // Display the string.
        for (int i = 0; i < 100; ++i) {
            System.out.println(i);
        }
    }
}
\end{pre}

\subsection{Python}
% from http://wiki.scipy.org/Numpy_Example_List
\begin{pre}[language=python]
>>> from numpy import *
>>> from numpy.fft import *
>>> signal = array([-2., 8., -6., 4., 1., 0., 3., 5.])
>>> fourier = fft(signal)
>>> N = len(signal)
>>> timestep = 0.1 # if unit=day -> freq unit=cycles/day
>>> freq = fftfreq(N, d=timestep) # freqs corresponding to 'fourier'
>>> freq
array([ 0. , 1.25, 2.5 , 3.75, -5. , -3.75, -2.5 , -1.25])
>>> fftshift(freq) # freqs in ascending order
array([-5. , -3.75, -2.5 , -1.25, 0. , 1.25, 2.5 , 3.75])
\end{pre}

\subsection{MATLAB/Octave}
% from http://wiki.scipy.org/Numpy_Example_List
\begin{pre}[language=octave]
octave:1> function xdot = f (x, t)
>
>  r = 0.25; k = 1.4;
>  a = 1.5; b = 0.16; c = 0.9; d = 0.8;
>
>  xdot(1) = r*x(1)*(1 - x(1)/k) - a*x(1)*x(2)/(1 + b*x(1));
>  xdot(2) = c*a*x(1)*x(2)/(1 + b*x(1)) - d*x(2);
>
> endfunction
\end{pre}

\subsection{c++}

\begin{pre}[language=c++]
#include <cstdio>
void main() {
    std::cout << "Hello, World!" << std::endl;
}
\end{pre}

\subsection{scilab}

\begin{pre}[language=scilab,numbers=left]
    N=[10, 1000, 1000000];
    for i=1:length(N)
        x=2*rand(1,N(i));
        ans=2*mean(x.*x);
        disp(ans);
    end
\end{pre}

%------------------------------------------------------------
\end{document}