\documentclass[../main]{subfiles}
\begin{document}
\setcounter{section}{2}
\section{数式}
%------------------------------------------------------------

\subsection{Note}
\begin{Note}{平方根の公式}
    $a>0, b>0$のとき,

    \begin{gather*}
        (\sqrt{a})^2 = a, \qquad \sqrt{a^2} = a, \qquad \sqrt{a^2b} = a\sqrt{b} \\
        \sqrt{a}\sqrt{b} = \sqrt{ab}, \qquad \frac{\sqrt{a}}{\sqrt{b}} = \sqrt{\frac{a}{b}} \\
    \end{gather*}
\end{Note}
\begin{Code}{language=tex}
\begin{Note}{平方根の公式}
    $a>0, b>0$のとき,

    \begin{gather*}
        (\sqrt{a})^2 = a, \qquad \sqrt{a^2} = a, \qquad \sqrt{a^2b} = a\sqrt{b} \\
        \sqrt{a}\sqrt{b} = \sqrt{ab}, \qquad \frac{\sqrt{a}}{\sqrt{b}} = \sqrt{\frac{a}{b}} \\
    \end{gather*}
\end{Note}
\end{Code}

\HRuleLeader
\subsection{ExprLine}

\[
    \ExprLine{(ax+b)(cx+d)}{因数分解} = \ExprLine{acx^2 + (ad+bc)x + bd}{展開}
\]
\begin{Code}{language=tex}
\[
    \ExprLine{(ax+b)(cx+d)}{因数分解} = \ExprLine{acx^2 + (ad+bc)x + bd}{展開}
\]
\end{Code}

\subsection{BracketBox}

\begin{BracketBox}

    \ExampleLabelText{$2x^3-3x^2-11x+6\cdots f(x)=0$を満たす$x$を求める.}

	たとえば$x$に$-2$を代入すると

    \begin{align*}
        f(-2) &= 2\cdot (-2)^3 - 3(-2)^2 - 11\cdot (-2)+6 \\
              &= -16-12+22+6 = 0
    \end{align*}

    よって,$f(x)$は$x+2$で割り切れる.

    \begin{equation*}
        \polylongdiv{2x^3-3x^2-11x+6}{x+2}
    \end{equation*}

    \begin{equation*}
        \therefore f(x) = (x+2)(\underline{2x^2-7x+3})
    \end{equation*}

    \begin{equation*}
        2x^3-3x^2-11x+6 = (x+2)(2x-1)(x-3)
    \end{equation*}

\end{BracketBox}
\begin{Code}{language=tex}
\begin{BracketBox}

    \ExampleLabelText{$2x^3-3x^2-11x+6\cdots f(x)=0$を満たす$x$を求める.}

	たとえば$x$に$-2$を代入すると

    \begin{align*}
        f(-2) &= 2\cdot (-2)^3 - 3(-2)^2 - 11\cdot (-2)+6 \\
                &= -16-12+22+6 = 0
    \end{align*}

    よって,$f(x)$は$x+2$で割り切れる.

    \begin{equation*}
        \polylongdiv{2x^3-3x^2-11x+6}{x+2}
    \end{equation*}

    \begin{equation*}
        \therefore f(x) = (x+2)(\underline{2x^2-7x+3})
    \end{equation*}

    \begin{equation*}
        2x^3-3x^2-11x+6 = (x+2)(2x-1)(x-3)
    \end{equation*}

\end{BracketBox}
\end{Code}

\subsection{boxed}

\[
    i \ln i = i\cdot \boxed{i\frac{\pi}{2}} = i^2 \cdot \frac{\pi}{2} = -\frac{\pi}{2}
\]
\begin{Code}{language=tex}
i \ln i = i\cdot \boxed{i\frac{\pi}{2}} &= i^2 \cdot \frac{\pi}{2} = -\frac{\pi}{2}
\end{Code}

\subsection{ExprBoxed}

\begin{align*}
    \EmLight{\{f(x)g(x)\}'} &= \lim_{h\to 0}\frac{F(x+h)-F(x)}{h} \\
                  &= \lim_{h\to 0}\frac{f(x+h)g(x+h)-f(x)g(x)}{h} \\
                  & \text{\scriptsize $f(x)g(x+h)$を引いて足す} \\
                  &= \lim_{h\to 0}\frac{\{f(x+h)-f(x)\}g(x+h)+f(x)\{g(x+h)-g(x)\}}{h} \\
                  &= \lim_{h\to 0}\left\{
                    \ExprBoxed{\frac{f(x+h)-f(x)}{h}}{$=f'(x)$}\cdot
                    \ExprBoxed{g(x+h)}{$=g(x)$}
                    +f(x)\cdot
                    \ExprBoxed{\frac{g(x+h)-g(x)}{h}}{$=g'(x)$}
                    \right\} \\
                  &= \EmLight{f'(x)g(x) + f(x)g'(x)}
\end{align*}
\begin{align*}
    \EmLight{\left\{\frac{f(x)}{g(x)}\right\}'} &=
        \lim_{h\to 0}\frac{F(x+h)-F(x)}{h} \\
        &= \lim_{h\to 0}\frac{\frac{f(x+h)}{g(x+h)}-\frac{f(x)}{g(x)}}{h} \\
        &= \lim_{h\to 0}\frac{
            \frac{f(x+h)g(x)-f(x)g(x+h)}{g(x+h)g(x)}
        }{h} \\
        &= \lim_{h\to 0}\frac{1}{h}
        \frac{\{f(x+h)-f(x)\}g(x)-f(x)\{g(x+h)-g(x)\}}{g(x+h)g(x)} \\
        &= {\scriptsize \lim_{h\to 0}\frac{1}{g(x+h)g(x)}\cdot
        \left\{\
            \ExprBoxed{\frac{f(x+h)-f(x)}{h}}{$=f'(x)$}\cdot
            g(x)-f(x)\cdot
            \ExprBoxed{\frac{g(x+h)-g(x)}{h}}{$=g'(x)$}
        \right\}} \\
        &= \EmLight{\frac{f'(x)g(x)-f(x)g'(x)}{\{g(x)\}^2}}
\end{align*}
\begin{Code}{language=tex}
\begin{align*}
    \EmLight{\{f(x)g(x)\}'} &= \lim_{h\to 0}\frac{F(x+h)-F(x)}{h} \\
                  &= \lim_{h\to 0}\frac{f(x+h)g(x+h)-f(x)g(x)}{h} \\
                  & \text{\scriptsize $f(x)g(x+h)$を引いて足す} \\
                  &= \lim_{h\to 0}\frac{\{f(x+h)-f(x)\}g(x+h)+f(x)\{g(x+h)-g(x)\}}{h} \\
                  &= \lim_{h\to 0}\left\{
                    \ExprBoxed{\frac{f(x+h)-f(x)}{h}}{$=f'(x)$}\cdot
                    \ExprBoxed{g(x+h)}{$=g(x)$}
                    +f(x)\cdot
                    \ExprBoxed{\frac{g(x+h)-g(x)}{h}}{$=g'(x)$}
                    \right\} \\
                  &= \EmLight{f'(x)g(x) + f(x)g'(x)}
\end{align*}
\begin{align*}
    \EmLight{\left\{\frac{f(x)}{g(x)}\right\}'} &=
        \lim_{h\to 0}\frac{F(x+h)-F(x)}{h} \\
        &= \lim_{h\to 0}\frac{\frac{f(x+h)}{g(x+h)}-\frac{f(x)}{g(x)}}{h} \\
        &= \lim_{h\to 0}\frac{
            \frac{f(x+h)g(x)-f(x)g(x+h)}{g(x+h)g(x)}
        }{h} \\
        &= \lim_{h\to 0}\frac{1}{h}
        \frac{\{f(x+h)-f(x)\}g(x)-f(x)\{g(x+h)-g(x)\}}{g(x+h)g(x)} \\
        &= {\scriptsize \lim_{h\to 0}\frac{1}{g(x+h)g(x)}\cdot
        \left\{\
            \ExprBoxed{\frac{f(x+h)-f(x)}{h}}{$=f'(x)$}\cdot
            g(x)-f(x)\cdot
            \ExprBoxed{\frac{g(x+h)-g(x)}{h}}{$=g'(x)$}
        \right\}} \\
        &= \EmLight{\frac{f'(x)g(x)-f(x)g'(x)}{\{g(x)\}^2}}
\end{align*}
\end{Code}

\HRuleLeader
\subsection{ExprNote}

\begin{align*}
    \EmLight{(\sin x)'} &= \lim_{h\to 0}\frac{\sin(x+h)-\sin x}{h}
    \qquad \ExprNote{f'(x) = \lim_{h\to 0}\frac{f(x+h)-f(x)}{h}} \\
    &= \lim_{h\to 0}\frac{2\cos\left(x+\frac{h}{2}\right)\cdot\sin\frac{h}{2}}{h}
    \qquad \ExprNote{\sin A\sin B = 2\cos\frac{A+B}{2}\sin\frac{A-B}{2}} \\
    &= \lim_{h\to 0}\frac{\sin\frac{h}{2}}{\frac{h}{2}}\cos\left(x+\frac{h}{2}\right)
    \qquad \ExprNote{\lim_{x\to 0}\frac{\sin x}{x} = 1} \\
    &= \EmLight{\cos x} \\
    \EmLight{(\cos x)'} &= \lim_{h\to 0}\frac{\cos(x+h)-\cos x}{h} \\
    &= \lim_{h\to 0}\frac{-2\sin\left(x+\frac{h}{2}\right)\cdot\sin\frac{h}{2}}{h} \\
    &= \lim_{h\to 0}\frac{\sin\frac{h}{2}}{\frac{h}{2}}
    \left\{-\sin\left(x+\frac{h}{2}\right)\right\} \\
    &= \EmLight{-sin x} \\
    \EmLight{(\tan x)'} &= \left(\frac{\sin x}{\cos x}\right)' \\
    &= \frac{(\sin x)'\cos x - \sin x(\cos x)'}{\cos^2 x}
    \qquad \ExprNote{
        \left\{
            \frac{f(x)}{g(x)}
        \right\}' = \frac{f'(x)g(x)-f(x)g'(x)}
        {\left\{g(x)\right\}^2}
    } \\
    &= \frac{\cos^2 x + \sin^2 x}{\cos^2 x} \\
    &= \EmLight{\frac{1}{\cos^2 x}}
\end{align*}
\begin{Code}{language=tex}
    \begin{align*}
        \EmLight{(\sin x)'} &= \lim_{h\to 0}\frac{\sin(x+h)-\sin x}{h}
        \qquad \ExprNote{f'(x) = \lim_{h\to 0}\frac{f(x+h)-f(x)}{h}} \\
        &= \lim_{h\to 0}\frac{2\cos\left(x+\frac{h}{2}\right)\cdot\sin\frac{h}{2}}{h}
        \qquad \ExprNote{\sin A\sin B = 2\cos\frac{A+B}{2}\sin\frac{A-B}{2}} \\
        &= \lim_{h\to 0}\frac{\sin\frac{h}{2}}{\frac{h}{2}}\cos\left(x+\frac{h}{2}\right)
        \qquad \ExprNote{\lim_{x\to 0}\frac{\sin x}{x} = 1} \\
        &= \EmLight{\cos x} \\
        \EmLight{(\cos x)'} &= \lim_{h\to 0}\frac{\cos(x+h)-\cos x}{h} \\
        &= \lim_{h\to 0}\frac{-2\sin\left(x+\frac{h}{2}\right)\cdot\sin\frac{h}{2}}{h} \\
        &= \lim_{h\to 0}\frac{\sin\frac{h}{2}}{\frac{h}{2}}
        \left\{-\sin\left(x+\frac{h}{2}\right)\right\} \\
        &= \EmLight{-sin x} \\
        \EmLight{(\tan x)'} &= \left(\frac{\sin x}{\cos x}\right)' \\
        &= \frac{(\sin x)'\cos x - \sin x(\cos x)'}{\cos^2 x}
        \qquad \ExprNote{
            \left\{
                \frac{f(x)}{g(x)}
            \right\}' = \frac{f'(x)g(x)-f(x)g'(x)}
            {\left\{g(x)\right\}^2}
        } \\
        &= \frac{\cos^2 x + \sin^2 x}{\cos^2 x} \\
        &= \EmLight{\frac{1}{\cos^2 x}}
    \end{align*}
\end{Code}

\subsection{PMat, PVec}

\[
    \PMat{\PVec{1\\2\\3}&\PVec{4\\5\\6}} =
    \PMat{1&4\\2&5\\3&6}
\]
\begin{Code}{language=tex}
\[
    \PMat{\PVec{1\\2\\3}&\PVec{4\\5\\6}} =
    \PMat{1&4\\2&5\\3&6}
\]
\end{Code}

\subsection{PVecs}

\[
    \PMat{\PVecs{1&2&3}\\\PVecs{4&5&6}} =
    \PMat{1&2&3\\4&5&6}
\]
\begin{Code}{language=tex}
\[
    \PMat{\PVecs{1&2&3}\\\PVecs{4&5&6}} =
    \PMat{1&2&3\\4&5&6}
\]
\end{Code}

\HRuleLeader

ここで係数行列の各行を$r_1 = \PVecs{a&b}$, $r_2 = \PVecs{c&d}$と表します.右側に基本変形の内容を記載し,左側は変形後を表しています.

\begin{align*}
    \PMat{ac&bc&\vdots&cp\\ac&ad&\vdots&aq} &
    \quad \begin{matrix}
        r_1\times c\\r_2\times a
    \end{matrix} \\
    \PMat{ac&bc&\vdots&cp\\0&ad-bc&\vdots&aq-cp} &
    \quad \begin{matrix}
        r_2-r_1
    \end{matrix} \\
    \PMat{1&\frac{bc}{ac}&\vdots&\frac{cp}{ac}\\0&ad-bc&\vdots&aq-cp} &
    \quad \begin{matrix}
        r_1\div ac
    \end{matrix} \\
    \PMat{1&\frac{bc}{ac}&\vdots&\frac{cp}{ac}\\0&\frac{bc}{ac}&\vdots&\frac{aq-cp}{ad-bc}\cdot\frac{bc}{ac}} &
    \quad \begin{matrix}
        r_2\times \frac{bc}{ac(ad-bc)}
    \end{matrix} \\
    \PMat{1&0&\vdots&\frac{cp}{ac}-\frac{aq-cp}{ad-bc}\cdot\frac{bc}{ac}\\0&\frac{bc}{ac}&\vdots&\frac{aq-cp}{ad-bc}\cdot\frac{bc}{ac}} &
    \quad \begin{matrix}
        r_1-r_2
    \end{matrix} \\
    \PMat{1&0&\vdots&\frac{dp-bq}{ad-bc}\\0&\frac{bc}{ac}&\vdots&\frac{aq-cp}{ad-bc}\cdot\frac{bc}{ac}} &
    \quad \begin{matrix}
    \end{matrix} \\
    \PMat{1&0&\vdots&\frac{dp-bq}{ad-bc}\\0&1&\vdots&\frac{aq-cp}{ad-bc}} &
    \quad \begin{matrix}
        r_2\div \frac{bc}{ac}
    \end{matrix}
\end{align*}
\begin{Code}{language=tex}
ここで係数行列の各行を$r_1 = \PVecs{a&b}$, $r_2 = \PVecs{c&d}$と表します.右側に基本変形の内容を記載し,左側は変形後を表しています.

\begin{align*}
    \PMat{ac&bc&\vdots&cp\\ac&ad&\vdots&aq} &
    \quad \begin{matrix}
        r_1\times c\\r_2\times a
    \end{matrix} \\
    \PMat{ac&bc&\vdots&cp\\0&ad-bc&\vdots&aq-cp} &
    \quad \begin{matrix}
        r_2-r_1
    \end{matrix} \\
    \PMat{1&\frac{bc}{ac}&\vdots&\frac{cp}{ac}\\0&ad-bc&\vdots&aq-cp} &
    \quad \begin{matrix}
        r_1\div ac
    \end{matrix} \\
    \PMat{1&\frac{bc}{ac}&\vdots&\frac{cp}{ac}\\0&\frac{bc}{ac}&\vdots&\frac{aq-cp}{ad-bc}\cdot\frac{bc}{ac}} &
    \quad \begin{matrix}
        r_2\times \frac{bc}{ac(ad-bc)}
    \end{matrix} \\
    \PMat{1&0&\vdots&\frac{cp}{ac}-\frac{aq-cp}{ad-bc}\cdot\frac{bc}{ac}\\0&\frac{bc}{ac}&\vdots&\frac{aq-cp}{ad-bc}\cdot\frac{bc}{ac}} &
    \quad \begin{matrix}
        r_1-r_2
    \end{matrix} \\
    \PMat{1&0&\vdots&\frac{dp-bq}{ad-bc}\\0&\frac{bc}{ac}&\vdots&\frac{aq-cp}{ad-bc}\cdot\frac{bc}{ac}} &
    \quad \begin{matrix}
    \end{matrix} \\
    \PMat{1&0&\vdots&\frac{dp-bq}{ad-bc}\\0&1&\vdots&\frac{aq-cp}{ad-bc}} &
    \quad \begin{matrix}
        r_2\div \frac{bc}{ac}
    \end{matrix}
\end{align*}
\end{Code}

\subsection{DMat}

\[
    \DMat{a_{11}&a_{12}&a_{13}\\a_{21}&a_{22}&a_{23}\\a_{31}&a_{32}&a_{33}} \Rightarrow
    \DMat{\cdot&a_{12}&a_{13}\\a_{21}&\cdot&\cdot\\\cdot&a_{32}&a_{33}} \Rightarrow
    \DMat{\cdot&a_{12}&\cdot\\a_{21}&\cdot&\cdot\\\cdot&\cdot&a_{33}}
\]
\begin{Code}{language=tex}
\[
    \DMat{a_{11}&a_{12}&a_{13}\\a_{21}&a_{22}&a_{23}\\a_{31}&a_{32}&a_{33}} \Rightarrow
    \DMat{\cdot&a_{12}&a_{13}\\a_{21}&\cdot&\cdot\\\cdot&a_{32}&a_{33}} \Rightarrow
    \DMat{\cdot&a_{12}&\cdot\\a_{21}&\cdot&\cdot\\\cdot&\cdot&a_{33}}
\]
\end{Code}

%------------------------------------------------------------
\end{document}
