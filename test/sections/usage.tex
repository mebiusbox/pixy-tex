\documentclass[../main]{subfiles}
\begin{document}
\renewcommand{\arraystretch}{1.2}
\setlength{\arrayrulewidth}{0.3pt}
\arrayrulecolor{cyan!50}
\setcounter{section}{0}
\section{使い方}
%------------------------------------------------------------
\subsection{はじめに}

このテンプレートは\TeX の勉強もかねてゼロから作ったもので,完全に私好みのデザインにしたものです.PDF形式で一般に公開することと,最終的には出版できるレベルぐらいまでにしたいと思いながら作成しています.

\TeX の環境はTeXLive2022とVisual Studio Codeです.(u)p-\LaTeX から lua\LaTeX に移行しました.Visual Studio Code では拡張機能\EMMCBOX{LaTeX Workshop}を使っています.

新規に文書を作成する場合は\EMMCBOX{pixy-tex/pixylua\_***.tex.sample}からコピーします.基本的に以下の構成になります.

\begin{pre}
/main.tex
/.gitignore
/.chktexrc
/sections/
/images/
\end{pre}

\subsection{セクション}

\EMMCBOX{sections}フォルダ以下にセクションごとのファイルを作成し,\EMMCBOX{main.tex}に追記します.\EMMCBOX{sections}にあるファイルのテンプレートは次のコードです.

\begin{pre}
../main]{subfiles}
\begin{document}
\setcounter{section}{0}
\section{}
...
\subsection{}
...
\end{document}
\end{pre}

セクションごとに分けて出力して,更新されたものだけ印刷というやり方をしていたため,\EMMCBOX{setcounter}でセクション番号を指定しています.特に必要がなければ削除したほうがよいです.

\subsection{画像}

\EMMCBOX{images}フォルダに入れます.\EMMCBOX{graphicspath}で設定していますので\EMMCBOX{includegraphics}等ではファイル名だけで十分です.

\subsection{レイアウト設定}

\EMMCBOX{main.tex}には,A4とA5の用紙サイズ,10ptと11ptのフォントサイズに合わせた設定が含まれています.まず,ドキュメントクラスを設定します.以下から適切なものを1つ選びます.

\begin{pre}
\documentclass[a4paper,10pt]{ltjsarticle}
\def\pixylayout{article_a4_10}
\def\pixypagestyle{plain}
\def\pixyheadrule{0pt}
\input{pixy-tex/pixylua_article_a4_10_two}
\input{pixy-tex/pixylua_article_a4_11}
\documentclass[twocolumn,a4paper,11pt]{ltjsarticle}
\def\pixylayout{article_a4_11}
\def\pixypagestyle{plain}
\def\pixyheadrule{0pt}
\documentclass[a4paper,10pt]{ltjsreport}
\def\pixylayout{report_a4_10}
\def\pixypagestyle{pixypagestyle3}
\def\pixyheadrule{0pt}
\documentclass[a4paper,11pt]{ltjsreport}
\def\pixylayout{report_a4_11}
\def\pixypagestyle{pixypagestyle3}
\def\pixyheadrule{0pt}
\documentclass[a5paper,10pt]{ltjsreport}
\def\pixylayout{report_a5_10}
\def\pixypagestyle{pixypagestyle2}
\def\pixyheadrule{0pt}
\documentclass[a5paper,11pt]{ltjsreport}
\def\pixylayout{report_a5_11}
\def\pixypagestyle{pixypagestyle2}
\def\pixyheadrule{0pt}
\documentclass[a4paper,10pt]{ltjsbook}
\def\pixylayout{book_a4_10}
\def\pixybooktype{single}
\def\pixypagestyle{pixypagestyle6}
\def\pixyheadrule{0pt}
\documentclass[a4paper,11pt]{ltjsbook}
\def\pixylayout{book_a4_11}
\def\pixybooktype{single}
\def\pixypagestyle{pixypagestyle6}
\def\pixyheadrule{0pt}
\end{pre}

各ページレイアウトは\EMMCBOX{pixy-tex/pixylua\_style.tex}を取り込むことで定義されます.

\begin{pre}
% \def\pixylayout{report_a4_11}
% \def\pixylayout{report_a4_10}
% \def\pixylayout{report_a5_11}
% \def\pixylayout{report_a5_10}

% \def\pixypagestyle{empty}
% \def\pixypagestyle{headings}
% \def\pixypagestyle{pixypagestyle1(2,3,4,5,6)}

% \def\pixybooktype{single}
% \def\pixybooktype{double}

% \def\pixysectionstyle{default}
% \def\pixysectionstyle{block}

% \def\pixyheadrule{0pt}

% \def\pixynombrefont{default}
% \def\pixynombrefont{genei}

% \def\pixymainfont{default}
% \def\pixymainfont{Merriweather}
% \def\pixymainfont{Libertine}

% \def\pixyjafont{default}
% \def\pixyjafont{BIZ}
% \def\pixyjafont{DigiKyouka}
% \def\pixyjafont{Yu}

% \def\pixymathfont{default}
% \def\pixymathfont{STIX}
% \def\pixymathfont{XITS}

% \def\pixycodefont{default}
% \def\pixycodefont{Iosevka}
% \def\pixycodefont{Sarasa}
% \def\pixycodefont{Fira}
% \def\pixycodefont{Inconsolata}
%------------------------------------------------------------
\usepackage[noto-otf,no-math,deluxe]{luatexja-preset}
\usepackage{ifthen}
%------------------------------------------------------------
\ifthenelse{\isundefined{\pixylayout}}{
    \def\pixylayout{default}
}{}
\ifthenelse{\isundefined{\pixypagestyle}}{
    \def\pixypagestyle{default}
}{}
\ifthenelse{\isundefined{\pixyheadrule}}{
    \def\pixyheadrule{.4pt}
}{}
\ifthenelse{\isundefined{\pixysectionstyle}}{
    \def\pixysectionstyle{block}
}{}
\ifthenelse{\isundefined{\pixybooktype}}{
    \def\pixybooktype{single}
}{}
\ifthenelse{\isundefined{\pixymainfont}}{
    \def\pixymainfont{Merriweather}
}{}
\ifthenelse{\isundefined{\pixyjafont}}{
    \def\pixyjafont{default}
}{}
\ifthenelse{\isundefined{\pixymathfont}}{
    \def\pixymathfont{default}
}{}
\ifthenelse{\isundefined{\pixycodefont}}{
    \def\pixycodefont{Inconsolata}
}{}
\ifthenelse{\isundefined{\pixynombrefont}}{
    \def\pixynombrefont{default}
}{}
%------------------------------------------------------------
% Layout
%
% \usepackage{geometry}
% \geometry{left=20mm,right=10mm,top=10mm,bottom=15mm}
\def\pixylayouttype{default}
\ifthenelse{\equal{\pixylayout}{article_a4_11}}{% a4paper,11pt
    \setlength{\textheight}{40\baselineskip}
    \def\pixylayouttype{article}
}{}
\ifthenelse{\equal{\pixylayout}{article_a4_10}}{% a4paper,10pt
    \setlength{\textheight}{43\baselineskip}
    \def\pixylayouttype{article}
}{}
\ifthenelse{\equal{\pixylayout}{report_a4_11}}{% a4paper,11pt
    \setlength{\textwidth}{51\zw}
    \setlength{\textheight}{41\baselineskip}
    \def\pixylayouttype{report}
}{}
\ifthenelse{\equal{\pixylayout}{report_a4_10}}{% a4paper,10pt
    \setlength{\textwidth}{57\zw}
    \setlength{\textheight}{45\baselineskip}
    \def\pixylayouttype{report}
}{}
\ifthenelse{\equal{\pixylayout}{report_a5_11}}{% a5paper,11pt
    \setlength{\textwidth}{33\zw}
    \setlength{\textheight}{27\baselineskip}
    \def\pixylayouttype{report}
}{}
\ifthenelse{\equal{\pixylayout}{report_a5_10}}{% a5paper,10pt
    \setlength{\textwidth}{37\zw}
    \setlength{\textheight}{31\baselineskip}
    \def\pixylayouttype{report}
}{}
\ifthenelse{\equal{\pixylayout}{book_a4_11}}{% a4paper,11pt
    \def\pixylayouttype{book}
}{}
\ifthenelse{\equal{\pixylayout}{book_a4_10}}{% a4paper,10pt
    \def\pixylayouttype{book}
}{}
\ifthenelse{\equal{\pixylayouttype}{article}}{
    % \setlength{\leftmargin}{0truecm}
    % \setlength{\topmargin}{0truecm}
    % \setlength{\topskip}{0truecm}
    % \setlength{\footskip}{0truecm}
    % \setlength{\oddsidemargin}{0truecm}
    % \setlength{\evensidemargin}{0truecm}
    % \setlength{\marginparwidth}{0truecm}
    % \setlength{\marginparsep}{1truecm}
    % \setlength{\fullwidth}{\textwidth}
    % \setlength{\hoffset}{-12truemm}
    \ifthenelse{\equal{\pixypagestyle}{default}}{
        \global\setlength{\headsep}{0truecm}
        \global\setlength{\headheight}{0truecm}
        \global\setlength{\voffset}{-12truemm}
    }{
        \global\setlength{\headsep}{0truemm}
        \global\setlength{\headheight}{0truecm}
        % \global\setlength{\voffset}{-20truemm}
    }
    % \ifthenelse{\equal{\pixypagestyle}{pixypagestyle3}}{\global\addtolength{\textheight}{1\baselineskip}}{}
    % \ifthenelse{\equal{\pixypagestyle}{pixypagestyle4}}{\global\addtolength{\textheight}{1\baselineskip}}{}
    % \ifthenelse{\equal{\pixypagestyle}{pixypagestyle5}}{\global\addtolength{\textheight}{1\baselineskip}}{}
    % \ifthenelse{\equal{\pixypagestyle}{pixypagestyle6}}{\global\addtolength{\textheight}{1\baselineskip}}{}
    % \setlength{\parindent}{0em}
    % \linespread{1.2} % 行間
}{}
\ifthenelse{\equal{\pixylayouttype}{report}}{
    \setlength{\leftmargin}{0truecm}
    \setlength{\topmargin}{0truecm}
    \setlength{\oddsidemargin}{0truecm}
    \setlength{\evensidemargin}{0truecm}
    % \setlength{\marginparwidth}{0truecm}
    % \setlength{\marginparsep}{1truecm}
    \setlength{\fullwidth}{\textwidth}
    \setlength{\hoffset}{-12truemm}
    \ifthenelse{\equal{\pixypagestyle}{default}}{
        \global\setlength{\headsep}{0truecm}
        \global\setlength{\headheight}{0truecm}
        \global\setlength{\voffset}{-12truemm}
    }{
        \global\setlength{\headsep}{10truemm}
        % \global\setlength{\headheight}{0truecm}
        \global\setlength{\voffset}{-20truemm}
    }
    \ifthenelse{\equal{\pixypagestyle}{pixypagestyle3}}{\global\addtolength{\textheight}{1\baselineskip}}{}
    \ifthenelse{\equal{\pixypagestyle}{pixypagestyle4}}{\global\addtolength{\textheight}{1\baselineskip}}{}
    \ifthenelse{\equal{\pixypagestyle}{pixypagestyle5}}{\global\addtolength{\textheight}{1\baselineskip}}{}
    \ifthenelse{\equal{\pixypagestyle}{pixypagestyle6}}{\global\addtolength{\textheight}{1\baselineskip}}{}
    \setlength{\parindent}{0em}
    \linespread{1.2} % 行間
}{}
\ifthenelse{\equal{\pixylayouttype}{book}}{
    \setlength{\leftmargin}{0truecm}
    \setlength{\topmargin}{0truecm}

    \ifthenelse{\equal{\pixybooktype}{double}}{
        \setlength{\oddsidemargin}{10truemm}
        \setlength{\evensidemargin}{0truemm}
    }{% single
        \setlength{\oddsidemargin}{5truemm}
        \setlength{\evensidemargin}{\oddsidemargin}
    }
    \setlength{\marginparwidth}{0truecm}
    \setlength{\marginparsep}{1truecm}
    \setlength{\textwidth}{\fullwidth}
    \setlength{\hoffset}{-10truemm}
    \ifthenelse{\equal{\pixypagestyle}{default}}{
        \global\setlength{\headsep}{0truecm}
        \global\setlength{\headheight}{0truecm}
        \global\setlength{\voffset}{-10truemm}
    }{
        \global\setlength{\headsep}{10truemm}
        % \global\setlength{\headheight}{0truecm}
        \global\setlength{\voffset}{-20truemm}
    }
    \ifthenelse{\equal{\pixypagestyle}{pixypagestyle3}}{\global\addtolength{\textheight}{1\baselineskip}}{}
    \ifthenelse{\equal{\pixypagestyle}{pixypagestyle4}}{\global\addtolength{\textheight}{1\baselineskip}}{}
    \ifthenelse{\equal{\pixypagestyle}{pixypagestyle5}}{\global\addtolength{\textheight}{1\baselineskip}}{}
    \ifthenelse{\equal{\pixypagestyle}{pixypagestyle6}}{\global\addtolength{\textheight}{1\baselineskip}}{}
    \setlength{\parindent}{0em}
    \linespread{1.2} % 行間
}{}
%--------------------------------------------------------
\def\codebasewidth{0.5em}
% \newfontfamily{\merriweather}{Merriweather Sans}[
%     UprightFont=* Regular,
%     BoldFont=* Bold,
%     BoldItalicFont=* Bold Italic,
%     ItalicFont=* Italic,
% ]
\ifthenelse{\equal{\pixycodefont}{default}}{
    \def\codefont{\ttfamily}
}{}
\ifthenelse{\equal{\pixycodefont}{Iosevka}}{
    \newfontfamily{\codefont}{Iosevka Fixed}[
        UprightFont=* Medium,
        BoldFont=* Bold,
        BoldItalicFont=* Bold Italic,
        ItalicFont=* Medium Italic,
    ]
    \setmonofont{Iosevka Fixed Medium}
}{}
\ifthenelse{\equal{\pixycodefont}{IosevkaSS16}}{
    \newfontfamily{\codefont}{Iosevka Fixed SS16}[
        UprightFont=* Medium,
        BoldFont=* Bold,
        BoldItalicFont=* Bold Italic,
        ItalicFont=* Medium Italic,
    ]
    \setmonofont{Iosevka Fixed Medium}
}{}
\ifthenelse{\equal{\pixycodefont}{Sarasa}}{
    \newfontfamily{\codefont}{Sarasa Fixed J}[
        UprightFont=* Semibold,
        BoldFont=* Bold,
        BoldItalicFont=* Bold Italic,
        ItalicFont=* Semibold Italic,
    ]
    \setmonofont{Sarasa Fixed J Semibold}
}{}
\ifthenelse{\equal{\pixycodefont}{Fira}}{
    \newfontfamily{\codefont}{Fira Mono}[
        UprightFont=* Medium,
        BoldFont=* Bold
    ]
    \setmonofont{Fira Mono Medium}
    \def\codebasewidth{0.6em}
}{}
\ifthenelse{\equal{\pixycodefont}{Inconsolata}}{
    \newfontfamily{\codefont}{Inconsolata Condensed}[
        UprightFont=* Medium,
        BoldFont=* Bold
    ]
    \setmonofont{Inconsolata Condensed Medium}
    \def\codebasewidth{0.45em}
}{}
\ifthenelse{\equal{\pixynombrefont}{genei}}{
    \newfontfamily{\genei}{GenEi Nombre Roman}
    \def\nombrefont{\genei}
}{\def\nombrefont{}}
\ifthenelse{\equal{\pixymainfont}{Merriweather}}{
    \setmainfont{Merriweather}[Scale=\Cjascale]
    \setsansfont{Merriweather Sans}[Scale=\Cjascale]
}{}
\ifthenelse{\equal{\pixymainfont}{Libertine}}{
    \setmainfont{Linux Libertine O}[Scale=\Cjascale]
    \setsansfont{Linux Biolinum O}[Scale=\Cjascale]
}{}
\ifthenelse{\equal{\pixyjafont}{BIZ}}{
    \setmainjfont[
        Script=Default,
        YokoFeatures={JFM=prop},
        CharacterWidth=Proportional,
        Kerning=On,
    ]{BIZ UDPMincho Medium}
}{}
\ifthenelse{\equal{\pixyjafont}{DigiKyouka}}{
    \setmainjfont[
        Script=Default,
        YokoFeatures={JFM=prop},
        CharacterWidth=Proportional,
        Kerning=On,
    ]{UD Digi Kyokasho NK-R}
}{}
\ifthenelse{\equal{\pixyjafont}{Yu}}{
    \setmainjfont[
        Script=Default,
        YokoFeatures={JFM=prop},
        CharacterWidth=Proportional,
        Kerning=On,
    ]{Yu Mincho Regular}
}{}

\usepackage{amsmath,amssymb}
\usepackage{mathtools}
\usepackage{polynom}
\usepackage{unicode-math}
\ifthenelse{\equal{\pixymathfont}{default}}{
    \setmathfont{TeXGyrePagella-Math}
}{}
% https://gzutetsu.hatenablog.jp/entry/2021/06/13/140823
\ifthenelse{\equal{\pixymathfont}{STIX}}{
    \unimathsetup{math-style=ISO,bold-style=ISO}
    \setmathfont{STIX Two Math}[Scale=1.1]
    \setmathfont{STIX Two Math}[Scale=1.1,range={"1D454,"1D488},StylisticSet={2},script-features={},sscript-features={}]
}{}
\ifthenelse{\equal{\pixymathfont}{XITS}}{
    \setmathfont{XITS-Math}
}{}
%--------------------------------------------------------
%--------------------------------------------------------
% Section, Subsection
%
\usepackage[explicit]{titlesec}
\usepackage{titleps}
\ifthenelse{\equal{\pixysectionstyle}{block}}{
\renewcommand*{\thesection}{\arabic{section}}
% \titleformat*{\section}{\Large\bfseries}
% \titleformat*{\subsection}{\normalize\bfseries}
%------------------------------
\titleformat{\section}[block]{}{}{0pt}{%
    \normalfont\Large\bfseries \thesection.\hspace{0.5\zw} #1%
}[\normalsize\vspace*{-.15\baselineskip}\titlerule]
\titleformat{\subsection}%
    {\large\bfseries}%
    {\colorbox{black}{\color{white}\thesubsection}}%
    {0.1cm}%
    {\gdef\currentsectiontitle{#1}#1}
\titleformat{\subsubsection}%
    {}%
    {\rlap{\color{MSLightBlue}\rule[-6pt]{\textwidth}{0.8pt}}\colorbox{MSLightBlue}{%
        \raisebox{0pt}[10pt][3pt]{\makebox[30pt]{%
        \normalfont\normalsize\bfseries\color{white}{\thesubsubsection}%
        }}}%
    }
    {0pt}%
    {\normalfont\normalsize\bfseries\color{MSLightBlue}\hspace{1\zw}#1}%
    [\normalsize\vspace*{.3\baselineskip}]
\setcounter{secnumdepth}{3}
}{}
\newpagestyle{pixypagestyle1}{%
\headrule%
\sethead{第~\thechapter~章 \quad \chaptertitle}{}{\thechapter.\thesection.~\sectiontitle}%
\setfoot{}{\nombrefont\usepage}{}%
}
\newpagestyle{pixypagestyle2}{%
\headrule%
\sethead{}{}{第~\thechapter~章 \quad \sectiontitle}%
\setfoot{}{\nombrefont\usepage}{}%
}
\newpagestyle{pixypagestyle3}{%
\sethead{第~\thechapter~章 \quad \chaptertitle}{}{\small \thechapter.\thesection.~\sectiontitle~\(\bullet\)~\nombrefont\usepage}%
\setfoot{}{}{}%
}
\newpagestyle{pixypagestyle4}{%
\sethead{}{}{\small \thesection.~\sectiontitle~\(\bullet\)~\nombrefont\usepage}%
\setfoot{}{}{}%
}
\newpagestyle{pixypagestyle5}[\small]{
    \setheadrule{\pixyheadrule}%
    \sethead[\nombrefont\thepage]%
            []%
            [\thesection~\sectiontitle]%
            {\thesubsection~\subsectiontitle}%
            {}%
            {\nombrefont\thepage}%
}
\newpagestyle{pixypagestyle6}[\small]{
    \setheadrule{\pixyheadrule}%
    \sethead[\usepage~~\(\bullet\)~第~\thechapter~章 \quad \chaptertitle]%
            []%
            []%
            {}%
            {}%
            {\thechapter.\thesection.~\sectiontitle~\(\bullet\)~~\nombrefont\usepage}%
}
\ifthenelse{\equal{\pixypagestyle}{default}}{}{
    \pagestyle{\pixypagestyle}
}

\end{pre}

このページレイアウトを調整できるように、いつかの変数で指定できます.

\subsubsection{pixypagestyle}

ページスタイルです.

% \begin{table}[H]
%     \begin{tabular}{ll} \toprule
%         値 & 説明 \\ \midrule
%         plain & pagestyle\{plain\}と同じ \\
%         empty & pagestyle\{empty\}と同じ \\
%         headings & pagestyle\{headings\}と同じ \\
%         myheadings & pagestyle\{myheadings\}と同じ \\
%         pixypagestyle1 & ヘッダの左側に章,右側に節,フッタの中央にページ番号 \\
%         pixypagestyle2 & ヘッダの右側に節,フッタの中央にページ番号 \\
%         pixypagestyle3 & ヘッダの左側に章,右側に節とページ番号 \\
%         pixypagestyle4 & ヘッダの右側に節とページ番号 \\
%         pixypagestyle5 & 偶数ページのヘッダの左側にページ番号,右側に節,奇数ページの左側に小節,右側にページ番号 \\
%         pixypagestyle6 & 偶数ページのヘッダの左側にページ番号と章,奇数ページの右側に節とページ番号 \\ \bottomrule
%     \end{tabular}
% \end{table}

\begin{table}[H]
    \begin{tabular}{|>{\columncolor{\spColumnColor}}l|l|} \hline
        plain & pagestyle\{plain\}と同じ \\ \hline
        empty & pagestyle\{empty\}と同じ \\ \hline
        headings & pagestyle\{headings\}と同じ \\ \hline
        myheadings & pagestyle\{myheadings\}と同じ \\ \hline
        pixypagestyle1 & ヘッダの左側に章,右側に節,フッタの中央にページ番号 \\ \hline
        pixypagestyle2 & ヘッダの右側に節,フッタの中央にページ番号 \\ \hline
        pixypagestyle3 & ヘッダの左側に章,右側に節とページ番号 \\ \hline
        pixypagestyle4 & ヘッダの右側に節とページ番号 \\ \hline
        pixypagestyle5 & 偶数ページのヘッダの左側にページ番号,右側に節,奇数ページの左側に小節,右側にページ番号 \\ \hline
        pixypagestyle6 & 偶数ページのヘッダの左側にページ番号と章,奇数ページの右側に節とページ番号 \\ \hline
    \end{tabular}
\end{table}

\subsubsection{pixyheadrule}

ヘッダー部の下線の太さを調整できます.

\begin{pre}
\def\pixyheadrule{.4pt}
\end{pre}

\subsubsection{pixysectionstyle}

セクションの装飾を指定します.

% \begin{table}[H]
%     \begin{tabular}{ll} \toprule
%         値 & 説明 \\ \midrule
%         default & 標準のまま \\
%         block & 章に下線、節は黒背景の白文字で囲まれた矩形 \\ \bottomrule
%     \end{tabular}
% \end{table}

\begin{table}[H]
    \begin{tabular}{|>{\columncolor{\spColumnColor}}l|l|} \hline
        default & 標準のまま \\ \hline
        block & 章に下線、節は黒背景の白文字で囲まれた矩形 \\ \hline
    \end{tabular}
\end{table}

\subsubsection{pixybooktype}

奇数・偶数ページの余白タイプを調整します.bookタイプのときのみ有効です.

% \begin{table}[H]
%     \begin{tabular}{ll} \toprule
%         値 & 説明 \\ \midrule
%         single & 単1ページ \\
%         double & 奇数・偶数ページに余白を入れる \\ \bottomrule
%     \end{tabular}
% \end{table}

\begin{table}[H]
    \begin{tabular}{|>{\columncolor{\spColumnColor}}l|l|} \hline
        single & 単1ページ \\ \hline
        double & 奇数・偶数ページに余白を入れる \\ \hline
    \end{tabular}
\end{table}

\subsubsection{pixymainfont}

英文フォントを指定します.

% \begin{table}[H]
%     \begin{tabular}{ll} \toprule
%         値 & 説明 \\ \midrule
%         default & 標準のまま (Noto) \\
%         Merriweather & Merriweather + Merriweather Sans \\
%         Libertine & Libertine + Biolinum \\ \bottomrule
%     \end{tabular}
% \end{table}

\begin{table}[H]
    \begin{tabular}{|>{\columncolor{\spColumnColor}}l|l|} \hline
        default & 標準のまま (Noto) \\ \hline
        Merriweather & Merriweather + Merriweather Sans \\ \hline
        Libertine & Libertine + Biolinum \\ \hline
    \end{tabular}
\end{table}

\subsubsection{pixyjafont}

和文フォントを指定します.

% \begin{table}[H]
%     \begin{tabular}{ll} \toprule
%         値 & 説明 \\ \midrule
%         default & 標準のまま (Noto) \\
%         BIZ & BIZ UDP明朝 (中字) \\
%         DigiKyouka & UD デジタル 教科書 NK-R \\
%         Yu & 游明朝 \\ \bottomrule
%     \end{tabular}
% \end{table}

\begin{table}[H]
    \begin{tabular}{|>{\columncolor{\spColumnColor}}l|l|} \hline
        default & 標準のまま (Noto) \\ \hline
        BIZ & BIZ UDP明朝 (中字) \\ \hline
        DigiKyouka & UD デジタル 教科書 NK-R \\ \hline
        Yu & 游明朝 \\ \hline
    \end{tabular}
\end{table}

\subsubsection{pixymathfont}

数式用フォントを指定します.

% \begin{table}[H]
%     \begin{tabular}{ll} \toprule
%         値 & 説明 \\ \midrule
%         default & TeX Gyre Pagella \\
%         STIX & STIX2 \\
%         XITS & XITS \\ \bottomrule
%     \end{tabular}
% \end{table}

\begin{table}[H]
    \begin{tabular}{|>{\columncolor{\spColumnColor}}l|l|} \hline
        default & TeX Gyre Pagella \\ \hline
        STIX & STIX2 \\ \hline
        XITS & XITS \\ \hline
    \end{tabular}
\end{table}

\subsubsection{pixycodefont}

コード用フォントを指定します.

% \begin{table}[H]
%     \begin{tabular}{ll} \toprule
%         値 & 説明 \\ \midrule
%         default & 標準のまま \\
%         Iosevka & Ioseva Fixed Medium \\
%         IosevkaSS16 & Iosevka Fixed SS16 Medium \\
%         Sarasa & Sarasa Fixed J Semibold \\
%         Fira & Fira Mono Medium \\
%         Inconsolata & Inconsolata Condensed Medium \\ \bottomrule
%     \end{tabular}
% \end{table}

\begin{table}[H]
    \begin{tabular}{|>{\columncolor{\spColumnColor}}l|l|} \hline
        default & 標準のまま \\ \hline
        Iosevka & Ioseva Fixed Medium \\ \hline
        IosevkaSS16 & Iosevka Fixed SS16 Medium \\ \hline
        Sarasa & Sarasa Fixed J Semibold \\ \hline
        Fira & Fira Mono Medium \\ \hline
        Inconsolata & Inconsolata Condensed Medium \\ \hline
    \end{tabular}
\end{table}

\subsubsection{pixynonbrefont}

ノンブレ(ページ番号)用フォントを指定します.

% \begin{table}[H]
%     \begin{tabular}{ll} \toprule
%         値 & 説明 \\ \midrule
%         default & 標準のまま \\
%         genei & 源瑛ノンブレ \\ \bottomrule
%     \end{tabular}
% \end{table}

\begin{table}[H]
    \begin{tabular}{|>{\columncolor{\spColumnColor}}l|l|} \hline
        default & 標準のまま \\ \hline
        genei & 源瑛ノンブレ \\ \hline
    \end{tabular}
\end{table}

\subsection{スタイル以外のコマンド}

スタイル以外のコマンドは\EMMCBOX{pixy-tex/pixylua.tex}に含まれています.

\begin{pre}
    \usepackage{subfiles}
\usepackage{ifthen}
\usepackage{amsmath,amssymb}
\usepackage{mathtools}
\usepackage{polynom}
\usepackage{multicol}
\ifthenelse{\isundefined{\codebasewidth}}{
    \def\codebasewidth{0.6em}
}{}
\ifthenelse{\isundefined{\pixycodesize}}{
    \def\pixycodesize{\small}
}{}
\usepackage[bookmarks=true,bookmarksnumbered=true,colorlinks=true,linkcolor={blue},urlcolor={blue},hyperfootnotes=false,pdfborder={0,0,0},pdfpagemode=UseNone,unicode=true]{hyperref}
\usepackage{bm}
\usepackage{stmaryrd}
%------------------------------------------------------------
% MARK: Common
% #1: 点同士の間隔
\makeatletter
\NewDocumentCommand \pixy@hruledots{O{.25\zw}}{%
	\leavevmode\cleaders\hb@xt@#1{\hss\(\cdot\m@th\)\hss}\hfill\kern\z@\ignorespaces
}
\NewDocumentCommand \HRuleDots{m}{%
	\noindent\pixy@hruledots\ #1 \pixy@hruledots\par%
}
\NewDocumentCommand \HRuleLeader{O{0pt}}{%
	\noindent\hspace{#1}\cleaders\hbox{…………}\hfill \hspace{#1}\par%
}
\makeatother
\def\HRuleDash{\cleaders\hbox\ to 2em{-}\hfill}
\def\Tasuki#1#2#3#4#5#6#7{% chktex 15
\setbox0\hbox{
\setlength\unitlength{0.1in}
\begin{picture}( 3.0000, 1.0000)( 0.0000, 0.0000)
    \put(0,0){\line(3,1){3}}
    \put(0,1){\line(3,-1){3}}
\end{picture}%
}
\begin{array}{cccccl}
#1 & &#3 & \longrightarrow & #5 & \\[-0.5ex]
#2 & \raise1ex\box0 &#4 & \longrightarrow & #6 & (+ \\[0.5ex] \cline{1-5}
& & & & #7 &
\end{array}
}% chktex 9
%
\NewDocumentCommand \Rev{m}{\frac{1}{#1}}
\NewDocumentCommand \Drv{m m}{\frac{d#1}{d#2}}
\NewDocumentCommand \PDrv{m m}{\frac{\partial #1}{\partial #2}}
\NewDocumentCommand \DDrv{m m}{\frac{\delta #1}{\delta #2}}
\NewDocumentCommand \PVec{m}{{\small\begin{pmatrix}#1\end{pmatrix}}}
\NewDocumentCommand \BVec{m}{{\small\begin{bmatrix}#1\end{bmatrix}}}
\NewDocumentCommand \PVecs{m}{{\scriptsize\begin{pmatrix}#1\end{pmatrix}}}
\NewDocumentCommand \BVecs{m}{{\scriptsize\begin{bmatrix}#1\end{bmatrix}}}
\NewDocumentCommand \PMat{O{} m}{{\small\begin{pmatrix}{#1}#2\end{pmatrix}}}
\NewDocumentCommand \BMat{O{} m}{{\small\begin{bmatrix}{#1}#2\end{bmatrix}}}
\NewDocumentCommand \DMat{O{} m}{{\small\left|\begin{matrix}{#1}#2\end{matrix}\right|}}
%\NewDocumentCommand \Vvec{m}{\left(\begin{array}{c}#1\end{array}\right)}
%\NewDocumentCommand \Vmatrix{O{} m}{\left(\begin{array}{#1}#2\end{array}\right)}
%\NewDocumentCommand \Dmatrix{O{} m}{\left|\begin{array}{#1}#2\end{array}\right|}
\NewDocumentCommand \Inversion{m m}{#1^{\!\mbox{\sf\tiny #2}}}
\NewDocumentCommand \Vecbm{m}{\mbox{\boldmath \(#1\)}}
%------------------------------------------------------------
% MARK: Color
\usepackage{color}
\usepackage[x11names]{xcolor}
\definecolor{ColorThemeBack}{RGB}{255,255,255}
\definecolor{ColorThemeFore}{RGB}{77,77,77}
\definecolor{ColorThemePrimary}{RGB}{0,113,188}
\definecolor{ColorThemeSecondary}{RGB}{255,80,80}
\definecolor{ColorThemeSub1}{RGB}{242,242,242}
\definecolor{ColorThemeSub2}{RGB}{226,241,250}
\definecolor{ColorThemeSub3}{RGB}{255,234,234}
\definecolor{ColorThemeBlack}{RGB}{49,44,44}
\definecolor{ColorThemeGraph}{RGB}{204,0,0}
\definecolor{ColorThemeGraphSub}{RGB}{0,80,255}
%------------------------------------------------------------
% MARK: Figure, Table, Graph
\usepackage{tikz}
\usetikzlibrary{calc,intersections,quotes,angles,shapes.misc,patterns,decorations.markings}
\usepackage{tkz-euclide}
% \usepackage{booktabs}
% \usepackage{colortbl}
\usepackage{here}
\graphicspath{{images/}{../images/}}
\usepackage{pgfplots}
\DeclareMathOperator{\CDF}{cdf}

\def\cdf(#1)(#2)(#3){0.5*(1+(erf((#1-#2)/(#3*sqrt(2)))))}%
\tikzset{
    declare function={
        normcdf(\x,\m,\s)=1/(1 + exp(-0.07056*((\x-\m)/\s)^3 - 1.5976*(\x-\m)/\s));
        normpdf(\x,\m,\s)=exp(-(\x-\m)^2/(2*\s)^2)/(sqrt(2*pi)*\s);
        gamma(\n)=(\n-1)!;
        beta(\a,\b)=(gamma(\a)*gamma(\b))/gamma(\a+\b);
        betapdf(\x,\a,\b)=(\x^(\a-1)*(1-\x)^(\b-1))/beta(\a,\b);
    }
}
% \pgfplotsset{compat=1.12}
\usepgfplotslibrary{fillbetween}
\pgfplotsset{compat=newest}
\pgfplotsset{grid style={dashed,gray}}
\usepackage{wrapfig}
\usepackage[hang,small,bf]{caption}
\usepackage[subrefformat=parens]{subcaption}
\captionsetup{compatibility=false}
\usepackage{tabularray}
% Common Table
\makeatletter
\pgfkeys{%
	/pixy/table/.cd,
	rowhdr/.store in=\pixy@table@rowhdr,
	rowhdr/.default=yes,
	rowhdr/.is choice,
	rowhdr/no/.style={rowbg=white,rowfg=black},
	rowhdr/yes/.style={rowbg=azure3,rowfg=white},
	rowbg/.store in=\pixy@table@rowbg,
	rowfg/.store in=\pixy@table@rowfg,
	colhdr/.store in=\pixy@table@colhdr,
	colhdr/.default=yes,
	colhdr/.is choice,
	colhdr/no/.style={colbg=white,colfg=black,col=no},
	colhdr/yes/.style={colbg=azure8,colfg=black,col=yes},
	colbg/.store in=\pixy@table@colbg,
	colfg/.store in=\pixy@table@colfg,
	col/.store in=\pixy@table@col,
	even/.store in=\pixy@table@even,
	even/.default=yes,
	even/.is choice,
	even/no/.style={evenbg=white},
	even/yes/.style={evenbg=azure9},
	evenbg/.store in=\pixy@table@evenbg
}
\NewDocumentEnvironment{Table}{o m +b}{%
	\pgfkeys{/pixy/table/.cd, rowhdr=no, colhdr=no, even=no}
	\IfValueT{#1}{\pgfkeys{/pixy/table/.cd, #1}}{}
	\ifthenelse{\equal{\pixy@table@col}{yes}}{
		\begin{tblr}{colspec={#2},vline{2}={solid},hlines={0.05em,solid,azure2},hline{1,Z}={1.5pt},%
			row{even}={bg={\pixy@table@evenbg}},%
			row{1}={bg={\pixy@table@rowbg},fg={\pixy@table@rowfg}},%
			column{1}={bg={\pixy@table@colbg},fg={\pixy@table@colfg}}}%
			#3
	}{
		\begin{tblr}{colspec={#2},vline{2}={solid},hlines={0.05em,solid,azure2},hline{1,Z}={1.5pt},%
			row{even}={bg={\pixy@table@evenbg}},%
			row{1}={bg={\pixy@table@rowbg},fg={\pixy@table@rowfg}}}%
			#3
	}
	\end{tblr}
}{}
\makeatother
%------------------------------------------------------------
% MARK: Frame, Box
\usepackage[many]{tcolorbox}
\tcbuselibrary{listings}
\usepackage{varwidth}
% 囲み(中央タイトル)
\makeatletter
\pgfkeys{
	/pixy/note/.cd,
	colback/.store in=\pixy@note@colback,
	colback/.default=black!5!white,
	trans/.store in=\pixy@note@trans,
	trans/.default=yes,
	type/.store in=\pixy@note@type,
	type/.default=normal,
	sharp/.store in=\pixy@note@sharp,
	sharp/.default=yes,
	thin/.store in=\pixy@note@thin,
	thin/.default=yes,
	eval/sharp/.is choice,
	eval/sharp/no/.style={},
	eval/sharp/yes/.style={sharp corners},
	eval/trans/.is choice,
	eval/trans/no/.style={},
	eval/trans/yes/.style={colback=white},
	eval/thin/.is choice,
	eval/thin/no/.style={},
	eval/thin/yes/.style={boxrule=0.05em},% default=0.5mm
	eval/colback/.style={colback=#1},
	eval/type/.is choice,
	eval/type/normal/.style={},
	eval/type/left/.style={%
		coltitle=black, colbacktitle=white,%
		attach boxed title to top left={yshift=-3mm,xshift=3mm},%
		boxed title style={colframe=white, sharp corners}%
	},
	eval/type/invleft/.style={%
		breakable, frame empty, interior empty,%
		coltitle=white, colbacktitle=ColorThemeBlack,%
		extras broken={frame empty, interior empty},%
		borderline={0.5mm}{0mm}{ColorThemeBlack},%
		attach boxed title to top left={yshift=-3mm,xshift=3mm},%
		boxed title style={boxrule=0pt,sharp corners=all}, varwidth boxed title%
	},
	eval/title/.estyle={\ifstrempty{#1}{}{title=#1}}
}
\NewDocumentEnvironment{Note}{o m}{
	\pgfkeys{/pixy/note/.cd, colback, type, sharp=no, trans=no, thin=no}
	\IfValueT{#1}{\pgfkeys{/pixy/note/.cd,#1}}
	\begin{tcolorbox}[%
		enhanced,%
		fonttitle=\bfseries,%
		top=4mm, before skip=3.5mm,%
		/pixy/note/eval/title={#2},%
		/pixy/note/eval/sharp={\pixy@note@sharp},%
		/pixy/note/eval/thin={\pixy@note@thin},%
		/pixy/note/eval/type={\pixy@note@type},%
		/pixy/note/eval/colback={\pixy@note@colback},%
		/pixy/note/eval/trans={\pixy@note@trans}%
	]
}{\end{tcolorbox}}
% http://tex.bootmath.com/how-to-create-highlight-boxes-in-latex.html
% \def\bracketcolor{red!75!black}
% \def\bracketwidth{3pt}
\def\bracketcolor{black}
\def\bracketwidth{.5pt}
\newtcolorbox{BracketBox}[1][]{%
    breakable,%
    freelance,%
    title=#1,%
    colback=white,%
    colbacktitle=white,%
    coltitle=black,%
    fonttitle=\bfseries,%
    before skip=20pt plus 2pt minus 2pt,%
    after skip=20pt plus 2pt minus 2pt,%
    bottomrule=0pt,%
    boxrule=0pt,%
    colframe=white,%
    overlay unbroken and first={
    \draw[\bracketcolor,line width=\bracketwidth]
        ([xshift=5pt]frame.north west) --
        (frame.north west) --
        (frame.south west);
    \draw[\bracketcolor,line width=\bracketwidth]
        ([xshift=-5pt]frame.north east) --
        (frame.north east) --
        (frame.south east);
    },
    overlay unbroken app={
    \draw[\bracketcolor,line width=\bracketwidth,line cap=rect]
        (frame.south west) --
        ([xshift=5pt]frame.south west);
    \draw[\bracketcolor,line width=\bracketwidth,line cap=rect]
        (frame.south east) --
        ([xshift=-5pt]frame.south east);
    },
    overlay middle and last={
    \draw[\bracketcolor,line width=\bracketwidth]
        (frame.north west) --
        (frame.south west);
    \draw[\bracketcolor,line width=\bracketwidth]
        (frame.north east) --
        (frame.south east);
    },
    overlay last app={
    \draw[\bracketcolor,line width=\bracketwidth,line cap=rect]
        (frame.south west) --
        ([xshift=5pt]frame.south west);
    \draw[\bracketcolor,line width=\bracketwidth,line cap=rect]
        (frame.south east) --
        ([xshift=-5pt]frame.south east);
    },
}
\pgfkeys{
	/pixy/adm/.cd,
	type/.store in=\pixy@adm@type,
	type/.default=note,
	eval/type/.is choice,
	eval/type/note/.style={colframe=ColorThemeFore, colback=ColorThemeBack},
	eval/type/info/.style={colframe=ColorThemePrimary, colback=ColorThemeBack},
	eval/type/warn/.style={colframe=ColorThemeSecondary, colback=ColorThemeBack},
	eval/type/error/.style={colframe=red!75!black, colback=red!5!white},
	eval/title/.estyle={\ifstrempty{#1}{}{title=#1}}
}
\NewDocumentEnvironment{Admonition}{o m}{%
	\pgfkeys{/pixy/adm/.cd, type}
	\IfValueT{#1}{\pgfkeys{/pixy/adm/.cd, #1}}
	\begin{tcolorbox}[%
		breakable,%
		before skip=20pt plus 2pt minus 2pt,%
		after skip=20pt plus 2pt minus 2pt,%
		boxrule=0.4pt,%
		fonttitle=\gtfamily\bfseries,%
		/pixy/adm/eval/title=#2,%
		/pixy/adm/eval/type={\pixy@adm@type}%
	]
}{\end{tcolorbox}}
\pgfkeys{
	/pixy/label/.cd,
	type/.store in=\pixy@label@type,
	type/.default=normal,
	eval/type/.is choice,
	eval/type/normal/.style={colframe=ColorThemeFore, colback=ColorThemeBack},
	eval/type/main/.style={colframe=ColorThemePrimary, colback=ColorThemeBack},
	eval/type/accent/.style={colframe=ColorThemeSecondary, colback=ColorThemeBack},
	eval/type/inv/.style={colframe=ColorThemeBack, colback=ColorThemeFore},
	eval/type/invmain/.style={colframe=ColorThemePrimary, colback=ColorThemePrimary},
	eval/type/invaccent/.style={colframe=ColorThemeSecondary, colback=ColorThemeSecondary},
	eval/type/gray/.style={colframe=ColorThemeFore, colback=ColorThemeSub1},
	eval/type/blue/.style={colframe=ColorThemePrimary, colback=ColorThemeSub2},
	eval/type/red/.style={colframe=ColorThemeSecondary, colback=ColorThemeSub3},
	eval/type/trans/.style={colframe=ColorThemeBack, colback=ColorThemeBack},
	eval/color/.is choice,
	eval/color/normal/.code={},
	eval/color/main/.code={\color{ColorThemePrimary}},
	eval/color/accent/.code={\color{ColorThemeSecondary}},
	eval/color/inv/.code={\color{ColorThemeBack}},
	eval/color/invmain/.code={\color{ColorThemeBack}},
	eval/color/invaccent/.code={\color{ColorThemeBack}},
	eval/color/gray/.code={\color{ColorThemeFore}},
	eval/color/blue/.code={\color{ColorThemePrimary}},
	eval/color/red/.code={\color{ColorThemeSecondary}},
	eval/color/trans/.code={}
}
\NewDocumentCommand \pixy@labelcbox{m m}{
	\pgfkeys{/pixy/label/.cd, type}
	\IfValueT{#1}{\pgfkeys{/pixy/label/.cd, #1}}
	\tcbox[%
		boxrule=0.4pt, top=0mm, bottom=0mm, left=0mm, right=0mm, on line, arc=0.5mm,%
		/pixy/label/eval/type={\pixy@label@type}%
	]{\pgfkeys{/pixy/label/eval/color={\pixy@label@type}}\small #2}
}
\NewDocumentCommand \LabelItem{o m m}{
	\item[\pixy@labelcbox{#1}{#2}] #3
}
\NewDocumentCommand \LabelText{o m m}{
	\begin{itemize}
		\LabelItem[#1]{#2}{#3}
    \end{itemize}
}
\makeatother
%------------------------------------------------------------
% MARK: Code
\usepackage{listings}
\usepackage{verbatim}
%------------------------------------------------------------
\newtcbox{\hlbox}[1][]{
    boxrule=0.4pt,
    boxsep=2pt,
    sharp corners,
    colframe=white,
    % colframe=gray!40,
    % colframe=black,
    colback=yellow,
    top=0mm,
    bottom=0mm,
    left=0mm,
    right=0mm,
    on line,
    #1
}
\NewDocumentCommand \Hl{m}{\hlbox{\mbox{\small\rmfamily#1}}}
\NewDocumentCommand \Em{m}{\color{red!80!black}\textbf{#1}\color{black}}
\NewDocumentCommand \EmLight{m}{\color{red!80!black}#1\color{black}}
\definecolor{MSBlue}{rgb}{.204,.353,.541}
\definecolor{MSLightBlue}{rgb}{.31,.506,.741}
\NewDocumentCommand \EmSub{m}{\mbox{\hspace{.3\zw}\color{MSLightBlue}\textbf{#1}\color{black}\hspace{.3\zw}}}

\newtcbox{\embox}{%
    boxrule=0.4pt,%
    colframe=red!75!black,%
    colback=red!75!black,%
    top=0mm, bottom=0mm, left=0mm, right=0mm,%
    on line, arc=0.5mm%
}
\NewDocumentCommand \EmBox{m}{\mbox{\hspace{.3\zw}\embox{\small\color{white} #1\color{black}}\hspace{.3\zw}}}

\newtcbox{\emcbox}{%
    boxrule=0.4pt,%
    colframe=ColorThemeSub1,%
    colback=ColorThemeSub1,%
    top=0mm, bottom=0mm, left=0mm, right=0mm,%
    on line, arc=0.5mm%
}
\NewDocumentCommand \EmSubBox{m}{\mbox{\hspace{.3\zw}\emcbox{\small\bfseries\color{black} #1}\hspace{.3\zw}}}
\NewDocumentCommand \EmCode{m}{\mbox{\hspace{.3\zw}\emcbox{\small\ttfamily\color{black} #1}\hspace{.3\zw}}}
\newtcbox{\Inline}{
    size=fbox, on line,
    colframe=black!60,
    colback=gray!10,
    boxrule=1pt,
    top=0.5mm,bottom=0.5mm,left=0.5mm,right=0.5mm,
    fontupper=\ttfamily\small
}
\definecolor{ColorThemeRound}{rgb}{0.0,0.56,0.0}
\newtcolorbox{RoundBox}[1][ColorThemeRound]{%
    % on line,
    nobeforeafter, math upper, tcbox raise base, enhanced,
    colframe=#1,
    colback=white,
    arc=4pt,
    boxrule=2pt,
    drop fuzzy shadow
}
%------------------------------------------------------------
% MARK: Listing
\renewcommand{\lstlistingname}{リスト}
\lstset{
    tabsize=4,
    breaklines=true,
    breakindent=0pt,
    showspaces=false,
    showstringspaces=false,
    % showlines=false,
    basicstyle=\codefont\pixycodesize,
    % keywordstyle=\bfseries\color{green},
    keywordstyle=\bfseries\color{blue},
    keywordstyle={[2]\color{red!90!black}},
    keywordstyle={[3]\color{red}},
    commentstyle=\itshape\color{purple!40!black},
    % identifierstyle=\color{blue},
    stringstyle=\color{magenta},
    xleftmargin=0mm,
    framexleftmargin=0mm,
    % framesep=0pt,
    framextopmargin=6pt,
    framexbottommargin=6pt,
    % framerule=0.5pt,%
    commentstyle=\color{green!50!black},
    comment=[l]\%\ ,% chktex 26
    % otherkeywords={String,async,await,Task,var},
    % keywords=[2]{DatabaseField,DatabaseTable},
    % keywords=[3]{@},
    % escapeinside={(*@}{@*)},%
    % escapeinside={\%*}{*)},
    % lineskip=-1.ex,%
    % lineskip=-1pt,
    belowcaptionskip=1\baselineskip,
    captionpos=b,
    columns=fixed,
    basewidth=\codebasewidth,
}
\lstdefinestyle{PixyCodeStyle}{
    tabsize=4,
    breaklines=true,
    breakindent=0pt,
    showspaces=false,
    showstringspaces=false,
    % showlines=false,
    basicstyle=\codefont\pixycodesize,
    % keywordstyle=\bfseries\color{green},
    keywordstyle=\bfseries\color{blue},
    keywordstyle={[2]\color{red!90!black}},
    keywordstyle={[3]\color{red}},
    commentstyle=\itshape\color{purple!40!black},
    % identifierstyle=\color{blue},
    stringstyle=\color{magenta},
    % xleftmargin=0mm,
    % framexleftmargin=0mm,
    % framesep=0pt,
    % framextopmargin=6pt,
    % framexbottommargin=6pt,
    % framerule=0.5pt,%
    commentstyle=\color{green!50!black},
    comment=[l]\%\ ,% chktex 26
    % otherkeywords={String,async,await,Task,var},
    % keywords=[2]{DatabaseField,DatabaseTable},
    % keywords=[3]{@},
    % escapeinside={(*@}{@*)},%
    % escapeinside={\%*}{*)},
    % lineskip=-1.ex,%
    % lineskip=-1pt,
    belowcaptionskip=1\baselineskip,
    captionpos=b,
    % columns=fixed,
    basewidth=\codebasewidth,
}
\renewcommand{\thelstnumber}{\arabic{lstnumber}\,:}
\NewTCBListing{Code}{O{} m}{
    breakable,
	enhanced,
	% colframe=gray!20,
	% title=#3,
	boxrule=0.4pt,
	% before skip=10mm,%
	top=0mm, bottom=0mm, middle=0pt, boxsep=0pt,%
	% borderline={0.5mm}{0mm}{ColorThemeBlack},%
	% leftlower=0pt,rightlower=0pt,ColorThemeGraph=0pt,
	colframe=black, colback=black!3!white,%
	% colframe=red!50!black, colback=yellow!10!white,
	sharp corners,%
	listing only,
	listing options={
		% numbers=left,
		% numberstyle={\codefont\pixycodesize},
		% numbersep=1.5\zw,
		% xleftmargin=3\zw,
		% lineskip=-0.5ex,
		style=PixyCodeStyle,
		#2},
	#1
}
\NewTCBListing{NCode}{O{} m}{
    breakable,
	enhanced,
	% colframe=gray!20,
	% title=#3,
	boxrule=0.4pt,
	% before skip=10mm,%
	top=0mm, bottom=0mm, middle=0pt, boxsep=0pt,%
	% borderline={0.5mm}{0mm}{ColorThemeBlack},%
	% leftlower=0pt,rightlower=0pt,ColorThemeGraph=0pt,
	colframe=black, colback=black!3!white,%
	% colframe=red!50!black, colback=yellow!10!white,
	sharp corners,%
	listing only,
	listing options={
		numbers=left,
		numberstyle={\codefont\pixycodesize},
		numbersep=1.5\zw,
		xleftmargin=3\zw,
		% lineskip=-0.5ex,
		style=PixyCodeStyle,
		#2},
	#1
}
\NewTCBListing{Source}{O{} m}{
    breakable,
	enhanced,
	% colframe=gray!20,
	% title=#3,
	boxrule=0.4pt,
	% before skip=10mm,%
	top=0mm, bottom=0mm, middle=0pt, boxsep=0pt,%
	% borderline={0.5mm}{0mm}{ColorThemeBlack},%
	% leftlower=0pt,rightlower=0pt,ColorThemeGraph=0pt,
	colframe=azure3, colback=cyan!3!white,%
	% colframe=red!50!black, colback=yellow!10!white,
	sharp corners,%
	listing only,
	listing options={
		% numbers=left,
		% numberstyle={\codefont\pixycodesize},
		% numbersep=1.5\zw,
		% xleftmargin=3\zw,
		% lineskip=-0.5ex,
		style=PixyCodeStyle,
		#2},
	#1
}
\NewTCBListing{NSource}{O{} m}{
    breakable,
	enhanced,
	% colframe=gray!20,
	% title=#3,
	boxrule=0.4pt,
	% before skip=10mm,%
	top=0mm, bottom=0mm, middle=0pt, boxsep=0pt,%
	% borderline={0.5mm}{0mm}{ColorThemeBlack},%
	% leftlower=0pt,rightlower=0pt,ColorThemeGraph=0pt,
	colframe=azure3, colback=cyan!3!white,%
	% colframe=red!50!black, colback=yellow!10!white,
	sharp corners,%
	listing only,
	listing options={
		numbers=left,
		numberstyle={\codefont\pixycodesize},
		numbersep=1.5\zw,
		xleftmargin=3\zw,
		% lineskip=-0.5ex,
		style=PixyCodeStyle,
		#2},
	#1
}
\NewTCBListing{Pre}{O{} m}{
    breakable,
	enhanced,
	% colframe=gray!20,
	% title=#3,
	boxrule=0.4pt,
	% before skip=10mm,%
	top=0mm, bottom=0mm, middle=0pt, boxsep=0pt,%
	% borderline={0.5mm}{0mm}{ColorThemeBlack},%
	% leftlower=0pt,rightlower=0pt,ColorThemeGraph=0pt,
	colframe=black, colback=white,%
	% colframe=red!50!black, colback=yellow!10!white,
	sharp corners,%
	listing only,
	listing options={
		% numbers=left,
		% numberstyle={\codefont\pixycodesize},
		% numbersep=1.5\zw,
		% xleftmargin=3\zw,
		% lineskip=-0.5ex,
		style=PixyCodeStyle,
		#2},
	#1
}
%------------------------------------------------------------
% MARK: Images
\makeatletter
\newdimen\fb@xsep%
\newdimen\fb@xrule%
\pgfkeys{%
	/pixy/image/.cd,
	size/.is choice,
	size/wf/.style={width},
	size/w0/.style={width=1.0\textwidth},
	size/w9/.style={width=0.9\textwidth},
	size/w8/.style={width=0.8\textwidth},
	size/w7/.style={width=0.7\textwidth},
	size/w6/.style={width=0.6\textwidth},
	size/w5/.style={width=0.5\textwidth},
	size/w4/.style={width=0.4\textwidth},
	size/w3/.style={width=0.3\textwidth},
	size/w2/.style={width=0.2\textwidth},
	size/w1/.style={width=0.1\textwidth},
	size/.default=wf,
	width/.store in=\pixy@image@width,
	width/.default={},
	frame/.store in=\pixy@image@frame,
	frame/.default=yes,
    center/.store in=\pixy@image@center,
    center/.default=yes,
    caption/.store in=\pixy@image@caption,
    caption/.default={},
    label/.store in=\pixy@image@label,
    label/.default={},
    eval/center/.is choice,
    eval/center/no/.code={},
    eval/center/yes/.code={\centering},
	eval/caption/.code={
        \ifthenelse{\equal{#1}{}}{}{\caption{#1}}
    },
	eval/label/.code={
        \ifthenelse{\equal{#1}{}}{}{\label{#1}}
    },
	eval/image/.code n args={3}{
		\ifthenelse{\equal{#1}{}}{
			\includegraphics[#2]{#3}
		}{
			\includegraphics[width=#1,#2]{#3}
		}
	}
}
\NewDocumentCommand \Image{o m m m}{%
	\pgfkeys{/pixy/image/.cd, size, frame=no, center=yes, caption, label}%
	\IfValueT{#1}{\pgfkeys{/pixy/image/.cd, #1}}%
	\begin{figure}[#2]%
        \pgfkeys{/pixy/image/eval/center=\pixy@image@center}%
		\ifthenelse{\equal{\pixy@image@frame}{yes}}{%
			\fb@xsep=\fboxsep%
    		\fb@xrule=\fboxrule%
    		\fboxsep=0pt%
    		\fboxrule=0.5pt%
			\fbox{%
				\pgfkeys{/pixy/image/eval/image={\pixy@image@width}{#3}{#4}}
			}%
			\fboxsep=\fb@xsep%
    		\fboxrule=\fb@xrule%
		}{
			\pgfkeys{/pixy/image/eval/image={\pixy@image@width}{#3}{#4}}
		}
		\pgfkeys{/pixy/image/eval/caption={\pixy@image@caption}}%
		\pgfkeys{/pixy/image/eval/label={\pixy@image@label}}%
	\end{figure}%
}
\makeatother
%------------------------------------------------------------
\definecolor{ColorThemeFrameInner}{RGB}{49,44,44}
\newcounter{theorem}
\numberwithin{theorem}{section}
\newtcolorbox{theoremcbox}[1]{%
    enhanced, frame empty, interior empty,
    coltitle=white, fonttitle=\bfseries, colbacktitle=ColorThemeFrameInner,
    extras broken={frame empty, interior empty},
    borderline={0.5mm}{0mm}{ColorThemeFrameInner},
    % sharp corners=downhill,
    sharp corners,
    breakable=true,
    top=4mm,
    before skip=3.5mm,
    attach boxed title to top left={yshift=-3mm,xshift=3mm},
    boxed title style={boxrule=0pt,sharp corners=all}, varwidth boxed title, title=#1}
\NewDocumentEnvironment{Theorem}{O{定理} m}
    {\refstepcounter{theorem}
     \ifstrempty{#2}{\begin{theoremcbox}{#1~\thetheorem.}}
     {\begin{theoremcbox}{#1~\thetheorem:~{#2}}}}
    {\end{theoremcbox}}
%------------------------------------------------------------
\makeatletter
\pgfkeys{%
	/pixy/memo/.cd,%
	width/.store in=\pixy@memo@width,
	width/.default=1mm,
	offset/.store in=\pixy@memo@offset,
	offset/.default=0pt,
	color/.store in=\pixy@memo@color,
	color/.default=black
}
\NewDocumentEnvironment{Memo}{o m}{%
	\pgfkeys{/pixy/memo/.cd, width, offset, color}
	\IfValueT{#1}{\pgfkeys{/pixy/memo/.cd, #1}}{}
	\begin{tcolorbox}[
		blanker,
		colback=white,%
		colbacktitle=white,%
		coltitle=black,%
		fonttitle=\bfseries,%
		left=2mm,
		before skip=6pt, after skip=6pt,
		bottomrule=0pt,%
		boxrule=0pt,%
		borderline west={\pixy@memo@width}{\pixy@memo@offset}{\pixy@memo@color},
		#2
	]
}{\end{tcolorbox}}
\makeatother
%------------------------------------------------------------
% 数式を下線または囲みで表示してその下に文字を表示
\NewDocumentCommand \ExprLine{m m}{%
    \mathop{\underline{#1}}_{\text{\scriptsize#2}}%
}
\NewDocumentCommand \ExprBoxed{m m}{%
    \mathop{\boxed{#1}}_{\text{\scriptsize#2}}%
}
\NewDocumentCommand \ExprNote{O{blue} m}{%
    {\color{#1}\leftarrow\scalebox{0.65}{\(\displaystyle #2\)}}%
}
%------------------------------------------------------------
\renewcommand{\contentsname}{目次}
% \renewcommand{\figurename}{図}
\renewcommand{\figurename}{Fig.}
\renewcommand{\tablename}{表}
%------------------------------------------------------------
% MARK: User definitions
\NewDocumentCommand \RefLabelText{m}{\LabelText[type=gray]{参考}{#1}}
\NewDocumentCommand \ExampleLabelText{m}{\LabelText{例題}{#1}}
\NewDocumentCommand \ExampleSentence{O{.5} m m}{
    \vspace{#1\Cvs}
    \begin{minipage}{\textwidth}
    \begin{itemize}
		\LabelItem[type=gray]{例文}{#2}
		\LabelItem[type=trans]{}{\scriptsize #3}
    \end{itemize}
    \end{minipage}
    \vspace{.5\Cvs}
}
\NewDocumentCommand \ExampleSentenceItem{m m}{
    \begin{itemize}
		\LabelItem[type=gray]{例文}{#1}
		\LabelItem[type=trans]{}{\scriptsize #2}
    \end{itemize}
}
\NewDocumentEnvironment{Rule}{O{}}
    {\begin{Theorem}[原則]{#1}}
    {\end{Theorem}}
\makeatletter
\NewDocumentCommand \nobreaklist{}{\par\nobreak\@afterheading}% chktex 21
\NewDocumentCommand \nobreaklistend{}{\vspace{-.5\Cvs}}
\makeatother
\renewcommand{\labelenumi}{\textcircled{\scriptsize \theenumi}}

\end{pre}

\subsection{この文書のレイアウト構成}

この文書では、次のようなレイアウトになっています

\begin{pre}
\documentclass[a4paper,10pt]{ltjsreport}
\def\pixylayout{report_a4_10}
\def\pixypagestyle{pixypagestyle3}
\def\pixyheadrule{0pt}
\def\pixycodefont{IosevkaSS16}
\def\pixymathfont{STIX}
\def\pixysectionstyle{block}
\def\pixyjafont{DigiKyouka}
% \def\pixylayout{report_a4_11}
% \def\pixylayout{report_a4_10}
% \def\pixylayout{report_a5_11}
% \def\pixylayout{report_a5_10}

% \def\pixypagestyle{empty}
% \def\pixypagestyle{headings}
% \def\pixypagestyle{pixypagestyle1(2,3,4,5,6)}

% \def\pixybooktype{single}
% \def\pixybooktype{double}

% \def\pixysectionstyle{default}
% \def\pixysectionstyle{block}

% \def\pixyheadrule{0pt}

% \def\pixynombrefont{default}
% \def\pixynombrefont{genei}

% \def\pixymainfont{default}
% \def\pixymainfont{Merriweather}
% \def\pixymainfont{Libertine}

% \def\pixyjafont{default}
% \def\pixyjafont{BIZ}
% \def\pixyjafont{DigiKyouka}
% \def\pixyjafont{Yu}

% \def\pixymathfont{default}
% \def\pixymathfont{STIX}
% \def\pixymathfont{XITS}

% \def\pixycodefont{default}
% \def\pixycodefont{Iosevka}
% \def\pixycodefont{Sarasa}
% \def\pixycodefont{Fira}
% \def\pixycodefont{Inconsolata}
%------------------------------------------------------------
\usepackage[noto-otf,no-math,deluxe]{luatexja-preset}
\usepackage{ifthen}
%------------------------------------------------------------
\ifthenelse{\isundefined{\pixylayout}}{
    \def\pixylayout{default}
}{}
\ifthenelse{\isundefined{\pixypagestyle}}{
    \def\pixypagestyle{default}
}{}
\ifthenelse{\isundefined{\pixyheadrule}}{
    \def\pixyheadrule{.4pt}
}{}
\ifthenelse{\isundefined{\pixysectionstyle}}{
    \def\pixysectionstyle{block}
}{}
\ifthenelse{\isundefined{\pixybooktype}}{
    \def\pixybooktype{single}
}{}
\ifthenelse{\isundefined{\pixymainfont}}{
    \def\pixymainfont{Merriweather}
}{}
\ifthenelse{\isundefined{\pixyjafont}}{
    \def\pixyjafont{default}
}{}
\ifthenelse{\isundefined{\pixymathfont}}{
    \def\pixymathfont{default}
}{}
\ifthenelse{\isundefined{\pixycodefont}}{
    \def\pixycodefont{Inconsolata}
}{}
\ifthenelse{\isundefined{\pixynombrefont}}{
    \def\pixynombrefont{default}
}{}
%------------------------------------------------------------
% Layout
%
% \usepackage{geometry}
% \geometry{left=20mm,right=10mm,top=10mm,bottom=15mm}
\def\pixylayouttype{default}
\ifthenelse{\equal{\pixylayout}{article_a4_11}}{% a4paper,11pt
    \setlength{\textheight}{40\baselineskip}
    \def\pixylayouttype{article}
}{}
\ifthenelse{\equal{\pixylayout}{article_a4_10}}{% a4paper,10pt
    \setlength{\textheight}{43\baselineskip}
    \def\pixylayouttype{article}
}{}
\ifthenelse{\equal{\pixylayout}{report_a4_11}}{% a4paper,11pt
    \setlength{\textwidth}{51\zw}
    \setlength{\textheight}{41\baselineskip}
    \def\pixylayouttype{report}
}{}
\ifthenelse{\equal{\pixylayout}{report_a4_10}}{% a4paper,10pt
    \setlength{\textwidth}{57\zw}
    \setlength{\textheight}{45\baselineskip}
    \def\pixylayouttype{report}
}{}
\ifthenelse{\equal{\pixylayout}{report_a5_11}}{% a5paper,11pt
    \setlength{\textwidth}{33\zw}
    \setlength{\textheight}{27\baselineskip}
    \def\pixylayouttype{report}
}{}
\ifthenelse{\equal{\pixylayout}{report_a5_10}}{% a5paper,10pt
    \setlength{\textwidth}{37\zw}
    \setlength{\textheight}{31\baselineskip}
    \def\pixylayouttype{report}
}{}
\ifthenelse{\equal{\pixylayout}{book_a4_11}}{% a4paper,11pt
    \def\pixylayouttype{book}
}{}
\ifthenelse{\equal{\pixylayout}{book_a4_10}}{% a4paper,10pt
    \def\pixylayouttype{book}
}{}
\ifthenelse{\equal{\pixylayouttype}{article}}{
    % \setlength{\leftmargin}{0truecm}
    % \setlength{\topmargin}{0truecm}
    % \setlength{\topskip}{0truecm}
    % \setlength{\footskip}{0truecm}
    % \setlength{\oddsidemargin}{0truecm}
    % \setlength{\evensidemargin}{0truecm}
    % \setlength{\marginparwidth}{0truecm}
    % \setlength{\marginparsep}{1truecm}
    % \setlength{\fullwidth}{\textwidth}
    % \setlength{\hoffset}{-12truemm}
    \ifthenelse{\equal{\pixypagestyle}{default}}{
        \global\setlength{\headsep}{0truecm}
        \global\setlength{\headheight}{0truecm}
        \global\setlength{\voffset}{-12truemm}
    }{
        \global\setlength{\headsep}{0truemm}
        \global\setlength{\headheight}{0truecm}
        % \global\setlength{\voffset}{-20truemm}
    }
    % \ifthenelse{\equal{\pixypagestyle}{pixypagestyle3}}{\global\addtolength{\textheight}{1\baselineskip}}{}
    % \ifthenelse{\equal{\pixypagestyle}{pixypagestyle4}}{\global\addtolength{\textheight}{1\baselineskip}}{}
    % \ifthenelse{\equal{\pixypagestyle}{pixypagestyle5}}{\global\addtolength{\textheight}{1\baselineskip}}{}
    % \ifthenelse{\equal{\pixypagestyle}{pixypagestyle6}}{\global\addtolength{\textheight}{1\baselineskip}}{}
    % \setlength{\parindent}{0em}
    % \linespread{1.2} % 行間
}{}
\ifthenelse{\equal{\pixylayouttype}{report}}{
    \setlength{\leftmargin}{0truecm}
    \setlength{\topmargin}{0truecm}
    \setlength{\oddsidemargin}{0truecm}
    \setlength{\evensidemargin}{0truecm}
    % \setlength{\marginparwidth}{0truecm}
    % \setlength{\marginparsep}{1truecm}
    \setlength{\fullwidth}{\textwidth}
    \setlength{\hoffset}{-12truemm}
    \ifthenelse{\equal{\pixypagestyle}{default}}{
        \global\setlength{\headsep}{0truecm}
        \global\setlength{\headheight}{0truecm}
        \global\setlength{\voffset}{-12truemm}
    }{
        \global\setlength{\headsep}{10truemm}
        % \global\setlength{\headheight}{0truecm}
        \global\setlength{\voffset}{-20truemm}
    }
    \ifthenelse{\equal{\pixypagestyle}{pixypagestyle3}}{\global\addtolength{\textheight}{1\baselineskip}}{}
    \ifthenelse{\equal{\pixypagestyle}{pixypagestyle4}}{\global\addtolength{\textheight}{1\baselineskip}}{}
    \ifthenelse{\equal{\pixypagestyle}{pixypagestyle5}}{\global\addtolength{\textheight}{1\baselineskip}}{}
    \ifthenelse{\equal{\pixypagestyle}{pixypagestyle6}}{\global\addtolength{\textheight}{1\baselineskip}}{}
    \setlength{\parindent}{0em}
    \linespread{1.2} % 行間
}{}
\ifthenelse{\equal{\pixylayouttype}{book}}{
    \setlength{\leftmargin}{0truecm}
    \setlength{\topmargin}{0truecm}

    \ifthenelse{\equal{\pixybooktype}{double}}{
        \setlength{\oddsidemargin}{10truemm}
        \setlength{\evensidemargin}{0truemm}
    }{% single
        \setlength{\oddsidemargin}{5truemm}
        \setlength{\evensidemargin}{\oddsidemargin}
    }
    \setlength{\marginparwidth}{0truecm}
    \setlength{\marginparsep}{1truecm}
    \setlength{\textwidth}{\fullwidth}
    \setlength{\hoffset}{-10truemm}
    \ifthenelse{\equal{\pixypagestyle}{default}}{
        \global\setlength{\headsep}{0truecm}
        \global\setlength{\headheight}{0truecm}
        \global\setlength{\voffset}{-10truemm}
    }{
        \global\setlength{\headsep}{10truemm}
        % \global\setlength{\headheight}{0truecm}
        \global\setlength{\voffset}{-20truemm}
    }
    \ifthenelse{\equal{\pixypagestyle}{pixypagestyle3}}{\global\addtolength{\textheight}{1\baselineskip}}{}
    \ifthenelse{\equal{\pixypagestyle}{pixypagestyle4}}{\global\addtolength{\textheight}{1\baselineskip}}{}
    \ifthenelse{\equal{\pixypagestyle}{pixypagestyle5}}{\global\addtolength{\textheight}{1\baselineskip}}{}
    \ifthenelse{\equal{\pixypagestyle}{pixypagestyle6}}{\global\addtolength{\textheight}{1\baselineskip}}{}
    \setlength{\parindent}{0em}
    \linespread{1.2} % 行間
}{}
%--------------------------------------------------------
\def\codebasewidth{0.5em}
% \newfontfamily{\merriweather}{Merriweather Sans}[
%     UprightFont=* Regular,
%     BoldFont=* Bold,
%     BoldItalicFont=* Bold Italic,
%     ItalicFont=* Italic,
% ]
\ifthenelse{\equal{\pixycodefont}{default}}{
    \def\codefont{\ttfamily}
}{}
\ifthenelse{\equal{\pixycodefont}{Iosevka}}{
    \newfontfamily{\codefont}{Iosevka Fixed}[
        UprightFont=* Medium,
        BoldFont=* Bold,
        BoldItalicFont=* Bold Italic,
        ItalicFont=* Medium Italic,
    ]
    \setmonofont{Iosevka Fixed Medium}
}{}
\ifthenelse{\equal{\pixycodefont}{IosevkaSS16}}{
    \newfontfamily{\codefont}{Iosevka Fixed SS16}[
        UprightFont=* Medium,
        BoldFont=* Bold,
        BoldItalicFont=* Bold Italic,
        ItalicFont=* Medium Italic,
    ]
    \setmonofont{Iosevka Fixed Medium}
}{}
\ifthenelse{\equal{\pixycodefont}{Sarasa}}{
    \newfontfamily{\codefont}{Sarasa Fixed J}[
        UprightFont=* Semibold,
        BoldFont=* Bold,
        BoldItalicFont=* Bold Italic,
        ItalicFont=* Semibold Italic,
    ]
    \setmonofont{Sarasa Fixed J Semibold}
}{}
\ifthenelse{\equal{\pixycodefont}{Fira}}{
    \newfontfamily{\codefont}{Fira Mono}[
        UprightFont=* Medium,
        BoldFont=* Bold
    ]
    \setmonofont{Fira Mono Medium}
    \def\codebasewidth{0.6em}
}{}
\ifthenelse{\equal{\pixycodefont}{Inconsolata}}{
    \newfontfamily{\codefont}{Inconsolata Condensed}[
        UprightFont=* Medium,
        BoldFont=* Bold
    ]
    \setmonofont{Inconsolata Condensed Medium}
    \def\codebasewidth{0.45em}
}{}
\ifthenelse{\equal{\pixynombrefont}{genei}}{
    \newfontfamily{\genei}{GenEi Nombre Roman}
    \def\nombrefont{\genei}
}{\def\nombrefont{}}
\ifthenelse{\equal{\pixymainfont}{Merriweather}}{
    \setmainfont{Merriweather}[Scale=\Cjascale]
    \setsansfont{Merriweather Sans}[Scale=\Cjascale]
}{}
\ifthenelse{\equal{\pixymainfont}{Libertine}}{
    \setmainfont{Linux Libertine O}[Scale=\Cjascale]
    \setsansfont{Linux Biolinum O}[Scale=\Cjascale]
}{}
\ifthenelse{\equal{\pixyjafont}{BIZ}}{
    \setmainjfont[
        Script=Default,
        YokoFeatures={JFM=prop},
        CharacterWidth=Proportional,
        Kerning=On,
    ]{BIZ UDPMincho Medium}
}{}
\ifthenelse{\equal{\pixyjafont}{DigiKyouka}}{
    \setmainjfont[
        Script=Default,
        YokoFeatures={JFM=prop},
        CharacterWidth=Proportional,
        Kerning=On,
    ]{UD Digi Kyokasho NK-R}
}{}
\ifthenelse{\equal{\pixyjafont}{Yu}}{
    \setmainjfont[
        Script=Default,
        YokoFeatures={JFM=prop},
        CharacterWidth=Proportional,
        Kerning=On,
    ]{Yu Mincho Regular}
}{}

\usepackage{amsmath,amssymb}
\usepackage{mathtools}
\usepackage{polynom}
\usepackage{unicode-math}
\ifthenelse{\equal{\pixymathfont}{default}}{
    \setmathfont{TeXGyrePagella-Math}
}{}
% https://gzutetsu.hatenablog.jp/entry/2021/06/13/140823
\ifthenelse{\equal{\pixymathfont}{STIX}}{
    \unimathsetup{math-style=ISO,bold-style=ISO}
    \setmathfont{STIX Two Math}[Scale=1.1]
    \setmathfont{STIX Two Math}[Scale=1.1,range={"1D454,"1D488},StylisticSet={2},script-features={},sscript-features={}]
}{}
\ifthenelse{\equal{\pixymathfont}{XITS}}{
    \setmathfont{XITS-Math}
}{}
%--------------------------------------------------------
%--------------------------------------------------------
% Section, Subsection
%
\usepackage[explicit]{titlesec}
\usepackage{titleps}
\ifthenelse{\equal{\pixysectionstyle}{block}}{
\renewcommand*{\thesection}{\arabic{section}}
% \titleformat*{\section}{\Large\bfseries}
% \titleformat*{\subsection}{\normalize\bfseries}
%------------------------------
\titleformat{\section}[block]{}{}{0pt}{%
    \normalfont\Large\bfseries \thesection.\hspace{0.5\zw} #1%
}[\normalsize\vspace*{-.15\baselineskip}\titlerule]
\titleformat{\subsection}%
    {\large\bfseries}%
    {\colorbox{black}{\color{white}\thesubsection}}%
    {0.1cm}%
    {\gdef\currentsectiontitle{#1}#1}
\titleformat{\subsubsection}%
    {}%
    {\rlap{\color{MSLightBlue}\rule[-6pt]{\textwidth}{0.8pt}}\colorbox{MSLightBlue}{%
        \raisebox{0pt}[10pt][3pt]{\makebox[30pt]{%
        \normalfont\normalsize\bfseries\color{white}{\thesubsubsection}%
        }}}%
    }
    {0pt}%
    {\normalfont\normalsize\bfseries\color{MSLightBlue}\hspace{1\zw}#1}%
    [\normalsize\vspace*{.3\baselineskip}]
\setcounter{secnumdepth}{3}
}{}
\newpagestyle{pixypagestyle1}{%
\headrule%
\sethead{第~\thechapter~章 \quad \chaptertitle}{}{\thechapter.\thesection.~\sectiontitle}%
\setfoot{}{\nombrefont\usepage}{}%
}
\newpagestyle{pixypagestyle2}{%
\headrule%
\sethead{}{}{第~\thechapter~章 \quad \sectiontitle}%
\setfoot{}{\nombrefont\usepage}{}%
}
\newpagestyle{pixypagestyle3}{%
\sethead{第~\thechapter~章 \quad \chaptertitle}{}{\small \thechapter.\thesection.~\sectiontitle~\(\bullet\)~\nombrefont\usepage}%
\setfoot{}{}{}%
}
\newpagestyle{pixypagestyle4}{%
\sethead{}{}{\small \thesection.~\sectiontitle~\(\bullet\)~\nombrefont\usepage}%
\setfoot{}{}{}%
}
\newpagestyle{pixypagestyle5}[\small]{
    \setheadrule{\pixyheadrule}%
    \sethead[\nombrefont\thepage]%
            []%
            [\thesection~\sectiontitle]%
            {\thesubsection~\subsectiontitle}%
            {}%
            {\nombrefont\thepage}%
}
\newpagestyle{pixypagestyle6}[\small]{
    \setheadrule{\pixyheadrule}%
    \sethead[\usepage~~\(\bullet\)~第~\thechapter~章 \quad \chaptertitle]%
            []%
            []%
            {}%
            {}%
            {\thechapter.\thesection.~\sectiontitle~\(\bullet\)~~\nombrefont\usepage}%
}
\ifthenelse{\equal{\pixypagestyle}{default}}{}{
    \pagestyle{\pixypagestyle}
}

\usepackage{subfiles}
\usepackage{ifthen}
\usepackage{amsmath,amssymb}
\usepackage{mathtools}
\usepackage{polynom}
\usepackage{multicol}
\ifthenelse{\isundefined{\codebasewidth}}{
    \def\codebasewidth{0.6em}
}{}
\ifthenelse{\isundefined{\pixycodesize}}{
    \def\pixycodesize{\small}
}{}
\usepackage[bookmarks=true,bookmarksnumbered=true,colorlinks=true,linkcolor={blue},urlcolor={blue},hyperfootnotes=false,pdfborder={0,0,0},pdfpagemode=UseNone,unicode=true]{hyperref}
\usepackage{bm}
\usepackage{stmaryrd}
%------------------------------------------------------------
% MARK: Common
% #1: 点同士の間隔
\makeatletter
\NewDocumentCommand \pixy@hruledots{O{.25\zw}}{%
	\leavevmode\cleaders\hb@xt@#1{\hss\(\cdot\m@th\)\hss}\hfill\kern\z@\ignorespaces
}
\NewDocumentCommand \HRuleDots{m}{%
	\noindent\pixy@hruledots\ #1 \pixy@hruledots\par%
}
\NewDocumentCommand \HRuleLeader{O{0pt}}{%
	\noindent\hspace{#1}\cleaders\hbox{…………}\hfill \hspace{#1}\par%
}
\makeatother
\def\HRuleDash{\cleaders\hbox\ to 2em{-}\hfill}
\def\Tasuki#1#2#3#4#5#6#7{% chktex 15
\setbox0\hbox{
\setlength\unitlength{0.1in}
\begin{picture}( 3.0000, 1.0000)( 0.0000, 0.0000)
    \put(0,0){\line(3,1){3}}
    \put(0,1){\line(3,-1){3}}
\end{picture}%
}
\begin{array}{cccccl}
#1 & &#3 & \longrightarrow & #5 & \\[-0.5ex]
#2 & \raise1ex\box0 &#4 & \longrightarrow & #6 & (+ \\[0.5ex] \cline{1-5}
& & & & #7 &
\end{array}
}% chktex 9
%
\NewDocumentCommand \Rev{m}{\frac{1}{#1}}
\NewDocumentCommand \Drv{m m}{\frac{d#1}{d#2}}
\NewDocumentCommand \PDrv{m m}{\frac{\partial #1}{\partial #2}}
\NewDocumentCommand \DDrv{m m}{\frac{\delta #1}{\delta #2}}
\NewDocumentCommand \PVec{m}{{\small\begin{pmatrix}#1\end{pmatrix}}}
\NewDocumentCommand \BVec{m}{{\small\begin{bmatrix}#1\end{bmatrix}}}
\NewDocumentCommand \PVecs{m}{{\scriptsize\begin{pmatrix}#1\end{pmatrix}}}
\NewDocumentCommand \BVecs{m}{{\scriptsize\begin{bmatrix}#1\end{bmatrix}}}
\NewDocumentCommand \PMat{O{} m}{{\small\begin{pmatrix}{#1}#2\end{pmatrix}}}
\NewDocumentCommand \BMat{O{} m}{{\small\begin{bmatrix}{#1}#2\end{bmatrix}}}
\NewDocumentCommand \DMat{O{} m}{{\small\left|\begin{matrix}{#1}#2\end{matrix}\right|}}
%\NewDocumentCommand \Vvec{m}{\left(\begin{array}{c}#1\end{array}\right)}
%\NewDocumentCommand \Vmatrix{O{} m}{\left(\begin{array}{#1}#2\end{array}\right)}
%\NewDocumentCommand \Dmatrix{O{} m}{\left|\begin{array}{#1}#2\end{array}\right|}
\NewDocumentCommand \Inversion{m m}{#1^{\!\mbox{\sf\tiny #2}}}
\NewDocumentCommand \Vecbm{m}{\mbox{\boldmath \(#1\)}}
%------------------------------------------------------------
% MARK: Color
\usepackage{color}
\usepackage[x11names]{xcolor}
\definecolor{ColorThemeBack}{RGB}{255,255,255}
\definecolor{ColorThemeFore}{RGB}{77,77,77}
\definecolor{ColorThemePrimary}{RGB}{0,113,188}
\definecolor{ColorThemeSecondary}{RGB}{255,80,80}
\definecolor{ColorThemeSub1}{RGB}{242,242,242}
\definecolor{ColorThemeSub2}{RGB}{226,241,250}
\definecolor{ColorThemeSub3}{RGB}{255,234,234}
\definecolor{ColorThemeBlack}{RGB}{49,44,44}
\definecolor{ColorThemeGraph}{RGB}{204,0,0}
\definecolor{ColorThemeGraphSub}{RGB}{0,80,255}
%------------------------------------------------------------
% MARK: Figure, Table, Graph
\usepackage{tikz}
\usetikzlibrary{calc,intersections,quotes,angles,shapes.misc,patterns,decorations.markings}
\usepackage{tkz-euclide}
% \usepackage{booktabs}
% \usepackage{colortbl}
\usepackage{here}
\graphicspath{{images/}{../images/}}
\usepackage{pgfplots}
\DeclareMathOperator{\CDF}{cdf}

\def\cdf(#1)(#2)(#3){0.5*(1+(erf((#1-#2)/(#3*sqrt(2)))))}%
\tikzset{
    declare function={
        normcdf(\x,\m,\s)=1/(1 + exp(-0.07056*((\x-\m)/\s)^3 - 1.5976*(\x-\m)/\s));
        normpdf(\x,\m,\s)=exp(-(\x-\m)^2/(2*\s)^2)/(sqrt(2*pi)*\s);
        gamma(\n)=(\n-1)!;
        beta(\a,\b)=(gamma(\a)*gamma(\b))/gamma(\a+\b);
        betapdf(\x,\a,\b)=(\x^(\a-1)*(1-\x)^(\b-1))/beta(\a,\b);
    }
}
% \pgfplotsset{compat=1.12}
\usepgfplotslibrary{fillbetween}
\pgfplotsset{compat=newest}
\pgfplotsset{grid style={dashed,gray}}
\usepackage{wrapfig}
\usepackage[hang,small,bf]{caption}
\usepackage[subrefformat=parens]{subcaption}
\captionsetup{compatibility=false}
\usepackage{tabularray}
% Common Table
\makeatletter
\pgfkeys{%
	/pixy/table/.cd,
	rowhdr/.store in=\pixy@table@rowhdr,
	rowhdr/.default=yes,
	rowhdr/.is choice,
	rowhdr/no/.style={rowbg=white,rowfg=black},
	rowhdr/yes/.style={rowbg=azure3,rowfg=white},
	rowbg/.store in=\pixy@table@rowbg,
	rowfg/.store in=\pixy@table@rowfg,
	colhdr/.store in=\pixy@table@colhdr,
	colhdr/.default=yes,
	colhdr/.is choice,
	colhdr/no/.style={colbg=white,colfg=black,col=no},
	colhdr/yes/.style={colbg=azure8,colfg=black,col=yes},
	colbg/.store in=\pixy@table@colbg,
	colfg/.store in=\pixy@table@colfg,
	col/.store in=\pixy@table@col,
	even/.store in=\pixy@table@even,
	even/.default=yes,
	even/.is choice,
	even/no/.style={evenbg=white},
	even/yes/.style={evenbg=azure9},
	evenbg/.store in=\pixy@table@evenbg
}
\NewDocumentEnvironment{Table}{o m +b}{%
	\pgfkeys{/pixy/table/.cd, rowhdr=no, colhdr=no, even=no}
	\IfValueT{#1}{\pgfkeys{/pixy/table/.cd, #1}}{}
	\ifthenelse{\equal{\pixy@table@col}{yes}}{
		\begin{tblr}{colspec={#2},vline{2}={solid},hlines={0.05em,solid,azure2},hline{1,Z}={1.5pt},%
			row{even}={bg={\pixy@table@evenbg}},%
			row{1}={bg={\pixy@table@rowbg},fg={\pixy@table@rowfg}},%
			column{1}={bg={\pixy@table@colbg},fg={\pixy@table@colfg}}}%
			#3
	}{
		\begin{tblr}{colspec={#2},vline{2}={solid},hlines={0.05em,solid,azure2},hline{1,Z}={1.5pt},%
			row{even}={bg={\pixy@table@evenbg}},%
			row{1}={bg={\pixy@table@rowbg},fg={\pixy@table@rowfg}}}%
			#3
	}
	\end{tblr}
}{}
\makeatother
%------------------------------------------------------------
% MARK: Frame, Box
\usepackage[many]{tcolorbox}
\tcbuselibrary{listings}
\usepackage{varwidth}
% 囲み(中央タイトル)
\makeatletter
\pgfkeys{
	/pixy/note/.cd,
	colback/.store in=\pixy@note@colback,
	colback/.default=black!5!white,
	trans/.store in=\pixy@note@trans,
	trans/.default=yes,
	type/.store in=\pixy@note@type,
	type/.default=normal,
	sharp/.store in=\pixy@note@sharp,
	sharp/.default=yes,
	thin/.store in=\pixy@note@thin,
	thin/.default=yes,
	eval/sharp/.is choice,
	eval/sharp/no/.style={},
	eval/sharp/yes/.style={sharp corners},
	eval/trans/.is choice,
	eval/trans/no/.style={},
	eval/trans/yes/.style={colback=white},
	eval/thin/.is choice,
	eval/thin/no/.style={},
	eval/thin/yes/.style={boxrule=0.05em},% default=0.5mm
	eval/colback/.style={colback=#1},
	eval/type/.is choice,
	eval/type/normal/.style={},
	eval/type/left/.style={%
		coltitle=black, colbacktitle=white,%
		attach boxed title to top left={yshift=-3mm,xshift=3mm},%
		boxed title style={colframe=white, sharp corners}%
	},
	eval/type/invleft/.style={%
		breakable, frame empty, interior empty,%
		coltitle=white, colbacktitle=ColorThemeBlack,%
		extras broken={frame empty, interior empty},%
		borderline={0.5mm}{0mm}{ColorThemeBlack},%
		attach boxed title to top left={yshift=-3mm,xshift=3mm},%
		boxed title style={boxrule=0pt,sharp corners=all}, varwidth boxed title%
	},
	eval/title/.estyle={\ifstrempty{#1}{}{title=#1}}
}
\NewDocumentEnvironment{Note}{o m}{
	\pgfkeys{/pixy/note/.cd, colback, type, sharp=no, trans=no, thin=no}
	\IfValueT{#1}{\pgfkeys{/pixy/note/.cd,#1}}
	\begin{tcolorbox}[%
		enhanced,%
		fonttitle=\bfseries,%
		top=4mm, before skip=3.5mm,%
		/pixy/note/eval/title={#2},%
		/pixy/note/eval/sharp={\pixy@note@sharp},%
		/pixy/note/eval/thin={\pixy@note@thin},%
		/pixy/note/eval/type={\pixy@note@type},%
		/pixy/note/eval/colback={\pixy@note@colback},%
		/pixy/note/eval/trans={\pixy@note@trans}%
	]
}{\end{tcolorbox}}
% http://tex.bootmath.com/how-to-create-highlight-boxes-in-latex.html
% \def\bracketcolor{red!75!black}
% \def\bracketwidth{3pt}
\def\bracketcolor{black}
\def\bracketwidth{.5pt}
\newtcolorbox{BracketBox}[1][]{%
    breakable,%
    freelance,%
    title=#1,%
    colback=white,%
    colbacktitle=white,%
    coltitle=black,%
    fonttitle=\bfseries,%
    before skip=20pt plus 2pt minus 2pt,%
    after skip=20pt plus 2pt minus 2pt,%
    bottomrule=0pt,%
    boxrule=0pt,%
    colframe=white,%
    overlay unbroken and first={
    \draw[\bracketcolor,line width=\bracketwidth]
        ([xshift=5pt]frame.north west) --
        (frame.north west) --
        (frame.south west);
    \draw[\bracketcolor,line width=\bracketwidth]
        ([xshift=-5pt]frame.north east) --
        (frame.north east) --
        (frame.south east);
    },
    overlay unbroken app={
    \draw[\bracketcolor,line width=\bracketwidth,line cap=rect]
        (frame.south west) --
        ([xshift=5pt]frame.south west);
    \draw[\bracketcolor,line width=\bracketwidth,line cap=rect]
        (frame.south east) --
        ([xshift=-5pt]frame.south east);
    },
    overlay middle and last={
    \draw[\bracketcolor,line width=\bracketwidth]
        (frame.north west) --
        (frame.south west);
    \draw[\bracketcolor,line width=\bracketwidth]
        (frame.north east) --
        (frame.south east);
    },
    overlay last app={
    \draw[\bracketcolor,line width=\bracketwidth,line cap=rect]
        (frame.south west) --
        ([xshift=5pt]frame.south west);
    \draw[\bracketcolor,line width=\bracketwidth,line cap=rect]
        (frame.south east) --
        ([xshift=-5pt]frame.south east);
    },
}
\pgfkeys{
	/pixy/adm/.cd,
	type/.store in=\pixy@adm@type,
	type/.default=note,
	eval/type/.is choice,
	eval/type/note/.style={colframe=ColorThemeFore, colback=ColorThemeBack},
	eval/type/info/.style={colframe=ColorThemePrimary, colback=ColorThemeBack},
	eval/type/warn/.style={colframe=ColorThemeSecondary, colback=ColorThemeBack},
	eval/type/error/.style={colframe=red!75!black, colback=red!5!white},
	eval/title/.estyle={\ifstrempty{#1}{}{title=#1}}
}
\NewDocumentEnvironment{Admonition}{o m}{%
	\pgfkeys{/pixy/adm/.cd, type}
	\IfValueT{#1}{\pgfkeys{/pixy/adm/.cd, #1}}
	\begin{tcolorbox}[%
		breakable,%
		before skip=20pt plus 2pt minus 2pt,%
		after skip=20pt plus 2pt minus 2pt,%
		boxrule=0.4pt,%
		fonttitle=\gtfamily\bfseries,%
		/pixy/adm/eval/title=#2,%
		/pixy/adm/eval/type={\pixy@adm@type}%
	]
}{\end{tcolorbox}}
\pgfkeys{
	/pixy/label/.cd,
	type/.store in=\pixy@label@type,
	type/.default=normal,
	eval/type/.is choice,
	eval/type/normal/.style={colframe=ColorThemeFore, colback=ColorThemeBack},
	eval/type/main/.style={colframe=ColorThemePrimary, colback=ColorThemeBack},
	eval/type/accent/.style={colframe=ColorThemeSecondary, colback=ColorThemeBack},
	eval/type/inv/.style={colframe=ColorThemeBack, colback=ColorThemeFore},
	eval/type/invmain/.style={colframe=ColorThemePrimary, colback=ColorThemePrimary},
	eval/type/invaccent/.style={colframe=ColorThemeSecondary, colback=ColorThemeSecondary},
	eval/type/gray/.style={colframe=ColorThemeFore, colback=ColorThemeSub1},
	eval/type/blue/.style={colframe=ColorThemePrimary, colback=ColorThemeSub2},
	eval/type/red/.style={colframe=ColorThemeSecondary, colback=ColorThemeSub3},
	eval/type/trans/.style={colframe=ColorThemeBack, colback=ColorThemeBack},
	eval/color/.is choice,
	eval/color/normal/.code={},
	eval/color/main/.code={\color{ColorThemePrimary}},
	eval/color/accent/.code={\color{ColorThemeSecondary}},
	eval/color/inv/.code={\color{ColorThemeBack}},
	eval/color/invmain/.code={\color{ColorThemeBack}},
	eval/color/invaccent/.code={\color{ColorThemeBack}},
	eval/color/gray/.code={\color{ColorThemeFore}},
	eval/color/blue/.code={\color{ColorThemePrimary}},
	eval/color/red/.code={\color{ColorThemeSecondary}},
	eval/color/trans/.code={}
}
\NewDocumentCommand \pixy@labelcbox{m m}{
	\pgfkeys{/pixy/label/.cd, type}
	\IfValueT{#1}{\pgfkeys{/pixy/label/.cd, #1}}
	\tcbox[%
		boxrule=0.4pt, top=0mm, bottom=0mm, left=0mm, right=0mm, on line, arc=0.5mm,%
		/pixy/label/eval/type={\pixy@label@type}%
	]{\pgfkeys{/pixy/label/eval/color={\pixy@label@type}}\small #2}
}
\NewDocumentCommand \LabelItem{o m m}{
	\item[\pixy@labelcbox{#1}{#2}] #3
}
\NewDocumentCommand \LabelText{o m m}{
	\begin{itemize}
		\LabelItem[#1]{#2}{#3}
    \end{itemize}
}
\makeatother
%------------------------------------------------------------
% MARK: Code
\usepackage{listings}
\usepackage{verbatim}
%------------------------------------------------------------
\newtcbox{\hlbox}[1][]{
    boxrule=0.4pt,
    boxsep=2pt,
    sharp corners,
    colframe=white,
    % colframe=gray!40,
    % colframe=black,
    colback=yellow,
    top=0mm,
    bottom=0mm,
    left=0mm,
    right=0mm,
    on line,
    #1
}
\NewDocumentCommand \Hl{m}{\hlbox{\mbox{\small\rmfamily#1}}}
\NewDocumentCommand \Em{m}{\color{red!80!black}\textbf{#1}\color{black}}
\NewDocumentCommand \EmLight{m}{\color{red!80!black}#1\color{black}}
\definecolor{MSBlue}{rgb}{.204,.353,.541}
\definecolor{MSLightBlue}{rgb}{.31,.506,.741}
\NewDocumentCommand \EmSub{m}{\mbox{\hspace{.3\zw}\color{MSLightBlue}\textbf{#1}\color{black}\hspace{.3\zw}}}

\newtcbox{\embox}{%
    boxrule=0.4pt,%
    colframe=red!75!black,%
    colback=red!75!black,%
    top=0mm, bottom=0mm, left=0mm, right=0mm,%
    on line, arc=0.5mm%
}
\NewDocumentCommand \EmBox{m}{\mbox{\hspace{.3\zw}\embox{\small\color{white} #1\color{black}}\hspace{.3\zw}}}

\newtcbox{\emcbox}{%
    boxrule=0.4pt,%
    colframe=ColorThemeSub1,%
    colback=ColorThemeSub1,%
    top=0mm, bottom=0mm, left=0mm, right=0mm,%
    on line, arc=0.5mm%
}
\NewDocumentCommand \EmSubBox{m}{\mbox{\hspace{.3\zw}\emcbox{\small\bfseries\color{black} #1}\hspace{.3\zw}}}
\NewDocumentCommand \EmCode{m}{\mbox{\hspace{.3\zw}\emcbox{\small\ttfamily\color{black} #1}\hspace{.3\zw}}}
\newtcbox{\Inline}{
    size=fbox, on line,
    colframe=black!60,
    colback=gray!10,
    boxrule=1pt,
    top=0.5mm,bottom=0.5mm,left=0.5mm,right=0.5mm,
    fontupper=\ttfamily\small
}
\definecolor{ColorThemeRound}{rgb}{0.0,0.56,0.0}
\newtcolorbox{RoundBox}[1][ColorThemeRound]{%
    % on line,
    nobeforeafter, math upper, tcbox raise base, enhanced,
    colframe=#1,
    colback=white,
    arc=4pt,
    boxrule=2pt,
    drop fuzzy shadow
}
%------------------------------------------------------------
% MARK: Listing
\renewcommand{\lstlistingname}{リスト}
\lstset{
    tabsize=4,
    breaklines=true,
    breakindent=0pt,
    showspaces=false,
    showstringspaces=false,
    % showlines=false,
    basicstyle=\codefont\pixycodesize,
    % keywordstyle=\bfseries\color{green},
    keywordstyle=\bfseries\color{blue},
    keywordstyle={[2]\color{red!90!black}},
    keywordstyle={[3]\color{red}},
    commentstyle=\itshape\color{purple!40!black},
    % identifierstyle=\color{blue},
    stringstyle=\color{magenta},
    xleftmargin=0mm,
    framexleftmargin=0mm,
    % framesep=0pt,
    framextopmargin=6pt,
    framexbottommargin=6pt,
    % framerule=0.5pt,%
    commentstyle=\color{green!50!black},
    comment=[l]\%\ ,% chktex 26
    % otherkeywords={String,async,await,Task,var},
    % keywords=[2]{DatabaseField,DatabaseTable},
    % keywords=[3]{@},
    % escapeinside={(*@}{@*)},%
    % escapeinside={\%*}{*)},
    % lineskip=-1.ex,%
    % lineskip=-1pt,
    belowcaptionskip=1\baselineskip,
    captionpos=b,
    columns=fixed,
    basewidth=\codebasewidth,
}
\lstdefinestyle{PixyCodeStyle}{
    tabsize=4,
    breaklines=true,
    breakindent=0pt,
    showspaces=false,
    showstringspaces=false,
    % showlines=false,
    basicstyle=\codefont\pixycodesize,
    % keywordstyle=\bfseries\color{green},
    keywordstyle=\bfseries\color{blue},
    keywordstyle={[2]\color{red!90!black}},
    keywordstyle={[3]\color{red}},
    commentstyle=\itshape\color{purple!40!black},
    % identifierstyle=\color{blue},
    stringstyle=\color{magenta},
    % xleftmargin=0mm,
    % framexleftmargin=0mm,
    % framesep=0pt,
    % framextopmargin=6pt,
    % framexbottommargin=6pt,
    % framerule=0.5pt,%
    commentstyle=\color{green!50!black},
    comment=[l]\%\ ,% chktex 26
    % otherkeywords={String,async,await,Task,var},
    % keywords=[2]{DatabaseField,DatabaseTable},
    % keywords=[3]{@},
    % escapeinside={(*@}{@*)},%
    % escapeinside={\%*}{*)},
    % lineskip=-1.ex,%
    % lineskip=-1pt,
    belowcaptionskip=1\baselineskip,
    captionpos=b,
    % columns=fixed,
    basewidth=\codebasewidth,
}
\renewcommand{\thelstnumber}{\arabic{lstnumber}\,:}
\NewTCBListing{Code}{O{} m}{
    breakable,
	enhanced,
	% colframe=gray!20,
	% title=#3,
	boxrule=0.4pt,
	% before skip=10mm,%
	top=0mm, bottom=0mm, middle=0pt, boxsep=0pt,%
	% borderline={0.5mm}{0mm}{ColorThemeBlack},%
	% leftlower=0pt,rightlower=0pt,ColorThemeGraph=0pt,
	colframe=black, colback=black!3!white,%
	% colframe=red!50!black, colback=yellow!10!white,
	sharp corners,%
	listing only,
	listing options={
		% numbers=left,
		% numberstyle={\codefont\pixycodesize},
		% numbersep=1.5\zw,
		% xleftmargin=3\zw,
		% lineskip=-0.5ex,
		style=PixyCodeStyle,
		#2},
	#1
}
\NewTCBListing{NCode}{O{} m}{
    breakable,
	enhanced,
	% colframe=gray!20,
	% title=#3,
	boxrule=0.4pt,
	% before skip=10mm,%
	top=0mm, bottom=0mm, middle=0pt, boxsep=0pt,%
	% borderline={0.5mm}{0mm}{ColorThemeBlack},%
	% leftlower=0pt,rightlower=0pt,ColorThemeGraph=0pt,
	colframe=black, colback=black!3!white,%
	% colframe=red!50!black, colback=yellow!10!white,
	sharp corners,%
	listing only,
	listing options={
		numbers=left,
		numberstyle={\codefont\pixycodesize},
		numbersep=1.5\zw,
		xleftmargin=3\zw,
		% lineskip=-0.5ex,
		style=PixyCodeStyle,
		#2},
	#1
}
\NewTCBListing{Source}{O{} m}{
    breakable,
	enhanced,
	% colframe=gray!20,
	% title=#3,
	boxrule=0.4pt,
	% before skip=10mm,%
	top=0mm, bottom=0mm, middle=0pt, boxsep=0pt,%
	% borderline={0.5mm}{0mm}{ColorThemeBlack},%
	% leftlower=0pt,rightlower=0pt,ColorThemeGraph=0pt,
	colframe=azure3, colback=cyan!3!white,%
	% colframe=red!50!black, colback=yellow!10!white,
	sharp corners,%
	listing only,
	listing options={
		% numbers=left,
		% numberstyle={\codefont\pixycodesize},
		% numbersep=1.5\zw,
		% xleftmargin=3\zw,
		% lineskip=-0.5ex,
		style=PixyCodeStyle,
		#2},
	#1
}
\NewTCBListing{NSource}{O{} m}{
    breakable,
	enhanced,
	% colframe=gray!20,
	% title=#3,
	boxrule=0.4pt,
	% before skip=10mm,%
	top=0mm, bottom=0mm, middle=0pt, boxsep=0pt,%
	% borderline={0.5mm}{0mm}{ColorThemeBlack},%
	% leftlower=0pt,rightlower=0pt,ColorThemeGraph=0pt,
	colframe=azure3, colback=cyan!3!white,%
	% colframe=red!50!black, colback=yellow!10!white,
	sharp corners,%
	listing only,
	listing options={
		numbers=left,
		numberstyle={\codefont\pixycodesize},
		numbersep=1.5\zw,
		xleftmargin=3\zw,
		% lineskip=-0.5ex,
		style=PixyCodeStyle,
		#2},
	#1
}
\NewTCBListing{Pre}{O{} m}{
    breakable,
	enhanced,
	% colframe=gray!20,
	% title=#3,
	boxrule=0.4pt,
	% before skip=10mm,%
	top=0mm, bottom=0mm, middle=0pt, boxsep=0pt,%
	% borderline={0.5mm}{0mm}{ColorThemeBlack},%
	% leftlower=0pt,rightlower=0pt,ColorThemeGraph=0pt,
	colframe=black, colback=white,%
	% colframe=red!50!black, colback=yellow!10!white,
	sharp corners,%
	listing only,
	listing options={
		% numbers=left,
		% numberstyle={\codefont\pixycodesize},
		% numbersep=1.5\zw,
		% xleftmargin=3\zw,
		% lineskip=-0.5ex,
		style=PixyCodeStyle,
		#2},
	#1
}
%------------------------------------------------------------
% MARK: Images
\makeatletter
\newdimen\fb@xsep%
\newdimen\fb@xrule%
\pgfkeys{%
	/pixy/image/.cd,
	size/.is choice,
	size/wf/.style={width},
	size/w0/.style={width=1.0\textwidth},
	size/w9/.style={width=0.9\textwidth},
	size/w8/.style={width=0.8\textwidth},
	size/w7/.style={width=0.7\textwidth},
	size/w6/.style={width=0.6\textwidth},
	size/w5/.style={width=0.5\textwidth},
	size/w4/.style={width=0.4\textwidth},
	size/w3/.style={width=0.3\textwidth},
	size/w2/.style={width=0.2\textwidth},
	size/w1/.style={width=0.1\textwidth},
	size/.default=wf,
	width/.store in=\pixy@image@width,
	width/.default={},
	frame/.store in=\pixy@image@frame,
	frame/.default=yes,
    center/.store in=\pixy@image@center,
    center/.default=yes,
    caption/.store in=\pixy@image@caption,
    caption/.default={},
    label/.store in=\pixy@image@label,
    label/.default={},
    eval/center/.is choice,
    eval/center/no/.code={},
    eval/center/yes/.code={\centering},
	eval/caption/.code={
        \ifthenelse{\equal{#1}{}}{}{\caption{#1}}
    },
	eval/label/.code={
        \ifthenelse{\equal{#1}{}}{}{\label{#1}}
    },
	eval/image/.code n args={3}{
		\ifthenelse{\equal{#1}{}}{
			\includegraphics[#2]{#3}
		}{
			\includegraphics[width=#1,#2]{#3}
		}
	}
}
\NewDocumentCommand \Image{o m m m}{%
	\pgfkeys{/pixy/image/.cd, size, frame=no, center=yes, caption, label}%
	\IfValueT{#1}{\pgfkeys{/pixy/image/.cd, #1}}%
	\begin{figure}[#2]%
        \pgfkeys{/pixy/image/eval/center=\pixy@image@center}%
		\ifthenelse{\equal{\pixy@image@frame}{yes}}{%
			\fb@xsep=\fboxsep%
    		\fb@xrule=\fboxrule%
    		\fboxsep=0pt%
    		\fboxrule=0.5pt%
			\fbox{%
				\pgfkeys{/pixy/image/eval/image={\pixy@image@width}{#3}{#4}}
			}%
			\fboxsep=\fb@xsep%
    		\fboxrule=\fb@xrule%
		}{
			\pgfkeys{/pixy/image/eval/image={\pixy@image@width}{#3}{#4}}
		}
		\pgfkeys{/pixy/image/eval/caption={\pixy@image@caption}}%
		\pgfkeys{/pixy/image/eval/label={\pixy@image@label}}%
	\end{figure}%
}
\makeatother
%------------------------------------------------------------
\definecolor{ColorThemeFrameInner}{RGB}{49,44,44}
\newcounter{theorem}
\numberwithin{theorem}{section}
\newtcolorbox{theoremcbox}[1]{%
    enhanced, frame empty, interior empty,
    coltitle=white, fonttitle=\bfseries, colbacktitle=ColorThemeFrameInner,
    extras broken={frame empty, interior empty},
    borderline={0.5mm}{0mm}{ColorThemeFrameInner},
    % sharp corners=downhill,
    sharp corners,
    breakable=true,
    top=4mm,
    before skip=3.5mm,
    attach boxed title to top left={yshift=-3mm,xshift=3mm},
    boxed title style={boxrule=0pt,sharp corners=all}, varwidth boxed title, title=#1}
\NewDocumentEnvironment{Theorem}{O{定理} m}
    {\refstepcounter{theorem}
     \ifstrempty{#2}{\begin{theoremcbox}{#1~\thetheorem.}}
     {\begin{theoremcbox}{#1~\thetheorem:~{#2}}}}
    {\end{theoremcbox}}
%------------------------------------------------------------
\makeatletter
\pgfkeys{%
	/pixy/memo/.cd,%
	width/.store in=\pixy@memo@width,
	width/.default=1mm,
	offset/.store in=\pixy@memo@offset,
	offset/.default=0pt,
	color/.store in=\pixy@memo@color,
	color/.default=black
}
\NewDocumentEnvironment{Memo}{o m}{%
	\pgfkeys{/pixy/memo/.cd, width, offset, color}
	\IfValueT{#1}{\pgfkeys{/pixy/memo/.cd, #1}}{}
	\begin{tcolorbox}[
		blanker,
		colback=white,%
		colbacktitle=white,%
		coltitle=black,%
		fonttitle=\bfseries,%
		left=2mm,
		before skip=6pt, after skip=6pt,
		bottomrule=0pt,%
		boxrule=0pt,%
		borderline west={\pixy@memo@width}{\pixy@memo@offset}{\pixy@memo@color},
		#2
	]
}{\end{tcolorbox}}
\makeatother
%------------------------------------------------------------
% 数式を下線または囲みで表示してその下に文字を表示
\NewDocumentCommand \ExprLine{m m}{%
    \mathop{\underline{#1}}_{\text{\scriptsize#2}}%
}
\NewDocumentCommand \ExprBoxed{m m}{%
    \mathop{\boxed{#1}}_{\text{\scriptsize#2}}%
}
\NewDocumentCommand \ExprNote{O{blue} m}{%
    {\color{#1}\leftarrow\scalebox{0.65}{\(\displaystyle #2\)}}%
}
%------------------------------------------------------------
\renewcommand{\contentsname}{目次}
% \renewcommand{\figurename}{図}
\renewcommand{\figurename}{Fig.}
\renewcommand{\tablename}{表}
%------------------------------------------------------------
% MARK: User definitions
\NewDocumentCommand \RefLabelText{m}{\LabelText[type=gray]{参考}{#1}}
\NewDocumentCommand \ExampleLabelText{m}{\LabelText{例題}{#1}}
\NewDocumentCommand \ExampleSentence{O{.5} m m}{
    \vspace{#1\Cvs}
    \begin{minipage}{\textwidth}
    \begin{itemize}
		\LabelItem[type=gray]{例文}{#2}
		\LabelItem[type=trans]{}{\scriptsize #3}
    \end{itemize}
    \end{minipage}
    \vspace{.5\Cvs}
}
\NewDocumentCommand \ExampleSentenceItem{m m}{
    \begin{itemize}
		\LabelItem[type=gray]{例文}{#1}
		\LabelItem[type=trans]{}{\scriptsize #2}
    \end{itemize}
}
\NewDocumentEnvironment{Rule}{O{}}
    {\begin{Theorem}[原則]{#1}}
    {\end{Theorem}}
\makeatletter
\NewDocumentCommand \nobreaklist{}{\par\nobreak\@afterheading}% chktex 21
\NewDocumentCommand \nobreaklistend{}{\vspace{-.5\Cvs}}
\makeatother
\renewcommand{\labelenumi}{\textcircled{\scriptsize \theenumi}}

\end{pre}


\subsection{使用パッケージ}

\subsubsection{luatexja-preset}

フォントのプリセット.pixy-texでは「Noto Serif CJK, Noto Sans CJK」をデフォルトに設定.lua\LaTeX\ の本来の標準フォントは「原の味フォント」(haranoaji).
オプションの\EMMCBOX{deluxe}は明朝体3ウェイト,ゴシック体3ウェイト,丸ゴシック体が使用可能になります.また,\EMMCBOX{no-math}は数式フォントに影響しません.

\begin{pre}
\usepackage[noto-otf,no-math,deluxe]{luatexja-preset}
\end{pre}

\subsubsection{ifthen}

条件分岐や反復処理が可能になります\footnote{\url{https://qiita.com/zr_tex8r/items/71ae46c9c4e8cb575073}}.

\subsubsection{subfiles}

文書ファイルを分割できるようになります.

\subsubsection{amsmath, amssymb}

アメリカ数学会 (American Mathematical Society) によって開発された拡張パッケージです.数式の記述が容易になり,数式出力の品質が向上します.
\EMMCBOX{amssymb}は使用できる数学記号が増えます.

\subsubsection{mathtools}

amsmathの拡張パッケージ\footnote{\url{https://qiita.com/Yarakashi_Kikohshi/items/6362bf26828bfcfdb289}}.

\subsubsection{polynom}

多項式表示パッケージ\footnote{\url{http://xyoshiki.web.fc2.com/tex/polynom.html}}.

\subsubsection{unicode-math}

Lua\LaTeX\ で,Unicodeの数学記号を利用して数式を組むパッケージ.たとえば,数式の \(x\)は,このパッケージを使わない場合は\EMMCBOX{U+0078}の文字,このパッケージを使った場合は\EMMCBOX{U+1D465}の文字が使われる。\EMMCBOX{U+0078}はテキスト文字である\(x\)のコード,\EMMCBOX{U+1D465}は数式文字(斜体)である\(x\)のコードになる。

\subsubsection{titlesec}

章や節などの見出しのスタイルを変更するための拡張パッケージ.

\subsubsection{titleps}

ヘッダーやフッターなどのページスタイルを変更するための拡張パッケージ.

\subsubsection{hyperref}

ハイパーリンクを追加するための拡張パッケージ.

\subsubsection{bm}

太字の斜体でベクトルを表現するコマンド(\bm{a})

\subsubsection{stmaryrd}

数学記号を拡張するためのパッケージ\footnote{\url{http://xyoshiki.web.fc2.com/tex/stmaryrd.html}}.

\subsubsection{color}

色を指定できるcolorやtextcolorコマンド.指定できる文字の色は「red」「blue」「green」「yellow」「magenta」「white」「black」の7色です.また,色を追加する\EMMCBOX{definecolor}コマンドも使えます.

\subsubsection{xcolor}

colorパッケージを拡張します.オプション\EMMCBOX{x11names}を有効にしているので,指定できる色が追加されます\footnote{\url{https://www.sciencetronics.com/greenphotons/wp-content/uploads/2016/10/xcolor_names.pdf}}.

\subsubsection{multicol}

段組みを簡単に行える拡張パッケージ\footnote{\url{http://xyoshiki.web.fc2.com/tex/multicol.html}}.

\subsubsection{tikz}

作図パッケージ.

\subsubsection{tikz-euclide}

tikzの拡張パッケージ.

\subsubsection{booktabs}

表に横罫線を引きます.

\subsubsection{colortbl}

表の行や列に色をつける拡張パッケージ.

\subsubsection{here}

図を好きな位置に強制的に配置できる拡張パッケージ.

\begin{pre}
\begin{figure}[H]
...
\end{figure}
\end{pre}

\subsubsection{pgtplots}

プロットを作成するための拡張パッケージ.

\subsubsection{wrapfig}

図と文章を並列に表示させる拡張パッケージ.

\subsubsection{caption}

キャプションの表示を調整する拡張パッケージ.

\subsubsection{subcaption}

複数の図がある場合に,それぞれのキャプションを追加する拡張パッケージ.

\subsubsection{framed}

途中で改ページを可能にするフレームを作成する拡張パッケージ.

\subsubsection{mdframed}

framedの拡張パッケージ.細かいカスタマイズが可能です.

\subsubsection{tcolorbox}

カスタマイズが可能なフレームを作成する拡張パッケージ.

\subsubsection{varwidth}

スマートなminipageです.minipageは実効幅ですが,varwidthは最大幅を指定して,最小の幅に調整します.

\subsubsection{listings}

ソースコードを挿入するためのコマンドを提供する拡張パッケージ.

\subsubsection{verbatim}

入力したとおりの文字を出力する拡張パッケージ.

%------------------------------------------------------------
\renewcommand{\arraystretch}{1}
\setlength{\arrayrulewidth}{1pt}
\arrayrulecolor{black}
\end{document}