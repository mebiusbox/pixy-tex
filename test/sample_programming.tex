\documentclass[a4paper,10pt]{ltjsarticle}
% \documentclass[a4paper,10pt]{ltjsreport}
% \documentclass[a4paper,10pt]{ltjsbook}
% \usepackage[noto-otf,no-math,deluxe]{luatexja-preset}
\title{\TeX\ 文書テンプレート}
\author{mebiusbox software}
\begin{document}
	\maketitle
    % \chapter{\TeX\ テンプレート概要}
    \section{Register}

	\count0=42
	The value is now '\the\count0'
	\def\macro{-123456}
	\count0=\macro
	The value is now '\the\count0'


	1. \count0=42 % a white space after the number aborts the reading process. It is discarded.
	The value is now `\the\count0'.
	2. The following code will absorb the `3' of '3.':
	\def\macro{1234}
	\count0=\macro % a white space after a macro will be absorbed by TeX, so this is wrong.
	3. The value is now `\the\count0'.
	4. Use \textbackslash relax after an assignment to end scanning:
	\count0=\macro\relax
	5. The value is now `\the\count0'.

	\section{edef}

	\def\a{3}
	\def\b{2\a}
	\def\c{1\b}
	\def\d{value=\c}
	\message{Macro `d' is defined to be `\meaning\d'}
	\edef\d{value=\c}
	\message{Macro `d' is e-defined to be `\meaning\d'}
	\expandafter\def\expandafter\d\expandafter{\c}
	\message{Macro `d' is defined to be `\meaning\d' using expandafter}
\end{document}
